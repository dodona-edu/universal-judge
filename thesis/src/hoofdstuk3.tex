\chapter{De universele judge}\label{ch:de-universele-judge}

\lettrine{H}{et} antwoord op de onderzoeksvraag uit het vorige hoofdstuk manifesteert zich als de \term{universele judge}.
Deze judge voor het Dodona-platform kan dezelfde oefening in meerdere talen evalueren.
Dit hoofdstuk licht de werking en implementatie van deze judge toe, beginnend met een algemeen overzicht, waarna elk onderdeel in meer detail besproken wordt.

TODO: terminologie uitleggen (oplossing, evaluatie, \ldots)
Een groot deel hiervan zal waarschijnlijk uitgelegd zijn bij de werking van Dodona.

\section{Overzicht}\label{sec:overzicht}

\begin{figure}
    \begin{adjustbox}{width=\textwidth}
        % Define the styles for various components in the architectural diagram.
\tikzstyle{node}=[draw, minimum height=1cm, text width=4cm, align=center, fill=white]
\tikzstyle{state}=[node, rectangle]
\tikzstyle{process}=[node, rectangle, rounded corners=0.5cm]
\tikzstyle{named}=[text=ugent-blue,font=\sffamily\scshape,align=center,text width=4cm]

\begin{tikzpicture}[node distance=1cm and 2.5cm]

    \node[state] (input) at (0,0) {Start};
    \node[process, below = of input] (generation) {Genereren \\ van code};
    % Special node to connect the arrow.
    \node[right = of generation,minimum height=1cm,xshift=-3.5cm] (gen1) {};
    \node[left = of generation,minimum height=1cm,xshift=3.5cm] (gen2) {};
    \node[state, right = of generation] (code) {Uitvoeringscode};
    \node[process, below = of code] (execution) {Uitvoeren code};
    \node[state, below = of execution] (execution state){Uitvoer};
    \node[state, left = of execution state] (core state) {Uitvoer};
    \node[state, left = of core state] (evaluation state) {Uitvoer};
    \node[state, below = of evaluation state] (custom evaluation code) {Evaluatiecode};
    \node[process, below = of custom evaluation code] (custom evaluation) {Evaluatie};
    \node[process, right = of custom evaluation] (core evaluation) {Evaluatie};
    \node[state, below = of execution state] (execution evaluation code) {Evaluatiecode};
    \node[process, below = of execution evaluation code] (execution evaluation) {Evaluatie};
    \node[state, below = of core evaluation] (feedback) {Resultaat};

    \node[named, below = of feedback] (core name) {kernkader};
    \node[named, left = of core name] (evaluation name) {evaluatiekader};
    \node[named, right = of core name] (execution name) {uitvoeringskader};

    \begin{scope}[on background layer]

        % Draw these first to ensure they are in the background.
        \path[draw,dashed,very thick,lightgray] (-3.5,0.5) -- (-3.5,-14.6);
        \path[draw,dashed,very thick,lightgray] (3.5,0.5) -- (3.5,-14.6);

        \draw [->] (input) -- (generation);
        \draw[->] (generation) -- (code);
        \draw[->, dashed] (gen2) |- (custom evaluation code);
        \draw[->, dashed] (gen1) |- (execution evaluation code);
        \draw[->] (code) -- (execution);
        \draw[->] (execution) -- (execution state);
        \draw[->] (execution state) -- node[above] {Serialisatie} ++ (core state);
        \draw[->] (core state) -- node[above] {Deserialisatie} ++ (evaluation state);
        \draw[->] (core state) -- (core evaluation);
        \draw[->] (evaluation state) -- (custom evaluation code);
        \draw[->] (execution state) -- (execution evaluation code);
        \draw[->] (custom evaluation code) -- (custom evaluation);
        \draw[->] (execution evaluation code) -- (execution evaluation);

        \draw[->] (core evaluation) -- (feedback);
        \draw[->] (custom evaluation) -- (feedback);
        \draw[->] (execution evaluation) -- (feedback);

    \end{scope}


\end{tikzpicture}

    \end{adjustbox}
    \caption{Schematische voorstelling van de opbouw van de universele judge.}
    \label{fig:universal-judge}
\end{figure}

\Cref{fig:universal-judge} toont de opbouw van de judge op schematische wijze.
De judge kan worden opgedeeld in drie gebieden, volgens hun verantwoordelijkheid:

\begin{enumerate}
    \item Het evaluatieproces, dat de verkregen resultaten interpreteert en beoordeelt.
    \item Het kernproces, dat zorgt voor de coördinatie tussen de andere processen, alsook de basistaken vervult.
          Dit proces is dan ook het start- en eindpunt van een evaluatie.
    \item Het uitvoeringsproces, dat de code van de student uitvoert om zo resultaten te bekomen.
\end{enumerate}

Daarnaast is het \term{testplan} ook van belang, dat de evaluatiecode definieert.

\section{Beschrijving van een oefening}

Uitleg over het testplan.

\section{Uitvoeren van de oplossing}

\section{Evalueren van een oplossing}

