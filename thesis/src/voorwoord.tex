% Special KOMAScript version of \chapter¨* that is added to the repository.
\addchap{Voorwoord}

In de zomer van 2011 belandde ik in het ziekenhuis met een ontstoken appendix.
Daar nam ik drie beslissingen.
Ik maakte komaf met het vak "Informatica" dat ik als jonge docent geërfd had.
Daarin leerden we studenten uit de faculteit Wetenschappen werken met de Microsoft Office-toepassingen, taken automatiseren met \emph{Visual Basic for Applications} (\acronym{VBA}), literatuur opzoeken en verwerken.
Te veel van het goede.
Dus gingen we ons alleen nog maar concentreren op programmeren.
We doopten het nieuwe vak "Programmeren".
\acronym{VBA} ging op de schop en werd vervangen door Python.
Toen nog \version{Python 2}.
Maar dat zouden we een jaar later al inruilen voor \version{Python 3}.
We besloten ook in te zetten op automatische feedback.
Op elke ingediende oplossing van onze studenten.
Geïnspireerd op wat we deden in de Vlaamse Programmeerwedstrijd.

Google bracht ons bij de Sphere Online Judge (\acronym{SPOJ}).
Een platform om programmeerwedstrijden te organiseren.
Die keuze werd vooral ingegeven door het feit dat we aan \acronym{SPOJ} een eigen judge konden toevoegen.
Een script om ingediende oplossingen te beoordelen.
Dat was nodig.
Deelnemers aan programmeerwedstrijden krijgen typisch slechts minimale feedback over de correctheid van hun ingediende oplossingen.
In een onderwijscontext wilden we echter inzetten op maximale feedback.
Leren programmeren is voor de meeste studenten al uitdagend genoeg en als lesgever kunnen we onmogelijk 24/24 en 7/7 klaar staan om elke student die vast zit tips te geven.

We vernoemden onze Python-judge naar de orakelpriesteres in het heiligdom van Apollo Pythios te Delphi -- Pythia.
De ontwikkeling ervan ging hand in hand met het uitwerken van een reeks programmeeroefeningen.
Veel tijd voor \english{prototyping} was er niet.
Wel veel \english{trial and error} door de beperkte documentatie van \acronym{SPOJ}.
Bij de start van het academiejaar hadden we een "werkende" judge en een honderdtal uitgewerkte oefeningen.
Met dank aan Karsten Naert.
We waren blij dat we ervoor gegaan waren en dat we het gehaald hadden.
Maar we waren vooral heel enthousiast om aan het avontuur te beginnen.

Tijdens die eerste jaren bleven we vooral in onze eigen \english{downtime} aan Pythia sleutelen.
Bugs fixen.
Met vallen en opstaan leren wat een judge moet doen om code van studenten te evalueren.
Hoe we nog beter feedback kunnen geven.
Dat alles wat kan misgaan bij het uitvoeren en testen van software die studenten schrijven ook misgaat.
Vroeger eerder dan later.
Onderweg introduceerden we ook ontzettend veel nieuwe \english{features}.
Gebaseerd op vragen die studenten ons stellen, observaties tijdens werkcolleges en een constante stroom aan oplossingen die studenten indienen.
Meer dan 120.000 tijdens het eerste academiejaar.
Gedreven om de grenzen van automatische feedback te verleggen bij het opstellen van nieuwe oefeningen.
Veel nieuwe oefeningen.
Gemiddeld meer dan 60 per semester.
Een huzarenstukje als je bedenkt dat het een halve tot een hele dag vraagt om een nieuwe oefening
te bedenken en op te stellen.

Voor het uitwerken van ideeën die tijdens de lesweken waren blijven liggen, konden we tijdens de zomermaanden altijd beroep doen op jobstudenten.
Cracks uit de opleiding informatica.
Financieel ondersteund door onderwijsinnovatieprojecten van de faculteit Wetenschappen.
Met Felix Van der Jeugt als contributor \#1.
Op die manier tekenden de contouren van Pythia zich steeds verder af.

Feedback stoppen bij het eerste gefaalde testgeval.
Testgevallen groeperen in contexten.
Oefeningen omzetten naar \version{Python 3}.
Toch beter om feedback te geven op alle testgevallen.
\english{Diff highlighting} bij het vergelijken van verwachte en gegenereerde resultaten.
\english{Pretty printing} bij het vergelijken van geneste datastructuren.
Learning analytics over ingediende oplossingen.
Aftelklok tot volgende deadline toevoegen.
Linting.
Stack trace koppelen aan ingediende code.
Dynamisch schakelen tussen \english{unified} en \english{split} weergave van \english{diff}.
Geprogrammeerde evaluatie.
Gewichten toekennen aan testgevallen.
Verborgen testgevallen.
Moeilijkheidsgraad weergeven bij oefeningen.
Beperkingen instellen op de lengte van gegenereerde resultaten.
Objecten afzonderlijk vergelijken op type en inhoud.
Stack trace opkuisen.
Worstelen met afhandeling bij overschrijding van geheugenlimiet.
Oefeningen exporteren uit \acronym{SPOJ}\@.
I18n van oefeningen en feedback.
Script om oefeningen vanuit git repo bij te werken in \acronym{SPOJ}\@.
Via breekbare \acronym{POST} requests bij gebrek aan \acronym{API}\@.
Script om oefeningen aan nieuwe versie van judge te koppelen.
Webhooks voor automatische synchronisatie met \acronym{SPOJ}\@.
Contexten groeperen in tabs.
Onderscheid maken tussen stdout, stderr, returnwaarde en exceptions in doctests.
Feedback toevoegen aan die individuele uitvoerkanalen.
Expliciete boodschappen bij ontbrekende of overtollige newlines.
Specifieke weergave van \english{multiline strings}.
Vergelijken van tekstbestanden.
Koppeling met Indianio.
Oefeningen exporteren als pdf.
Integratie van de Online Python Tutor.
\english{Lucky shot} want bleek een \english{killer app} voor studenten.
Maar een worsteling om draaiende te houden.
Optie om geslaagde testgevallen te verbergen.
Overzicht voor studenten met hun status voor de opgelegde oefeningen.

\english{Fast forward} naar de zomer van 2016.
Bart Mesuere -- kersverse doctor in de informatica -- had goed nieuws gekregen dat hij in oktober kon starten als postdoc bij het \acronym{FWO}.
Als tijdverdrijf zocht hij een interessant hobbyproject om nog eens een nieuwe web app te programmeren.
Even weg van Unipept.
Het kindje uit zijn doctoraat.
Een jaar eerder had ik op het einde van de paasvakantie inderhaast een \english{client side} judge voor JavaScript geschreven.
Donderdag begonnen met de implementatie.
De dinsdag daarop al gebruikt tijdens een werkcollege.
Een jaar later schreef Bart een webapp rond de JavaScript-judge.
Vermoedelijk in hetzelfde korte tijdsbestek waarin ook de judge ontwikkeld was.
Een laatste stuiptrekking van zijn leven als assistent.
De app gaf studenten overzicht over hun ingediende oplossingen voor een aantal reeksen opgelegde oefeningen.
Hij noemde hem Dodona.
Naar de tweede grootste orakelplaats van Griekenland.
Na Delphi waar het orakel Pythia gehuisvest was.

In die zomer van 2016 -- vlak voor de Gentse Feesten -- besloten we een online leerplatform te ontwikkelen.
Waarin we onze programmeercursussen zouden kunnen aanbieden.
Weg van \acronym{SPOJ}.
Daarin waren onze uitbreidingen ondertussen één grote hack geworden.
Sommige beperkingen van \acronym{SPOJ}.
Daar hadden we mee leren leven.
Maar ze vormden een serieuze rem.
Een strak keurslijf waar we vanaf wilden.
Om de vele ideeën uit te werken die we nog wilden realiseren om activerend leren te bevorderen.
Al hadden we van die term nog nooit gehoord.
Evenmin als van \english{blended learning} of de \english{flipped classroom}.
We deden het al zonder dat we er erg in hadden.

Uit die as werd Dodona geboren.
Het leerplatform zoals we het nu kennen.
Design sterk beïnvloed door ervaringen uit voorgaande jaren.
Positief en negatief.
Aanbod van cursussen met een leerpad van oefeningen gegroepeerd in reeksen.
Strikte scheiding van verantwoordelijkheden tussen judges en programmeeroefeningen.
Aan de ene kant een generiek test framework voor een bepaalde programmeertaal.
Aan de andere kant specifieke testen voor het automatisch beoordelen van ingediende oplossingen volgens de specificaties van een oefening. 
Docker-containers als sandbox waarin judges een ingediende oplossing konden uitvoeren en beoordelen.
Eveneens een strikte scheiding tussen het genereren van feedback door de judge en het weergeven van feedback door Dodona.
Met een gestandaardiseerd \acronym{JSON} Schema om feedback vast te leggen.
We hielden ook vast aan beheer van content (judges en oefeningen) in externe git \english{repositories}.
Inclusief \english{webhooks} voor synchronisatie met Dodona.
Zo moest Dodona geen gebruikersinterface krijgen om content te beheren en behielden auteurs ook volledige controle over hun content.
Bonus: content kan eenvoudig toegevoegd worden aan meerdere Dodona-instanties.

Een universeel ontwerp waarmee we theoretisch gezien alle programmeertalen kunnen ondersteunen.
Strak plan.
Strak tijdschema.
Strakke regie.
Harm Delva als jobstudent mee in bad getrokken.
Bestaande Python- en JavaScript-judges overgezet naar Dodona.
Alle bestaande oefeningen overgezet.
Ondertussen al meer dan 600.
Half augustus.
Bart en Felix trekken naar de Ghent University Global Campus (\acronym{GUGC}) in Zuid-Korea.
Om er een zomercursus Python te doceren.
Ideaal om de bèta-versie van Dodona voor de leeuwen te gooien.

7 september 2016.
Mail van collega prof.\ dr.\ ir.\ Bart Dhoedt.
Faculteit Ingenieurswetenschappen en Architectuur.
Via via had hij vernomen dat we met iets bezig waren.
Of ze het ook mochten gebruiken.
Voor een programmeercursus met 400 eerstejaarsstudenten.
In het nieuwe academiejaar maakten ze de overstap van Java naar Python.
Plots kwam er dus meer in het vizier dan die ene cursus die we initieel voorzien hadden.

26 september 2016.
D-day.
Start van het nieuwe academiejaar aan de UGent.
Geboorte van Dodona.
Versie \version{1.0} uitgerold.
Blij dat we ervoor gegaan waren en dat we het gehaald hadden.
De spanning was te snijden.
Die maandagavond bekeek Bart de statistieken.
Hij verwedde erom dat we tegen het einde van de week 25.000 ingediende oplossingen zouden halen.
Ik ging de weddenschap aan.
Doe ik anders nooit.
We haalden meer dan 75.000 ingediende oplossingen.
Gegokt en verloren.
Maar bijzonder enthousiast om aan de volgende etappe van het avontuur te beginnen.

Het volgende academiejaar maakten al 40 cursussen aan UGent gebruik van Dodona.
In 2018 waren we laureaat van de UGent Minervaonderscheiding voor onderwijs.
Voor onze activerende manier van lesgeven en voor het ontwikkelen van het innovatieve EdTech platform Dodona.
Beide nauw met elkaar verstrengeld.
Datzelfde jaar gaf UGent een impuls aan de verdere uitbouw van Dodona door ons een interfacultair onderwijsinnovatieproject toe te kennen.
Daarin waren 7 faculteiten en de \acronym{GUGC} betrokken.
Samen met de Directie Informatie- en Communicatietechnologie (\acronym{DICT}) van de UGent.
Voor logistieke ondersteuning.
Eventjes pauze.
Hoe kon ik in voorgaande nog vergeten vermelden dat zij ons al jarenlang de nodige \acronym{ICT}-infrastructuur geboden hebben.
Een dikke merci aan Johan Van Camp en zijn team is hier dus zeker op zijn plaats.
Soit.
Met de projectfinanciering konden we voor het eerst op zoek naar een fulltime softwareontwikkelaar voor Dodona.
Charlotte Van Petegem was bij ons haar masterproef aan het afwerken.
Zij zag het zitten.
Wij zagen het zitten.
Vanaf dan konden we beginnen spreken van team Dodona.

Kunnen we Dodona ook aan de Universiteit Hasselt gebruiken?
We spreken december 2017.
De vraag kwam van collega prof.\ dr.\ Frank Neven.
\acronym{SAML}-authenticatie opgezet.
In februari 2018 ging hij met zijn studenten aan de slag.
Snel daarna begon dezelfde vraag ook te komen van andere Vlaamse universiteiten en hogescholen.

Zouden ook leerkrachten uit het secundair onderwijs in Dodona een co-teacher kunnen vinden?
De vraag kwam initieel van leerkrachten die deelnamen aan nascholingen rond programmeren en algoritmen van prof.\ dr.\ Veerle Fack.
Daarin was ze ook Dodona beginnen gebruiken.
Hoe pakken we authenticatie aan voor het secundair onderwijs?
Rien Maertens \english{to the rescue}.
Als masterstudent kwam hij enkele weken team Dodona vervoegen -- in volle blok nota bene -- om OAuth-koppelingen te voorzien met Office 365- en Smartschool-accounts van scholen.
Op 3 september 2018 gingen drie pilootscholen aan de slag met Dodona.
Op 5 september kregen we een enthousiaste mail van Dominiek Vandewalle.
Leerkracht \acronym{ICT} aan de Sint-Paulusschool campus College Waregem.
Eerste programmeerles gegeven op woensdagvoormiddag.
Op woensdagmiddag waren zijn leerlingen thuis al oplossingen aan het indienen voor de meer uitdagende oefeningen.

Uitnodiging van Annick Van Daele.
Voorzitster van 2Link2.
De vakvereniging voor leraren informatica en \acronym{STEM}.
Of we ons op 23 juni 2019 konden vrijmaken om een workshop rond Dodona te geven?
Wegens veel interesse werden het twee workshops.
Allebei volgeboekt.
We maakten van de gelegenheid gebruik om alle scholen toegang te geven tot Dodona.
Gratis uiteraard.
Kunnen we ook toegang krijgen met de Google G-Suite-accounts van onze school?
Charlotte?
\english{Consider it done}.
We hadden de tamtam van het secundair onderwijs duidelijk onderschat.
Tien maanden later waren al ruim 300 secundaire scholen op Dodona geregistreerd.
Sommige waren al met leerlingen aan de slag.
Anderen waren nog wat aan het verkennen.
Toen kwam corona.
Plots ging iedereen op zoek naar digitale ondersteuning voor het onderwijs.
Twee maanden later stond de teller boven de 450 scholen.

Vlak voor de zomer van 2019 kreeg Bart een senior postdoc mandaat bij het \acronym{FWO}\@.
Voor Unipept kreeg hij naast Felix ook Pieter Verschaffelt als \english{wingman}.
Die had ook een doctoraatsmandaat van het \acronym{FWO} gekregen.
Charlotte en Rien kozen na de zomer voor een doctoraatsproject binnen het Dodona-ecosysteem.
Verzekering dat Unipept en Dodona alvast voor de komende jaren in goede handen waren.
Bart kon zijn energie over de twee projecten verdelen.

Ondertussen waren er ten huize Dodona ook steeds meer judges voor andere programmeertalen bijgekomen.
Prof.\ dr.\ Christophe Scholliers en Robbert Gurdeep Singh schreven judges voor Haskell en Prolog.
Dr.\ Niels Neirynck voor bash.
Dr.\ Dieter Mourisse voor C\#\@.
Charlotte voor R\@.
Felix voor Java.
Later met hulp van masterstudent Pieter De Clercq.
Die samen met studiegenoot Tobiah Lissens ook een JetBrains-plugin voor Dodona ontwikkelde.

Vaststelling.
Al die judges waren ontwikkeld door de \english{inner circle} rond Dodona.
Allemaal medewerkers van de opleiding informatica.
Blijkbaar voelen velen zich geroepen om programmeeroefeningen uit te werken op basis van een bestaande judge.
Weinigen durven het aan om zelf een judge te ontwikkelen.
Het helpt om wat af te weten van de programmeertaal waarvoor de beoordeling moet gebeuren.
Maar toegegeven.
Het bouwen van een judge is geen \english{rocket science}.
Al vraagt het wel wat doorzettingsvermogen om hem robuust te maken tegen weerbarstige code van studenten en rijke feedback te voorzien die typisch ook door andere judges geboden wordt.
Want -- vaststelling \#2 -- de meeste judges implementeren \latin{grosso modo} dezelfde features maar dan elk voor/in een specifieke programmeertaal.

Wacht eens even.
Judges lijken zeer sterk op elkaar?
Op de details van de programmeertaal na.
Dus schuilt er zekere routine in het schrijven van nieuwe judges?
Hoeveel programmeertalen zijn er?
Sorry, maar als informaticus haal ik mijn neus op om steeds hetzelfde te moeten doen.
Zouden we geen generieke judge kunnen schrijven?
Die feedback geeft op ingediende oplossingen voor programmeeroefeningen.
Ongeacht de programmeertaal waarin die oplossingen geschreven zijn.
\english{One judge to rule them all}, zeg maar.
Het idee begon te rijpen op het einde van Charlotte Van Petegem haar masterproef.
We werkten het ontwerp van een architectuur uit.
Testen uitvoeren loskoppelen van resultaten beoordelen.
De resultaten moeten we daarvoor kunnen serialiseren en deserialiseren.
Het uitvoeren kunnen we misschien veralgemenen door de kernels van Jupyter Notebooks te gebruiken.
Daarmee hebben we meteen ondersteuning voor tientallen programmeertalen.
Enter \tested{}.
Conceptueel dan toch.
We klopten aan bij het \acronym{FWO}\@.
Interessant!
Maar blijkbaar toch niet interessant genoeg.
Ook wat bedenkingen bij de haalbaarheid.

Enter Niko Strijbol.
Master in de informatica in wording.
Coding ninja \english{by night}.
Voor zijn masterproef zou hij er wel eens zijn vingers inzetten (tanden zijn alleen maar spreekwoordelijk).
Jupyter-kernels gingen snel op de schop.
De overhead om er afzonderlijk voor elke context één op te starten bleek een onoverkomelijke bottleneck.
Werken met een pool van kernels bracht ook geen zoden aan de dijk.
\english{No problemo}.
Dacht Niko.
Dan zetten we het testplan van een oefening maar statisch om naar uitvoerbare testcode.
Dat werkt ook.
Serialisatie/deserialisatie van waarden?
Grasduinen in bestaande technologieën leverde enkele interessante pistes op.
Maar niets dat uiteindelijk aan alle verwachtingen voldeed.
\english{No problemo}.
Dacht Niko.
Met een \english{template engine} moet dat toch ook lukken.
Jinja2?
Leek aanvankelijk een goede keuze.
Maar gaf uiteindelijke onvoldoende controle over witruimte.
Mako dan maar.
Hoeveel performantie verliezen we met de extra vertaalslag van een generiek testplan naar testcode voor een specifieke programmeertaal?
Met batchcompilatie en parallelle verwerking van contexten bleek \tested{} zelfs sneller dan de meeste judges die specifiek voor één programmeertaal een beoordeling uitvoeren.
Maar nu ben ik al te veel aan het verklappen.
In dit proefschrift lees je hoe alle puzzelstukjes in elkaar gevallen zijn.
Hopelijk beleef je evenveel plezier aan het lezen als wij gehad hebben bij het uitdenken, implementeren en configureren van \tested{}.

Wordt \tested{} de heilige graal voor het automatisch beoordelen van software in een educatieve context?
Daarvoor moet het framework zijn sporen op het slagveld nog verdienen.
Zal een generieke judge ooit alle features kunnen aanbieden als judges die specifiek op één programmeertaal gericht zijn?
Voer voor minstens nog een masterproef.
Gelukkig is er al een masterstudent informatica geïnteresseerd om die uitdaging aan te gaan.
We moeten ook nog meer oefeningen opstellen voor \tested{}.
Veel meer oefeningen.
Kunnen we bestaande oefeningen automatisch omzetten naar \tested{}?
Welke programmeertalen zullen er volgend jaar toegevoegd zijn aan \tested{}?
Hoeveel programmeertalen zal \tested{} over vier jaar ondersteunen?
Dat deel van de geschiedenis moeten we nog schrijven.
Maar ik ben blij dat we voor \tested{} gegaan zijn.
Trots op het prototype dat er nu al ligt.
Enthousiast om aan de volgende etappe van het avontuur te beginnen.
Samen met Niko die vanaf september team Dodona komt vervoegen.

\begin{flushright}
    Peter Dawyndt\\
    juni 2020
\end{flushright}
