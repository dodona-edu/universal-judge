\chapter*{Samenvatting}

Veel programmeeroefeningen zijn niet inherent gebonden aan een programmeertaal en kunnen opgelost worden in meerdere programmeertalen.
Het manueel vertalen van oefeningen van de ene programmeertaal naar de andere is een tijdrovende bezigheid.
De focus van deze thesis is de onderzoeksvraag "Is het mogelijk om oefeningen slechts één keer op een programmeertaalonafhankelijke manier te schrijven en toch oplossingen in verschillende programmeertalen te aanvaarden?".

Als antwoord hierop introduceert de thesis \tested{}, een prototype van een judge voor het Dodona-platform die dezelfde oefening in meerdere programmeertalen kan beoordelen.
Dodona is een online platform om oplossingen voor programmeeroefeningen van feedback te voorzien (zoals de juistheid van de oplossing) en wordt ontwikkeld aan de UGent.

Een kernaspect van \tested{} is het testplan.
Dit is een specificatie van hoe een oplossing voor een oefening beoordeeld moet worden.
Bij bestaande judges in Dodona gebeurt dit op een programmeertaalafhankelijke manier: jUnit in Java, een eigen formaat in Python, enz.
Het testplan bij \tested{} is programmeertaalonafhankelijk en wordt opgesteld in \acronym{JSON}.
Op deze manier wordt één testplan opgesteld, waarna de oefening kan opgelost worden in alle programmeertalen die \tested{} ondersteunt.
Momenteel zijn dat Python, Java, Haskell, C en JavaScript.

Het programmeertaalonafhankelijk zijn van het testplan verhindert niet dat een testplan geschreven kan worden dat bepaalde functionaliteit gebruikt die niet aanwezig is in elke programmeertaal.
Een voorbeeld hiervan is een testplan met klassen en object, dat enkel opgelost zal kunnen worden in objectgericht programmeertalen.
Als invoer voor een test ondersteunt het testplan \texttt{stdin}, functieoproepen en commandoargumenten.
De beoordeling gebeurt dan aan de hand van \texttt{stdout}, \texttt{stderr}, exceptions, returnwaarden, aangemaakte bestanden en de exitcode van de uitvoering.

Voor de vertaling van het testplan naar testcode in de programmeertaal van de oplossing gebruikt \tested{} een sjabloonsysteem genaamd mako, gelijkaardig aan sjabloonsystemen bij webapplicaties (\acronym{ERB} bij Rails, Blade bij Laravel, enz.).
Deze sjablonen worden opgesteld bij de configuratie van een programmeertaal in \tested{}, samen met een configuratieklasse, die details over de programmeertaal bevat, zoals hoe de programmeertaal gecompileerd moet worden (dit is optioneel) en hoe ze uitgevoerd moet worden.

Doordat het testplan onafhankelijk is van de programmeertaal kunnen bij het configureren van een nieuwe programmeertaal bestaande oefeningen zonder wijziging gebruikt opgelost worden in de nieuwe programmeertaal.
