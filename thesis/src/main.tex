%! Suppress = Makeatletter
%! Suppress = MultipleIncludes
%! Author = strij
%! Date = 12/01/2020

% Preamble
\documentclass[
    12pt,
    parskip=half,
    titleUppercase=false,
    titleUnderline=false,
    uppercase=false,
    captions=tableheading,
    bibliography=totoc
]{ugent2016-report}

% Packages
% Babel uses the last language as main language of the file.
\usepackage[latin,british,dutch,shorthands=off]{babel}
\usepackage{unicode-math}
\usepackage{lettrine}
\usepackage{microtype}
\usepackage[backend=biber,style=ieee]{biblatex}
\usepackage{imakeidx}
\usepackage{markdown}
\usepackage[newfloat]{minted}
\usepackage{csquotes}
\usepackage{hyperref} % Laad hyperref voor cleverref.
\usepackage[dutch,noabbrev]{cleveref}
\usepackage{adjustbox}
\usepackage{amsfonts}
\usepackage{standalone}
\usepackage{newunicodechar}
\usepackage{placeins}
\usepackage{multirow}
\usepackage{threeparttable}
\usepackage[labelsep=quad]{caption}
\usepackage{pdfpages}
%\usepackage{showframe}
\usepackage{tikz}
\usetikzlibrary{shapes,arrows,positioning,backgrounds,calc,intersections}

\renewcommand{\mkbegdispquote}[2]{\itshape}

% Gebruik dit om de afbreekpatronen zichtbaar te maken.
%\usepackage{showhyphens}

% Engelse bastaardwoorden die geen patroon hebben.
% De Nederlandse afbreekpatronen zijn sowieso oud.
\hyphenation{judge} % Geen afbreekpunten in het Engels.

% Fix woordafbrekingen met een trema.
\makeatletter
\newunicodechar{ë}{\@trema e}
\makeatother

\newfontfamily{\fallbackfont}{DejaVu Sans Mono}[Scale=1.05]
\DeclareTextFontCommand{\textfallback}{\fallbackfont}
\newunicodechar{⏎}{\textfallback{⏎}}

\hypersetup{
    linkcolor  = ugent-blue,
    citecolor  = ugent-blue,
    urlcolor   = ugent-blue,
    colorlinks = true,
}

\markdownSetup{
    headerAttributes = true,
    hybrid = true,
    fencedCode = true,
    tightLists = false,
    definitionLists = true
}

\definecolor{bg}{rgb}{0.98,0.98,0.98}
\setminted{
    bgcolor=bg,
    linenos,
    breaklines
}
\setmintedinline{bgcolor={}}

% Gebruik tabelcijfers in tabellen
\AtBeginEnvironment{tabular}{\addfontfeature{Numbers={Lining,Monospaced}}}



\title{TESTed: one judge to rule them all}
\subtitle{Een universele judge voor het beoordelen van software in een educative context}
\author{\large Niko Strijbol}
\studentnumber{01404620}

\academicyear{2019 -- 2020}

\titletext{%
Promotoren: prof.\ dr.\ Peter Dawyndt, dr.\ ir.\ Bart Mesuere \\%
Begeleiding: Charlotte Van Petegem\\%
\\%
{\small Masterproef ingediend tot het behalen van de academische graad van\\%
Master of Science in de Informatica%
}%
}

\newfontfamily\lsi[Scale=2.5]{Libertinus Serif Initials}
\renewcommand*{\LettrineFont}{\lsi}
\renewcommand{\lettrine}[3][]{#2#3}

%%%%%%
% Taalgerelateerde zaken
%%%%%%
% Zorg ervoor dat aanhalingstekens correct zijn.
\MakeOuterQuote{"}
% Vertaal "listings" als codefragmenten, niet als listing.
\SetupFloatingEnvironment{listing}{name=Codefragment}
\crefname{listing}{codefragment}{codefragmenten}

\addbibresource{main.bib}

% Commando's voor termen
\newcommand*{\term}[1]{\textit{#1}\index{#1}}
\newcommand*{\termen}[1]{\foreignlanguage{british}{\textit{#1}}\index{#1}}
\newcommand*{\english}[1]{\foreignlanguage{british}{\textit{#1}}}
\newcommand*{\latin}[1]{\foreignlanguage{latin}{\textit{#1}}}
\newcommand*{\acronym}[1]{{\addfontfeature{Letters=UppercaseSmallCaps}#1}}
\newcommand*{\version}[1]{{\addfontfeature{Numbers={Lining,Monospaced}}#1}}
\newcommand*{\unit}[1]{{\addfontfeature{Numbers={Lining}}#1}}
\newcommand*{\tested}{\acronym{TESTed}}

% Enable for index generation
%\makeindex

% Fix wrong hyphens
\babelhyphenation[dutch]{be-oor-de-lings-om-ge-ving}
\babelhyphenation[dutch]{uit-voe-rings-om-ge-ving}
\babelhyphenation[dutch]{Python}

% We don't want a big TOC, but do want it in the bookmarks
\hypersetup{bookmarksdepth=3}

% Document
\begin{document}

    \setmainfont{UGent Panno Text}

    \maketitle

    \setmainfont[Ligatures=TeX,Numbers=OldStyle,Contextuals=Alternate]{Libertinus Serif}
    \setsansfont[Ligatures=TeX,Numbers=OldStyle,Contextuals=Alternate]{Libertinus Sans}
    \setmonofont[Scale=MatchLowercase,Contextuals={Alternate}]{Jetbrains Mono}
    \setmathfont{Libertinus Math}

    \pagenumbering{roman}
    
    \chapter*{Samenvatting}

Dodona is een online leerplatform om programmeeroefeningen op te lossen.
Het voorziet oplossingen in realtime van feedback, bijvoorbeeld over de correctheid van de ingediende oplossing.
Veel oefeningen zijn niet inherent gebonden aan een programmeertaal en zouden opgelost kunnen worden in meerdere programmeertalen.
Het manueel configureren van oefeningen voor verschillende programmeertalen is echter een tijdrovende bezigheid.
In deze masterproef focussen we daarom op de onderzoeksvraag of het mogelijk is om oefeningen slechts één keer op een programmeertaalonafhankelijke manier te configureren en toch oplossingen in verschillende programmeertalen te beoordelen.
Als antwoord hierop introduceren we \tested{}, een prototype van een judge voor Dodona, die dezelfde oefening in meerdere programmeertalen kan beoordelen.

Een kernaspect van \tested{} is het testplan: een specificatie van hoe een oplossing voor een oefening moet beoordeeld worden.
Bij bestaande judges in Dodona gebeurt dit op een programmeertaalafhankelijke manier: jUnit in Java, doctests in Python, enz.
Het testplan bij \tested{} is programmeertaalonafhankelijk en wordt opgesteld in \acronym{JSON}.
Op deze manier wordt één specificatie opgesteld, waarna de oefening kan opgelost worden in alle programmeertalen die \tested{} ondersteunt.
Momenteel zijn dat Python, Java, Haskell, C en JavaScript.

Een testplan bestaat uit een reeks testgevallen, waarin telkens een invoer gekoppeld wordt aan een uitvoer.
Als invoer ondersteunt het testplan \texttt{stdin}, functieoproepen, commandoargumenten en bestanden.
De beoordeelde uitvoer bestaat uit \texttt{stdout}, \texttt{stderr}, exceptions, returnwaarden, aangemaakte bestanden en de exitcode.
Een programmeertaalonafhankelijk testplan betekent niet dat we geen specificatie kunnen schrijven die bepaalde functionaliteit gebruikt die niet aanwezig is in elke programmeertaal.
Een voorbeeld is een testplan met klassen, dat enkel opgelost zal kunnen worden in objectgerichte programmeertalen.

Voor de vertaling van het testplan naar testcode in de programmeertaal van de oplossing gebruikt \tested{} een sjabloonsysteem genaamd Mako, gelijkaardig aan sjabloonsystemen die gebruikt worden bij webapplicaties (\acronym{ERB} bij Rails, Blade bij Laravel, enz.).
Deze sjablonen worden eenmalig opgesteld bij de configuratie van een programmeertaal in \tested{}.
Ze worden bijvoorbeeld gebruikt voor het vertalen van expressies en statements uit het testplan.

We trekken volgende conclusies uit de ontwikkeling van \tested{}:

\begin{enumerate}
    \item Het opstellen van een programmeertaalonafhankelijke specificatie voor een oefening is haalbaar, zonder verlies aan functionaliteit voor de oefeningen die beoordeeld kunnen worden en zonder significant performantieverlies tijdens het beoordelen.
    \item Nieuwe programmeertalen kunnen snel toegevoegd worden aan \tested{}, zonder impact op bestaande oefeningen.
    \item We zijn van mening dat het de moeite loont om \tested{} uit te bouwen tot een volwaardige judge voor Dodona en te onderzoeken hoe we aan Dodona ondersteuning voor oefeningen in meerdere programmeertalen kunnen toevoegen.
\end{enumerate}


    \includepdf[pages=-]{extended.pdf}
    
    \chapter*{Vulgariserende samenvatting}

In onze maatschappij wordt technologie, en informatica in het bijzonder, alsmaar belangrijker.
In steeds meer sectoren worden problemen opgelost met behulp van digitale apparatuur.
Het is daarom belangrijk dat iedereen een basis digitale geletterdheid aangeleerd krijgt.
Het is niet voldoende om te kunnen werken met de programma's en technologie van vandaag.
Wie weet of programma's zoals Word of PowerPoint binnen twintig jaar nog relevant zijn?
De technologie verandert snel, waardoor het nodig is een begrip van de onderliggende systemen te hebben.

Zo komen we uit bij het begrip \emph{computationeel denken}.
Computationeel denken is een breed begrip, maar een goede definitie is het oplossen van problemen met behulp van de computer.
Het gaat om het vertalen van het probleem uit de "echte wereld" naar de informaticawereld, zodat het probleem kan begrepen worden door een computer.
Dit computationeel denken is recent ook opgenomen in de eindtermen van zowel het basisonderwijs als het secundair onderwijs.

Programmeren is een goede manier om computationeel denken aan te leren aan studenten.
Studenten ervaren programmeren echter vaak als moeilijk.
Het spreekwoord "oefening baart kunst" indachtig, denken we dat het maken van veel oefeningen een goede manier is om programmeren onder de knie te krijgen.
Het aanbieden van veel oefeningen leidt tot uitdagingen voor de lesgevers.

Ten eerste moeten lesgevers geschikte oefeningen opstellen, aangepast aan het niveau van de studenten.
Oefeningen moeten rekening houden met welke concepten studenten al kennen, hoeveel werk het kost om ze op te oplossen, enzovoort.
Een deel van de oefeningen is bedoeld voor beginners: mensen die leren programmeren.
Bij deze oefeningen is de exacte programmeertaal minder van belang: de onderliggende concepten en denkwijzen zijn het belangrijkste.
Een tweede soort oefeningen zijn wat we algoritmische oefeningen noemen.
Bij dergelijke oefeningen is het belangrijkste hoe een probleem wordt opgelost, niet in welke programmeertaal dit gebeurt.
Beide soorten oefeningen zijn dus geschikte kandidaten om ze in meerdere programmeertalen aan te bieden.

Ten tweede moeten de oplossingen voor deze oefeningen voorzien worden van kwalitatieve feedback.
Enkel en alleen oefeningen oplossen is niet voldoende: goede feedback laat toe dat studenten hun vaardigheden verbeteren, doordat ze een idee krijgen wat beter kon of waar ze fout zaten.
Hiervoor wordt vaak gebruikgemaakt van een platform voor programmeeroefeningen, dat een eerste vorm van automatische feedback geeft, zoals de correctheid van de ingediende oplossing voor een oefening.
Aan de onderzoeksgroep Computationele Biologie van de UGent is hiervoor het Dodona-platform ontwikkeld.
Om zo'n platform te kunnen gebruiken is het van belang dat er goede testen geschreven worden voor de oefeningen: enkel zo is een automatische beoordeling nuttig.

Bij het aanbieden van een programmeeroefening in meerdere programmeertalen is het veel werk om alle testen voor die oefening om te zetten van de ene programmeertaal naar de andere, ook al zijn veel oefeningen niet programmeertaalafhankelijk.
Veel programmeeroefeningen zijn eenvoudig en komen neer op een lijst sorteren, woorden zoeken in een bestand, iets berekenen op basis van gegevens enzovoort.
Deze taken kunnen in elke programmeertaal.
Dit omzetten van de testen is bovendien saai en repetitief werk.

In deze masterproef zoeken we hier een oplossing voor: is het mogelijk om een oefening slechts één keer op een programmeertaalonafhankelijke manier te schrijven en toch oplossingen in verschillende programmeertalen te beoordelen?

Als onderdeel van het antwoord op deze vraag hebben we \tested{} ontwikkeld: een prototype van een judge voor het Dodona-platform (een judge in Dodona is het onderdeel dat verantwoordelijk is voor het beoordelen van een ingediende oplossing).
In \tested{} moet de lesgever voor een oefening een programmeertaalonafhankelijk testplan opstellen.
Dit testplan bevat de specificatie van hoe een ingediende oplossing moet beoordeeld worden.
Daarna vertaalt \tested{} dit testplan naar de verschillende programmeertalen, waardoor één oefening kan opgelost worden in de programmeertalen die \tested{} momenteel ondersteunt: Python, Java, JavaScript, Haskell en C.\@

Daarnaast hebben we er ook aandacht aan besteed om het toevoegen van nieuwe programmeertalen aan \tested{} zo eenvoudig mogelijk te houden.
Als we een nieuwe programmeertaal toevoegen aan \tested{}, zullen bestaande oefeningen ook meteen opgelost kunnen worden in de nieuwe programmeertaal, zonder dat er iets aan het testplan van deze oefeningen moet veranderen.

    \chapter*{Dankwoord}\label{ch:dankwoord}

De masterproef als culminatie van de opleiding maakt het mijn aangename plicht om ieder te bedanken met wie ik in contact ben gekomen tijdens deze opleiding, in het bijzonder de professoren, assistenten en ondersteunend personeel, die deze opleiding mogelijk gemaakt hebben.
Daarnaast zijn er enkele personen wier steun een expliciete vermelding verdient.

In de eerste plaats wil ik mijn promotoren, prof.\ dr.\ Peter Dawyndt en dr.\ Bart Mesuere, en mijn begeleiding, Charlotte Van Petegem, bedanken.
Niet alleen voor het aanbieden van dit onderwerp, waarzonder deze masterproef niet zou bestaan, maar ook voor al hun tijd en moeite die ze in deze masterproef gestoken hebben.
Zonder de wekelijkse thesismeetings, het nalezen van de verschillende iteraties van de tekst en de vele suggesties zou de masterproef niet zijn wat ze nu is.
Ik wil ook prof.\ Dawyndt in het bijzonder bedanken voor het grondig nalezen van de tekst en het uitproberen van wat ik geschreven heb.

Verder wil ik de leden van Zeus \acronym{WPI} bedanken, die ervoor zorgden dat er in de kelder altijd wel iets anders te doen was dan werken aan de masterproef.
Het is jammer dat ik hen in het tweede semester heb moeten missen.

Tot slot wil ik mijn familie bedanken voor hun onvoorwaardelijke steun en gezelschap, waardoor het ook thuis aangenaam werken was.

\begin{flushright}
    Niko Strijbol
\end{flushright}

    \chapter*{Toelating tot bruikleen}

De auteur geeft de toelating deze masterproef voor consultatie beschikbaar te stellen en delen van de masterproef te kopiëren voor persoonlijk gebruik.
Elk ander gebruik valt onder de bepalingen van het auteursrecht, in het bijzonder met betrekking tot de verplichting de bron uitdrukkelijk te vermelden bij het aanhalen van resultaten uit deze masterproef.

Niko Strijbol
\today

TODO: handtekening

    \addtocounter{tocdepth}{-1}
    \tableofcontents

    \newpage
    
    \pagenumbering{arabic}
    
    % Special KOMAScript version of \chapter¨* that is added to the repository.
\addchap{Voorwoord}

In de zomer van 2011 belandde ik in het ziekenhuis met een ontstoken appendix.
Daar nam ik drie beslissingen.
Ik maakte komaf met het vak "Informatica" dat ik als jonge docent geërfd had.
Daarin leerden we studenten uit de faculteit Wetenschappen werken met de Microsoft Office-toepassingen, taken automatiseren met \emph{Visual Basic for Applications} (\acronym{VBA}), literatuur opzoeken en verwerken.
Te veel van het goede.
Dus gingen we ons alleen nog maar concentreren op programmeren.
We doopten het nieuwe vak "Programmeren".
\acronym{VBA} ging op de schop en werd vervangen door Python.
Toen nog \version{Python 2}.
Maar dat zouden we een jaar later al inruilen voor \version{Python 3}.
We besloten ook in te zetten op automatische feedback.
Op elke ingediende oplossing van onze studenten.
Geïnspireerd op wat we deden in de Vlaamse Programmeerwedstrijd.

Google bracht ons bij de Sphere Online Judge (\acronym{SPOJ}).
Een platform om programmeerwedstrijden te organiseren.
Die keuze werd vooral ingegeven door het feit dat we aan \acronym{SPOJ} een eigen judge konden toevoegen.
Een script om ingediende oplossingen te beoordelen.
Dat was nodig.
Deelnemers aan programmeerwedstrijden krijgen typisch slechts minimale feedback over de correctheid van hun ingediende oplossingen.
In een onderwijscontext wilden we echter inzetten op maximale feedback.
Leren programmeren is voor de meeste studenten al uitdagend genoeg en als lesgever kunnen we onmogelijk 24/24 en 7/7 klaar staan om elke student die vast zit tips te geven.

We vernoemden onze Python-judge naar de orakelpriesteres in het heiligdom van Apollo Pythios te Delphi -- Pythia.
De ontwikkeling ervan ging hand in hand met het uitwerken van een reeks programmeeroefeningen.
Veel tijd voor \english{prototyping} was er niet.
Wel veel \english{trial and error} door de beperkte documentatie van \acronym{SPOJ}.
Bij de start van het academiejaar hadden we een "werkende" judge en een honderdtal uitgewerkte oefeningen.
Met dank aan Karsten Naert.
We waren blij dat we ervoor gegaan waren en dat we het gehaald hadden.
Maar we waren vooral heel enthousiast om aan het avontuur te beginnen.

Tijdens die eerste jaren bleven we vooral in onze eigen \english{downtime} aan Pythia sleutelen.
Bugs fixen.
Met vallen en opstaan leren wat een judge moet doen om code van studenten te evalueren.
Hoe we nog beter feedback kunnen geven.
Dat alles wat kan misgaan bij het uitvoeren en testen van software die studenten schrijven ook misgaat.
Vroeger eerder dan later.
Onderweg introduceerden we ook ontzettend veel nieuwe \english{features}.
Gebaseerd op vragen die studenten ons stellen, observaties tijdens werkcolleges en een constante stroom aan oplossingen die studenten indienen.
Meer dan 120.000 tijdens het eerste academiejaar.
Gedreven om de grenzen van automatische feedback te verleggen bij het opstellen van nieuwe oefeningen.
Veel nieuwe oefeningen.
Gemiddeld meer dan 60 per semester.
Een huzarenstukje als je bedenkt dat het een halve tot een hele dag vraagt om een nieuwe oefening
te bedenken en op te stellen.

Voor het uitwerken van ideeën die tijdens de lesweken waren blijven liggen, konden we tijdens de zomermaanden altijd beroep doen op jobstudenten.
Cracks uit de opleiding informatica.
Financieel ondersteund door onderwijsinnovatieprojecten van de faculteit Wetenschappen.
Met Felix Van der Jeugt als contributor \#1.
Op die manier tekenden de contouren van Pythia zich steeds verder af.

Feedback stoppen bij het eerste gefaalde testgeval.
Testgevallen groeperen in contexten.
Oefeningen omzetten naar \version{Python 3}.
Toch beter om feedback te geven op alle testgevallen.
\english{Diff highlighting} bij het vergelijken van verwachte en gegenereerde resultaten.
\english{Pretty printing} bij het vergelijken van geneste datastructuren.
Learning analytics over ingediende oplossingen.
Aftelklok tot volgende deadline toevoegen.
Linting.
Stack trace koppelen aan ingediende code.
Dynamisch schakelen tussen \english{unified} en \english{split} weergave van \english{diff}.
Geprogrammeerde evaluatie.
Gewichten toekennen aan testgevallen.
Verborgen testgevallen.
Moeilijkheidsgraad weergeven bij oefeningen.
Beperkingen instellen op de lengte van gegenereerde resultaten.
Objecten afzonderlijk vergelijken op type en inhoud.
Stack trace opkuisen.
Worstelen met afhandeling bij overschrijding van geheugenlimiet.
Oefeningen exporteren uit \acronym{SPOJ}\@.
I18n van oefeningen en feedback.
Script om oefeningen vanuit git repo bij te werken in \acronym{SPOJ}\@.
Via breekbare \acronym{POST} requests bij gebrek aan \acronym{API}\@.
Script om oefeningen aan nieuwe versie van judge te koppelen.
Webhooks voor automatische synchronisatie met \acronym{SPOJ}\@.
Contexten groeperen in tabs.
Onderscheid maken tussen stdout, stderr, returnwaarde en exceptions in doctests.
Feedback toevoegen aan die individuele uitvoerkanalen.
Expliciete boodschappen bij ontbrekende of overtollige newlines.
Specifieke weergave van \english{multiline strings}.
Vergelijken van tekstbestanden.
Koppeling met Indianio.
Oefeningen exporteren als pdf.
Integratie van de Online Python Tutor.
\english{Lucky shot} want bleek een \english{killer app} voor studenten.
Maar een worsteling om draaiende te houden.
Optie om geslaagde testgevallen te verbergen.
Overzicht voor studenten met hun status voor de opgelegde oefeningen.

\english{Fast forward} naar de zomer van 2016.
Bart Mesuere -- kersverse doctor in de informatica -- had goed nieuws gekregen dat hij in oktober kon starten als postdoc bij het \acronym{FWO}.
Als tijdverdrijf zocht hij een interessant hobbyproject om nog eens een nieuwe web app te programmeren.
Even weg van Unipept.
Het kindje uit zijn doctoraat.
Een jaar eerder had ik op het einde van de paasvakantie inderhaast een \english{client side} judge voor JavaScript geschreven.
Donderdag begonnen met de implementatie.
De dinsdag daarop al gebruikt tijdens een werkcollege.
Een jaar later schreef Bart een webapp rond de JavaScript-judge.
Vermoedelijk in hetzelfde korte tijdsbestek waarin ook de judge ontwikkeld was.
Een laatste stuiptrekking van zijn leven als assistent.
De app gaf studenten overzicht over hun ingediende oplossingen voor een aantal reeksen opgelegde oefeningen.
Hij noemde hem Dodona.
Naar de tweede grootste orakelplaats van Griekenland.
Na Delphi waar het orakel Pythia gehuisvest was.

In die zomer van 2016 -- vlak voor de Gentse Feesten -- besloten we een online leerplatform te ontwikkelen.
Waarin we onze programmeercursussen zouden kunnen aanbieden.
Weg van \acronym{SPOJ}.
Daarin waren onze uitbreidingen ondertussen één grote hack geworden.
Sommige beperkingen van \acronym{SPOJ}.
Daar hadden we mee leren leven.
Maar ze vormden een serieuze rem.
Een strak keurslijf waar we vanaf wilden.
Om de vele ideeën uit te werken die we nog wilden realiseren om activerend leren te bevorderen.
Al hadden we van die term nog nooit gehoord.
Evenmin als van \english{blended learning} of de \english{flipped classroom}.
We deden het al zonder dat we er erg in hadden.

Uit die as werd Dodona geboren.
Het leerplatform zoals we het nu kennen.
Design sterk beïnvloed door ervaringen uit voorgaande jaren.
Positief en negatief.
Aanbod van cursussen met een leerpad van oefeningen gegroepeerd in reeksen.
Strikte scheiding van verantwoordelijkheden tussen judges en programmeeroefeningen.
Aan de ene kant een generiek test framework voor een bepaalde programmeertaal.
Aan de andere kant specifieke testen voor het automatisch beoordelen van ingediende oplossingen volgens de specificaties van een oefening. 
Docker-containers als sandbox waarin judges een ingediende oplossing konden uitvoeren en beoordelen.
Eveneens een strikte scheiding tussen het genereren van feedback door de judge en het weergeven van feedback door Dodona.
Met een gestandaardiseerd \acronym{JSON} Schema om feedback vast te leggen.
We hielden ook vast aan beheer van content (judges en oefeningen) in externe git \english{repositories}.
Inclusief \english{webhooks} voor synchronisatie met Dodona.
Zo moest Dodona geen gebruikersinterface krijgen om content te beheren en behielden auteurs ook volledige controle over hun content.
Bonus: content kan eenvoudig toegevoegd worden aan meerdere Dodona-instanties.

Een universeel ontwerp waarmee we theoretisch gezien alle programmeertalen kunnen ondersteunen.
Strak plan.
Strak tijdschema.
Strakke regie.
Harm Delva als jobstudent mee in bad getrokken.
Bestaande Python- en JavaScript-judges overgezet naar Dodona.
Alle bestaande oefeningen overgezet.
Ondertussen al meer dan 600.
Half augustus.
Bart en Felix trekken naar de Ghent University Global Campus (\acronym{GUGC}) in Zuid-Korea.
Om er een zomercursus Python te doceren.
Ideaal om de bèta-versie van Dodona voor de leeuwen te gooien.

7 september 2016.
Mail van collega prof.\ dr.\ ir.\ Bart Dhoedt.
Faculteit Ingenieurswetenschappen en Architectuur.
Via via had hij vernomen dat we met iets bezig waren.
Of ze het ook mochten gebruiken.
Voor een programmeercursus met 400 eerstejaarsstudenten.
In het nieuwe academiejaar maakten ze de overstap van Java naar Python.
Plots kwam er dus meer in het vizier dan die ene cursus die we initieel voorzien hadden.

26 september 2016.
D-day.
Start van het nieuwe academiejaar aan de UGent.
Geboorte van Dodona.
Versie \version{1.0} uitgerold.
Blij dat we ervoor gegaan waren en dat we het gehaald hadden.
De spanning was te snijden.
Die maandagavond bekeek Bart de statistieken.
Hij verwedde erom dat we tegen het einde van de week 25.000 ingediende oplossingen zouden halen.
Ik ging de weddenschap aan.
Doe ik anders nooit.
We haalden meer dan 75.000 ingediende oplossingen.
Gegokt en verloren.
Maar bijzonder enthousiast om aan de volgende etappe van het avontuur te beginnen.

Het volgende academiejaar maakten al 40 cursussen aan UGent gebruik van Dodona.
In 2018 waren we laureaat van de UGent Minervaonderscheiding voor onderwijs.
Voor onze activerende manier van lesgeven en voor het ontwikkelen van het innovatieve EdTech platform Dodona.
Beide nauw met elkaar verstrengeld.
Datzelfde jaar gaf UGent een impuls aan de verdere uitbouw van Dodona door ons een interfacultair onderwijsinnovatieproject toe te kennen.
Daarin waren 7 faculteiten en de \acronym{GUGC} betrokken.
Samen met de Directie Informatie- en Communicatietechnologie (\acronym{DICT}) van de UGent.
Voor logistieke ondersteuning.
Eventjes pauze.
Hoe kon ik in voorgaande nog vergeten vermelden dat zij ons al jarenlang de nodige \acronym{ICT}-infrastructuur geboden hebben.
Een dikke merci aan Johan Van Camp en zijn team is hier dus zeker op zijn plaats.
Soit.
Met de projectfinanciering konden we voor het eerst op zoek naar een fulltime softwareontwikkelaar voor Dodona.
Charlotte Van Petegem was bij ons haar masterproef aan het afwerken.
Zij zag het zitten.
Wij zagen het zitten.
Vanaf dan konden we beginnen spreken van team Dodona.

Kunnen we Dodona ook aan de Universiteit Hasselt gebruiken?
We spreken december 2017.
De vraag kwam van collega prof.\ dr.\ Frank Neven.
\acronym{SAML}-authenticatie opgezet.
In februari 2018 ging hij met zijn studenten aan de slag.
Snel daarna begon dezelfde vraag ook te komen van andere Vlaamse universiteiten en hogescholen.

Zouden ook leerkrachten uit het secundair onderwijs in Dodona een co-teacher kunnen vinden?
De vraag kwam initieel van leerkrachten die deelnamen aan nascholingen rond programmeren en algoritmen van prof.\ dr.\ Veerle Fack.
Daarin was ze ook Dodona beginnen gebruiken.
Hoe pakken we authenticatie aan voor het secundair onderwijs?
Rien Maertens \english{to the rescue}.
Als masterstudent kwam hij enkele weken team Dodona vervoegen -- in volle blok nota bene -- om OAuth-koppelingen te voorzien met Office 365- en Smartschool-accounts van scholen.
Op 3 september 2018 gingen drie pilootscholen aan de slag met Dodona.
Op 5 september kregen we een enthousiaste mail van Dominiek Vandewalle.
Leerkracht \acronym{ICT} aan de Sint-Paulusschool campus College Waregem.
Eerste programmeerles gegeven op woensdagvoormiddag.
Op woensdagmiddag waren zijn leerlingen thuis al oplossingen aan het indienen voor de meer uitdagende oefeningen.

Uitnodiging van Annick Van Daele.
Voorzitster van 2Link2.
De vakvereniging voor leraren informatica en \acronym{STEM}.
Of we ons op 23 juni 2019 konden vrijmaken om een workshop rond Dodona te geven?
Wegens veel interesse werden het twee workshops.
Allebei volgeboekt.
We maakten van de gelegenheid gebruik om alle scholen toegang te geven tot Dodona.
Gratis uiteraard.
Kunnen we ook toegang krijgen met de Google G-Suite-accounts van onze school?
Charlotte?
\english{Consider it done}.
We hadden de tamtam van het secundair onderwijs duidelijk onderschat.
Tien maanden later waren al ruim 300 secundaire scholen op Dodona geregistreerd.
Sommige waren al met leerlingen aan de slag.
Anderen waren nog wat aan het verkennen.
Toen kwam corona.
Plots ging iedereen op zoek naar digitale ondersteuning voor het onderwijs.
Twee maanden later stond de teller boven de 450 scholen.

Vlak voor de zomer van 2019 kreeg Bart een senior postdoc mandaat bij het \acronym{FWO}\@.
Voor Unipept kreeg hij naast Felix ook Pieter Verschaffelt als \english{wingman}.
Die had ook een doctoraatsmandaat van het \acronym{FWO} gekregen.
Charlotte en Rien kozen na de zomer voor een doctoraatsproject binnen het Dodona-ecosysteem.
Verzekering dat Unipept en Dodona alvast voor de komende jaren in goede handen waren.
Bart kon zijn energie over de twee projecten verdelen.

Ondertussen waren er ten huize Dodona ook steeds meer judges voor andere programmeertalen bijgekomen.
Prof.\ dr.\ Christophe Scholliers en Robbert Gurdeep Singh schreven judges voor Haskell en Prolog.
Dr.\ Niels Neirynck voor bash.
Dr.\ Dieter Mourisse voor C\#\@.
Charlotte voor R\@.
Felix voor Java.
Later met hulp van masterstudent Pieter De Clercq.
Die samen met studiegenoot Tobiah Lissens ook een JetBrains-plugin voor Dodona ontwikkelde.

Vaststelling.
Al die judges waren ontwikkeld door de \english{inner circle} rond Dodona.
Allemaal medewerkers van de opleiding informatica.
Blijkbaar voelen velen zich geroepen om programmeeroefeningen uit te werken op basis van een bestaande judge.
Weinigen durven het aan om zelf een judge te ontwikkelen.
Het helpt om wat af te weten van de programmeertaal waarvoor de beoordeling moet gebeuren.
Maar toegegeven.
Het bouwen van een judge is geen \english{rocket science}.
Al vraagt het wel wat doorzettingsvermogen om hem robuust te maken tegen weerbarstige code van studenten en rijke feedback te voorzien die typisch ook door andere judges geboden wordt.
Want -- vaststelling \#2 -- de meeste judges implementeren \latin{grosso modo} dezelfde features maar dan elk voor/in een specifieke programmeertaal.

Wacht eens even.
Judges lijken zeer sterk op elkaar?
Op de details van de programmeertaal na.
Dus schuilt er zekere routine in het schrijven van nieuwe judges?
Hoeveel programmeertalen zijn er?
Sorry, maar als informaticus haal ik mijn neus op om steeds hetzelfde te moeten doen.
Zouden we geen generieke judge kunnen schrijven?
Die feedback geeft op ingediende oplossingen voor programmeeroefeningen.
Ongeacht de programmeertaal waarin die oplossingen geschreven zijn.
\english{One judge to rule them all}, zeg maar.
Het idee begon te rijpen op het einde van Charlotte Van Petegem haar masterproef.
We werkten het ontwerp van een architectuur uit.
Testen uitvoeren loskoppelen van resultaten beoordelen.
De resultaten moeten we daarvoor kunnen serialiseren en deserialiseren.
Het uitvoeren kunnen we misschien veralgemenen door de kernels van Jupyter Notebooks te gebruiken.
Daarmee hebben we meteen ondersteuning voor tientallen programmeertalen.
Enter \tested{}.
Conceptueel dan toch.
We klopten aan bij het \acronym{FWO}\@.
Interessant!
Maar blijkbaar toch niet interessant genoeg.
Ook wat bedenkingen bij de haalbaarheid.

Enter Niko Strijbol.
Master in de informatica in wording.
Coding ninja \english{by night}.
Voor zijn masterproef zou hij er wel eens zijn vingers inzetten (tanden zijn alleen maar spreekwoordelijk).
Jupyter-kernels gingen snel op de schop.
De overhead om er afzonderlijk voor elke context één op te starten bleek een onoverkomelijke bottleneck.
Werken met een pool van kernels bracht ook geen zoden aan de dijk.
\english{No problemo}.
Dacht Niko.
Dan zetten we het testplan van een oefening maar statisch om naar uitvoerbare testcode.
Dat werkt ook.
Serialisatie/deserialisatie van waarden?
Grasduinen in bestaande technologieën leverde enkele interessante pistes op.
Maar niets dat uiteindelijk aan alle verwachtingen voldeed.
\english{No problemo}.
Dacht Niko.
Met een \english{template engine} moet dat toch ook lukken.
Jinja2?
Leek aanvankelijk een goede keuze.
Maar gaf uiteindelijke onvoldoende controle over witruimte.
Mako dan maar.
Hoeveel performantie verliezen we met de extra vertaalslag van een generiek testplan naar testcode voor een specifieke programmeertaal?
Met batchcompilatie en parallelle verwerking van contexten bleek \tested{} zelfs sneller dan de meeste judges die specifiek voor één programmeertaal een beoordeling uitvoeren.
Maar nu ben ik al te veel aan het verklappen.
In dit proefschrift lees je hoe alle puzzelstukjes in elkaar gevallen zijn.
Hopelijk beleef je evenveel plezier aan het lezen als wij gehad hebben bij het uitdenken, implementeren en configureren van \tested{}.

Wordt \tested{} de heilige graal voor het automatisch beoordelen van software in een educatieve context?
Daarvoor moet het framework zijn sporen op het slagveld nog verdienen.
Zal een generieke judge ooit alle features kunnen aanbieden als judges die specifiek op één programmeertaal gericht zijn?
Voer voor minstens nog een masterproef.
Gelukkig is er al een masterstudent informatica geïnteresseerd om die uitdaging aan te gaan.
We moeten ook nog meer oefeningen opstellen voor \tested{}.
Veel meer oefeningen.
Kunnen we bestaande oefeningen automatisch omzetten naar \tested{}?
Welke programmeertalen zullen er volgend jaar toegevoegd zijn aan \tested{}?
Hoeveel programmeertalen zal \tested{} over vier jaar ondersteunen?
Dat deel van de geschiedenis moeten we nog schrijven.
Maar ik ben blij dat we voor \tested{} gegaan zijn.
Trots op het prototype dat er nu al ligt.
Enthousiast om aan de volgende etappe van het avontuur te beginnen.
Samen met Niko die vanaf september team Dodona komt vervoegen.

\begin{flushright}
    Peter Dawyndt\\
    juni 2020
\end{flushright}


    \chapter{Educational software testing}\label{ch:dodona}

\section{Inleiding}\label{sec:inleiding}

\subsection{De digitale maatschappij}\label{subsec:de-digitale-maatschappij}

\lettrine{D}{e evoluties} op technologisch vlak hebben ervoor gezorgd dat onze maatschappij de laatste decennia in hoge mate gedigitaliseerd is, een proces dat nog steeds aan de gang is.
Bovendien kan, door de snelheid waarmee deze veranderingen vaak optreden, eerder gesproken worden van een revolutie dan een evolutie: de veranderingen zijn vaak ingrijpend en veranderen fundamentele aspecten van de sectoren waarin de digitalisering plaatsvindt.
Dit gaat over nieuwe sectoren, zoals de deeleconomie, of ingrijpende veranderingen bij bestaande sectoren, zoals de opkomst van \english{ride sharing} in de taxisector.
Ook de impact op maatschappelijk vlak, zoals de sociale media in de politiek, mag niet vergeten worden.
Al deze uitdagingen vragen een gepast antwoord, zoals onder andere \autocite{hipeac2019}.

Ook op educatief vlak heeft de digitalisering een grote impact.
Enerzijds biedt digitalisering nieuwe mogelijkheden aan voor onderwijsdoeleinden, zoals het lesgeven op afstand, het online aanbieden van leermateriaal en het online indien en verbeteren van opdrachten.

Anderzijds biedt het ook uitdagingen: om studenten voor te bereiden op de steeds digitalere maatschappij is een basis van digitale geletterdheid nodig.
Net door de snelle evolutie op technologisch vlak volstaat het niet om studenten te leren werken met de technologie van vandaag;
een grondige kennis van de onderliggende werking van de technologie is onontbeerlijk.

Een belangrijk aspect hierin is het concept van \term{computationeel denken}.
Dit concept gaat over het omvormen van problemen zodat de door een computer opgelost kunnen worden.
Dat het aanleren van deze vaardigheden nodig is, bewijst ook de opname van computationeel denken in de eindtermen, bijvoorbeeld in het katholieke basisonderwijs \autocite{zinin2017} of in de onderwijsdoelen voor het secundair onderwijs als \term{Computationeel denken en handelen} \autocite{2019040867}.

\subsection{Computationeel denken}\label{subsec:computationeel-denken}

Over de vraag wat computationeel denken nu precies betekent lopen de antwoorden uiteen.
Het Departement Onderwijs en Vorming van de Vlaams Overheid \autocite{bastiaensen2017} definieert de term als volgt:

\begin{quote}
    Computationeel denken verwijst dus naar het menselijke vermogen om complexe problemen op te lossen en daarbij computers als hulpmiddel te zien.
    Met andere woorden, computationeel denken is het proces waarbij aspecten van informaticawetenschappen herkend worden in de ons omringende wereld, en waarbij de methodes en technieken uit de informaticawetenschappen toegepast worden om problemen uit de fysische en virtuele wereld te begrijpen en op te lossen.
\end{quote}

Computationeel denken is dus ruimer dan programmeren, maar programmeren vormt wel een uitstekende manier om het computationeel denken aan te leren en te oefenen.
Bovendien is programmeren op zich ook een nuttige vaardigheid om studenten aan te leren.

\subsection{Programmeeroefeningen}\label{subsec:programmeeroefeningen}

Het aanleren van programmeren is niet eenvoudig en wordt door veel studenten als moeilijk ervaren \autocite{10.1145/3293881.3295779}.
Het maken van oefeningen kan daarbij helpen, indachtig het spreekwoord "oefening baart kunst".
De studenten veel oefeningen laten maken, resulteert wel in twee uitdagingen voor de lesgevers:
\begin{enumerate}
    \item De lesgevers moeten geschikte oefeningen opstellen, die rekening houden met welke programmeerconcepten studenten al kennen, tijdslimieten, moeilijkheidsgraden, enz.
    Het opstellen van deze oefeningen vraagt veel tijd.
    \item De oplossingen voor deze oefeningen moeten voorzien worden van kwalitatieve feedback.
    Bij het aanleren van programmeren is feedback namelijk een belangrijk element om de programmeervaardigheden van de studenten te verbeteren \autocite{10.1145/2899415.2899422}.
\end{enumerate}

Deze thesis focust op de eerste uitdaging, al wordt ook met de tweede uitdaging rekening gehouden, onder andere in \cref{sec:robuustheid}.
Hiervoor werkt de thesis binnen de context van Dodona, een online leerplatform dat sinds september 2016 beschikbaar is aan de Universiteit Gent.

\section{Het leerplatform Dodona}\label{sec:wat-is-dodona}

Sinds 2011 wordt aan de vakgroep Computationele Biologie aan de Universiteit Gent gewerkt met programmeeroefeningen die in een online systeem ingediend en beoordeeld worden.
Oorspronkelijk werd hiervoor gebruik gemaakt van de \english{Sphere Online Judge} (SPOJ) \autocite{10.1007/978-3-540-78139-4_31}.
Op basis van ervaringen met SPOJ ontwikkelde de vakgroep een eigen leerplatform, Dodona, dat in september 2016 beschikbaar werd.
Het doel van Dodona is eenvoudig: lesgevers bijstaan om hun studenten niet alleen zo goed mogelijk te leren programmeren, maar dit ook op een zo efficiënt mogelijke manier te doen.

Het online leerplatform Dodona kunnen we opdelen in verschillende onderdelen:
\begin{enumerate}
    \item Het platform zelf.
    Dit is een webapplicatie, die verantwoordelijk is om alle modules samen te laten werken en ook de webinterface aanbiedt die de studenten en lesgevers gebruiken.
    Het is via deze interface dat lesgevers oefeningen beschikbaar maken en dat studenten hun oplossingen indienen.
    Het platform zelf is programmeertaalonafhankelijk geschreven.
    \item Judges.
    De judges binnen Dodona zijn verantwoordelijk voor het beoordelen van ingediende oplossingen.
    Dit onderdeel wordt uitgebreid besproken in \cref{subsec:de-judge}.
    \item Oefeningen.
    Oefeningen worden niet in Dodona zelf opgeslagen, maar worden door de lesgever aangeleverd via een git-repository.
    De oefeningen bevatten de opgave en de testen die uitgevoerd worden tijdens de beoordeling van een oplossing.
\end{enumerate}

Met Dodona kunnen lesgevers een leertraject opstellen door een reeks oefeningen te selecteren.
Studenten die dit leertraject volgen, zien onmiddellijk hun voortgang binnen het traject.
Bij het indienen van hun oplossingen ontvangen de studenten ook onmiddellijk feedback over hun oplossing: deze feedback bevat niet alleen de juistheid van de oplossing, maar kan ook andere aspecten belichten, zoals de kwaliteit en performantie van de oplossing.

\section{Beoordelen van oplossingen}\label{sec:evalueren-van-een-oplossing}

\subsection{De judge}\label{subsec:de-judge}

In Dodona wordt elke ingediende oplossing beoordeeld door een evaluatieprogramma, de \termen{judge}.
In wezen is dit een eenvoudig programma: via de standaardinvoerstroom (\texttt{stdin}) krijgt het programma een configuratie binnen van Dodona.
Deze configuratie bevat de invoer, bestaande uit onder andere de programmeertaal van de oplossing, het pad naar het oplossingsbestand en geheugen- en tijdslimieten.
Het resultaat van de beoordeling wordt uitgeschreven naar de standaarduitvoerstroom (\texttt{stdout}).
Zowel de invoer als de uitvoer van de judge zijn json, waarvan het formaat vastgelegd is in een json-schema.\footnote{Dit schema en een tekstuele beschrijving ervan is te vinden in de handleiding op \url{https://dodona-edu.github.io/en/guides/creating-a-judge/}.}

Concreet wordt elke beoordeling uitgevoerd in een Docker-container.
Deze Docker-container wordt gemaakt op basis van een Docker-image die bij de judge hoort, en alle dependencies bevat die de judge in kwestie nodig heeft.
Bij het uitvoeren van de beoordeling zal Dodona een \english{bind mount}\footnote{Informatie over deze term vindt u op \url{https://docs.docker.com/storage/bind-mounts/}} voorzien, zodat de code van de judge zelf, de code van de oefening en de code van de student beschikbaar zijn in de container.
Via de configuratie geeft Dodona aan de judge aan waar deze bestanden zich bevinden.

Samenvattend bestaat interface tussen de judge en Dodona uit drie onderdelen:

\begin{enumerate}
    \item De judge zal uitgevoerd worden in een Docker-container, dus een Docker-image met alle dependencies moet voorzien worden.
    Deze Docker-image moet ook de judge opstarten.
    \item De judge stelt de invoer van een beoordeling ter beschikking voor de judge.
    Bestanden worden via een bind mount aan de Docker-container gekoppeld.
    De paden naar deze bestanden binnen de container en andere informatie (zoals programmeertaal van de oplossing of natuurlijke taal van de gebruiker) worden via de configuratie aan de judge gegeven (via standaardinvoer).
    \item De judge moet het resultaat van zijn beoordeling uitschrijven naar standaarduitvoer, in een vastgelegd formaat.
\end{enumerate}

Buiten deze interface legt Dodona geen vereisten op aan de werking van judge.
Door deze vrijheid lopen de manieren waarop de bestaande judges geïmplementeerd zijn uiteen.
Sommige judges beoordelen oplossingen in dezelfde programmeertaal als de taal waarin ze geschreven zijn.
Zo is de judge voor Python-oplossingen geschreven in Python en de judge voor Java-oplossingen in Java.
Bij andere judges is dat niet het geval: de judges voor Bash en Prolog zijn bijvoorbeeld ook in Python geschreven.
Ook heeft elke judge een eigen manier waarop de testen voor een oplossing opgesteld moeten worden.
Zo worden in de Java-judge jUnit-testen gebruikt, terwijl de Python-judge doctests en een eigen formaat ondersteunt.

\subsection{De beoordeling zelf}\label{subsec:de-beoordeling-zelf}

De beoordeling van een oplossing van een student laat zich beschreven als het volgende stappenplan:

\begin{enumerate}
    \item De student dient de oplossing in via de webinterface van Dodona.
    \item Dodona start een Docker-container voor de judge.
    \item Dodona voorziet de container van de bestanden van de judge, de oefening en de ingediende oplossing.
    \item De judge wordt uitgevoerd met de configuratie als invoer.
    \item De judge beoordeelt de oplossing aan de beoordelingsmethodes opgesteld door de lesgever (d.w.z.\ de jUnit-test, de doctests, \ldots).
    Sommige judges voeren ook bijkomende taken, zoals linting, beoordeling van de performantie of \english{grading} van de code van de oplossing.
    \item De judge vertaalt zijn beoordeling naar het Dodona-formaat en schrijft het resultaat naar het standaarduitvoerkanaal.
    \item Dodona slaat dat resultaat op in de databank.
    \item Op de webinterface krijgt de student het resultaat te zien als feedback op de ingediende oplossing.
\end{enumerate}



\section{Probleemstelling}\label{sec:probleemstelling}

De manier waarop de huidige judges werken, resulteert in twee belangrijke nadelen.
Bij het bespreken hiervan is het nuttig een voorbeeld in het achterhoofd te houden, teneinde de nadelen te kunnen concretiseren.
Als voorbeeld gebruiken we de "Lotto"-oefening\footnote{Vrij naar een oefening van prof.\ Dawyndt. De originele oefening is beschikbaar op \url{https://dodona.ugent.be/nl/exercises/2025591548/}}, met volgende opgave:

\begin{quote}
    \markdownInput{generated/description.md}
\end{quote}

Oplossingen voor deze oefening staan in \cref{lst:java-solution,lst:python-solution}, voor respectievelijk Python en Java.

\begin{listing}
    \inputminted{java}{../../exercise/lotto/solution/correct.java}
    \caption{Voorbeeldoplossing in Java.}
    \label{lst:java-solution}
\end{listing}

\begin{listing}
    \inputminted{python3}{../../exercise/lotto/solution/correct.py}
    \caption{Voorbeeldoplossing in Python.}
    \label{lst:python-solution}
\end{listing}

\subsubsection{Implementatie van oefeningen}

Het eerste en belangrijkste nadeel aan de werking van de huidige judges heeft betrekking op de lesgevers en komt voor als zij een oefening willen aanbieden in meerdere programmeertalen.
Enerzijds is dit een zware werklast: de oefening, en vooral de code voor de beoordeling, moet voor elke judge opnieuw geschreven worden.
Voor de Python-judge zullen doctests nodig zijn, terwijl de Java-judge jUnit-testen vereist.
Anderzijds lijdt dit ook tot verschillende versies van dezelfde oefening, wat het onderhouden van de oefeningen moelijker maakt.
Als er bijvoorbeeld een fout sluipt in de beoordelingscode, zal de lesgever er aan moeten denken om de fout te verhelpen in alle varianten van de oefening.
Bovendien geeft elke nieuwe versie van de oefening een nieuwe mogelijkheid voor het introduceren van fouten.

Kijken we naar onze Lotto-oefening, merken we dat het gaat om een eenvoudige opgave en een eenvoudige oplossing.
Bovendien zijn de verschillen tussen oplossingen in verschillende programmeertalen niet zo groot.
In de voorbeeldoplossingen in Python en Java zijn de verschillen minimaal, zij het dat de Java-oplossing wat langer is.
De Lotto-oefening zou zonder problemen in nog vele andere programmeertalen opgelost kunnen worden.
Eenvoudige programmeeroefeningen, zoals de Lotto-oefening, zijn voornamelijk nuttig in twee gevallen: studenten die voor het eerst leren programmeren en studenten die een nieuwe programmeertaal leren.
In het eerste geval is de eigenlijke programmeertaal minder relevant: het zijn vooral de concepten die belangrijk zijn.
In het tweede geval is de programmeertaal wel van belang, maar moeten soortgelijke oefeningen gemaakt worden voor elke programmeertaal die aangeleerd moet worden.
In beide gevallen is het dus een meerwaarde om de oefening in meerdere programmeertalen aan te bieden.

We kunnen tot eenzelfde constatatie komen bij meer complexe oefeningen die zich concentreren op algoritmen: ook daar zijn de concepten belangrijker dan in welke programmeertaal een algoritme uiteindelijk geïmplementeerd wordt.
Een voorbeeld hiervan is het vak "Algoritmen en Datastructuren" dat gegeven wordt door prof.\ Fack binnen de opleiding wiskunde\footnote{De studiefiche is beschikbaar op \url{https://studiegids.ugent.be/2019/NL/studiefiches/C002794.pdf}}.
Daar zijn de meeste opgaven vandaag al beschikbaar in Java en Python op Dodona, maar dan als afzonderlijke oefeningen.

Een ander aspect is de beoordeling van een oefening.
Voor de Lotto-oefening is de beoordeling niet triviaal, door het gebruik van niet-deterministische functies.
Het volstaat voor dit soort oefeningen niet om de uitvoer gegenereerd door de oplossing te vergelijken met een op voorhand vastgelegde verwachte uitvoer.
De geproduceerde uitvoer zal moeten gecontroleerd worden met code, specifiek gericht op deze oefening, die de verwachte vereisten van de oplossing controleert.
Deze evaluatiecode moet momenteel voor elke programmeertaal en dus elke judge opnieuw geschreven worden.
In de context van de Lotto-oefening controleert deze code bijvoorbeeld of de gegeven getallen binnen het bereik liggen en of ze gesorteerd zijn.

\subsubsection{Implementatie van judges}

Een tweede nadeel aan de werking zijn de judges zelf: voor elke programmeertaal die men wil aanbieden in Dodona moet een nieuwe judge ontwikkeld worden.
Ook hier is er dubbel werk: dezelfde concepten en features, die eigenlijk programmeertaalonafhankelijk zijn, moeten in elke judge opnieuw geïmplementeerd worden.
Hierbij denken we aan bijvoorbeeld de logica om te bepalen wanneer een beoordeling positief of negatief moet zijn.

\subsubsection{Onderzoeksvraag}

We beschouwen het eerste nadeel als het belangrijkste nadeel, en vatten het samen als de onderzoeksvraag waarop deze thesis een antwoord wil bieden:

\begin{quote}
    Is het mogelijk om een judge zo te implementeren dat de opgave en beoordelingsmethoden van een oefening slechts eenmaal opgesteld dienen te worden, waarna de oefening beschikbaar is in alle programmeertalen die de judge ondersteunt?
    Hierbij willen we dat eens een oefening opgesteld is, deze niet meer gewijzigd moet worden wanneer talen toegevoegd worden aan de judge.
\end{quote}

Als bijzaak zijn we ook geïnteresseerd of de judge uit de onderzoeksvraag een voordeel kan bieden voor het implementeren van judges zelf (het tweede nadeel).

De aandachtige lezer zal opmerken dat de opgave voor de Lotto-oefening programmeertaalspecifieke en taalspecifieke elementen bevat.
Zo zijn de voorbeelden in Python en zijn de namen van functies en argumenten in het Nederlands.
Beide zaken worden voor deze thesis expliciet als \english{out-of-scope} gezien en zullen niet behandeld worden.

\section{Opbouw van de thesis}\label{sec:opbouw}

\Cref{ch:de-universele-judge} handelt over het antwoord op bovenstaande vraag, waar een prototype van een dergelijke judge wordt voorgesteld.
Daarna volgt ter illustratie een gedetailleerde beschrijving van hoe een oefening opgesteld moet worden voor deze judge.
Nadien volgt een beschrijving van hoe een nieuwe programmeertaal moet toegevoegd worden.
Daar deze twee hoofdstukken voornamelijk ten doel hebben zij die met de judge moeten werken te informeren, nemen deze hoofdstukken de vorm aan van meer traditionele softwarehandleidingen.
Tot slot volgt met een hoofdstuk over beperkingen van de huidige implementaties, en waar er verbeteringen mogelijk zijn (het "toekomstige werk").

    \chapter{TESTed}\label{ch:tested}

\lettrine{I}{n het kader} van deze masterproef werd een prototype geïmplementeerd van een judge voor Dodona.
Het doel hiervan is een antwoord te bieden aan de onderzoeksvraag uit het vorige hoofdstuk en de beperkingen van deze aanpak in kaart te brengen.
Deze judge heeft de naam \term{TESTed} gekregen.
Bij TESTed is een oefening programmeertaalonafhankelijk en kunnen oplossingen in verschillende programmeertalen beoordeeld worden aan de hand van een en dezelfde specificatie.
Dit hoofdstuk begint met het ontwerp en de algemene werking van de judge toe te lichten, waarna elk onderdeel in meer detail besproken wordt.

\section{Overzicht}\label{sec:ontwerp}

\subsection{Architecturaal ontwerp}\label{subsec:architecturaal-overzicht}

\Cref{fig:universal-judge} toont het architecturaal ontwerp van TESTed.
De twee stippellijnen geven programmeertaalbarrières aan, en verdelen TESTed in drie logisch omgevingen:

\begin{enumerate}
    \item TESTed zelf is geschreven in Python: in het middelste deel staat de programmeertaal dus vast.
    Dit onderdeel is verantwoordelijk voor de regie van de beoordeling op basis van het testplan.
    \item De ingediende oplossing wordt uitgevoerd in de \term{uitvoeringsomgeving}, waar de programmeertaal overeenkomt met de programmeertaal van de oplossing.
    \item Tot slot is er nog de \term{evaluatieomgeving}, waar door de lesgever geschreven evaluatiecode wordt uitgevoerd.
    Deze moet niet in dezelfde programmeertaal als de oplossing of TESTed geschreven zijn.
\end{enumerate}

\subsection{Stappenplan van een beoordeling}\label{subsec:stappenplan-van-een-beoordeling}

De rest van het hoofdstuk bespreekt alle onderdelen van en stappen die gebeuren bij een beoordeling van een ingediende oplossing in detail.
In \cref{fig:tested-flow} zijn deze stappen gegeven als een flowchart, en een uitgeschreven versie volgt:

\begin{enumerate}
    \item De Docker-container voor TESTed wordt gestart.
    Dodona stelt de invoer ter beschikking aan de container: het testplan komt uit de oefening, terwijl de ingediende oplossing en de configuratie uit Dodona komen.
    \item Als eerste stap wordt gecontroleerd dat het testplan de programmeertaal van de ingediende oplossing ondersteunt.
    De programmeertaal van de oplossing wordt gegeven via de configuratie uit Dodona.
    Merk op dat de ingediende oplossing zelf hierbij niet nodig is: deze controle zou idealiter gebeuren bij het importeren van de oefening in Dodona, zodat Dodona weet in welke programmeertalen een bepaalde oefening aangeboden kan worden (zie \cref{ch:beperkingen-en-toekomstig-werk}).
    Als het testplan bijvoorbeeld programmeertaalspecifieke code bevat die enkel in Java geschreven is, zal een oplossing in Python niet beoordeeld kunnen worden.
    Bevat het testplan bijvoorbeeld een functie die een verzameling moet teruggeven, dan zullen talen als Bash niet in aanmerking komen.
    \item Het testplan (details in \cref{subsec:het-testplan}) bestaat uit verschillende contexten.
    Elke context is een onafhankelijke uitvoering van de ingediende oplossing en kan verschillende aspecten van die uitvoering beoordelen.
    Voor elk van die contexten wordt in deze stap de testcode gegenereerd.
    Deze stap is de overgang naar de \term{uitvoeringsomgeving}.
    \item De testcode wordt optioneel gecompileerd.
    Dit kan op twee manieren gebeuren (details in \cref{subsec:testcode-genereren}):
    \begin{enumerate}
        \item Batchcompilatie: hierbij wordt de testcode van alle contexten verzameld en gecompileerd tot één uitvoerbaar bestand (executable).
        Dit heeft als voordeel dat er slechts een keer een compilatie nodig is, wat voor een betere performantie zorgt.
        Bij deze manier resulteert de compilatiestap in één uitvoerbaar bestand.
        \item Contextcompilatie: hierbij wordt de testcode voor elke context afzonderlijk gecompileerd tot een uitvoerbaar bestand.
        Bij deze manier worden er $n$ uitvoerbare bestanden geproduceerd tijdens de compilatiestap.
    \end{enumerate}
    In talen die geen compilatie nodig hebben of ondersteunen, wordt deze stap overgeslagen.
    \item Nu kan het uitvoeren van de beoordeling zelf beginnen: de gegenereerde code wordt uitgevoerd (nog steeds in de uitvoeringsomgeving).
    Elke context uit het testplan wordt in een afzonderlijk subproces uitgevoerd, teneinde het delen van informatie tegen te gaan.
    \item De uitvoering van de executable in de vorige stap produceert resultaten (voor elke context), zoals de standaarduitvoerstroom, de standaardfoutstroom, returnwaardes, exceptions of exitcodes.
    Deze bundel resultaten wordt nu geëvalueerd op juistheid.
    Hiervoor zijn drie mogelijke manieren:
    \begin{enumerate}
        \item Programmeertaalspecifieke evaluatie (afgekort tot SE in de flowchart).
        De code voor de evaluatie is opgenomen in de executable en wordt onmiddellijk uitgevoerd in hetzelfde proces.
        Via deze mogelijkheid kunnen taalspecifieke aspecten gecontroleerd worden.
        Daar de evaluatie in hetzelfde proces gebeurt, blijft dit in de uitvoeringsomgeving.
        \item Geprogrammeerde evaluatie (afgekort tot PE in de flowchart).
        Hierbij is er evaluatiecode geschreven die los staat van de oplossing, waardoor deze evaluatiecode ook in een andere programmeertaal geschreven kan zijn.
        De code ter uitvoering van de geprogrammeerde evaluatiecode wordt gegenereerd en dan uitgevoerd.
        Het doel van deze modus is om complexe evaluaties toe te laten op een programmeertaalonafhankelijke manier.
        Deze stap vindt plaats in de evaluatieomgeving.
        \item Generieke evaluatie.
        Hierbij evalueert TESTed zelf het resultaat.
        Deze modus is bedoeld voor gestandaardiseerde evaluaties, zoals het vergelijken van geproduceerde uitvoer en verwachte uitvoer.
        Hier gebeurt de evaluatie binnen TESTed zelf.
    \end{enumerate}
    \item Tot slot verzamelt TESTed alle evaluatieresultaten en stuurt ze gebundeld door naar Dodona, waarna ze getoond worden aan de gebruiker.
\end{enumerate}

\begin{figure}
    \centering
    \documentclass[class=ugent2016-report,crop=false]{standalone}

\usepackage{tikz}
\usetikzlibrary{shapes,arrows,positioning,backgrounds,calc,intersections,calc}
\usepackage[top=2cm, bottom=2cm, left=2cm, right=2cm]{geometry}

\begin{document}


    % Define the styles for various components in the architectural diagram.
    \tikzstyle{node}=[draw, minimum height=1cm, text width=3cm, align=center, fill=white]
    \tikzstyle{state}=[node, rectangle]
    \tikzstyle{process}=[node, rectangle, rounded corners=0.5cm]
    \tikzstyle{named}=[text=ugent-blue,font=\sffamily\scshape,align=center,text width=3cm]

    \begin{tikzpicture}

        %\draw[step=1.0,gray,thin] (0,0) grid (15,-25);

        \node[state] (input) at (7.5,2) {Invoer};

        \node[process] (generation) at (7.5,0) {Genereren \\ testcode};

        \node[state] (code) at (13.5,0) {Testcode};
        \node[process] (execution) at (13.5,-2) {Uitvoeren};

        \node[state] (execution state) at (13.5,-4) {Uitvoer};
        \node[state] (core state) at (7.5,-4) {Uitvoer};
        \node[state] (evaluation state) at (1.5,-4) {Uitvoer};

        \node[process] (custom evaluation) at (1.5,-6) {Evaluatie};
        \node[process] (core evaluation) at (7.5,-6) {Evaluatie};
        \node[process] (execution evaluation) at (13.5,-6)  {Evaluatie};

        \node[state] (feedback) at (7.5,-8) {Beoordeling};

        \node[named] (core name) at (7.5,-10) {TESTed};
        \node[named] (evaluation name) at (1.5,-10) {evaluatieomgeving};
        \node[named] (execution name) at (13.5,-10) {uitvoeringsomgeving};

        \begin{scope}[on background layer]

            % Draw these first to ensure they are in the background.
            \path[draw,dashed,very thick,lightgray] (4.5,3) -- (4.5,-11);
            \path[draw,dashed,very thick,lightgray] (10.5,3) -- (10.5,-11);

            \draw [->] (input) -- (generation);
            \draw[->] (generation) -- (code);
            \draw[->] (code) -- (execution);
            \draw[->] (execution) -- (execution state);
            \draw[->] (execution state) -- node[above] {Serialisatie} ++ (core state);
            \draw[->] (core state) -- node[above] {Deserialisatie} ++ (evaluation state);
            \draw[->] (core state) -- (core evaluation);
            \draw[->] (evaluation state) -- (custom evaluation);
            \draw[->] (execution state) -- (execution evaluation);

            \draw[->] (core evaluation.south) -- (feedback);
            \draw[->] (custom evaluation.south) -- (feedback.west);
            \draw[->] (execution evaluation.south) -- (feedback.east);

        \end{scope}


    \end{tikzpicture}

\end{document}

    \caption{Schematische voorstelling van het architecturale ontwerp van de TESTed.}
    \label{fig:universal-judge}
\end{figure}

\begin{figure}
    \centering
    %! Suppress = MultipleIncludes
\documentclass[class=ugent2016-report,crop=false,12pt]{standalone}

\usepackage{tikz}
\usetikzlibrary{shapes,arrows,positioning,backgrounds,calc,intersections,calc}
\usepackage[top=2cm, bottom=2cm, left=2cm, right=2cm]{geometry}

\begin{document}

    \tikzstyle{node}=[draw, minimum height=1cm, align=center, fill=white, text height=1.5ex, text depth=.25ex]
    \tikzstyle{process}=[node, rectangle]
    \tikzstyle{terminator}=[node, rectangle, rounded corners=0.5cm]
    \tikzstyle{document}=[node,tape,tape bend top=none]
    \tikzstyle{io}=[node,trapezium,trapezium left angle=70,trapezium right angle=-70,minimum width=3cm]
    \tikzstyle{nothing}=[align=center]
    \tikzstyle{inner}=[process,draw=gray]
    \tikzstyle{arrow}=[draw, -latex]
    \tikzstyle{ind}=[fill=ugent-we!50!white]
    \tikzstyle{pre}=[fill=ugent-ps!50!white]
    \tikzstyle{inda}=[draw=ugent-we!70!black]
    \tikzstyle{prea}=[draw=ugent-ps!70!black]
    \tikzstyle{ae}=[fill=ugent-re!50!white]
    \tikzstyle{ie}=[fill=ugent-ea!50!white]
    \tikzstyle{se}=[fill=ugent-ge!50!white]
    \tikzstyle{aea}=[draw=ugent-re!70!black]
    \tikzstyle{iea}=[draw=ugent-ea!70!black]
    \tikzstyle{sea}=[draw=ugent-ge!70!black]

    \begin{tikzpicture}
%        \draw[step=1.0,gray,thin] (0,0) grid (15,-25);

        \node[io] at (5.625,-1) (exercise) {Oefening};
        \node[io] at (12.125,-1) (dodonaIn) {Dodona};
        
        % These needs to be drawn first, otherwise they are on top of the
        % input node.
        \node[process,minimum width=3cm] at (1.5,-7) (g1) {Genereren};
        \node[process,minimum width=3cm] at (5.5,-7) (g2) {Genereren};
        \node[process,minimum width=3cm] at (9.5,-7) (gn) {Genereren};

        \node[document,minimum width=3cm,text height=9ex,text depth=2ex,pre] at (13.5,-8) (ta) {Testcode 1\\Testcode 2\\Testcode $n$};

        \node[document,minimum width=3cm] at (1.5,-8.5) (t1) {Testcode 1};
        \draw[arrow,prea] (t1.10) -- (t1.10-|ta.west);
        \node[document,minimum width=3cm] at (5.5,-8.5) (t2) {Testcode 2};
        \draw[arrow,prea] (t2.east) -- (t2.east-|ta.west);
        \node[document,minimum width=3cm] at (9.5,-8.5) (tn) {Testcode $n$};
        \draw[arrow,prea] (tn.350) -- (tn.350-|ta.west);

        \node[process,minimum width=3cm,ind] at (1.5,-10) (co1) {Compileren};
        \draw[arrow] (11.5,-9.66) -- (11.5,-9.66-|co1.east);
        \node[process,minimum width=3cm,ind] at (5.5,-10) (co2) {Compileren};
        \draw[arrow] (11.5,-10) -- (11.5,-10-|co2.east);
        \node[process,minimum width=3cm,ind] at (9.5,-10) (con) {Compileren};
        \node[process,minimum width=3cm,pre] at (13.5,-10) (coa) {Compileren};

        \draw[arrow] (0,-5-|g1.120) -- (g1.120);
        \draw[arrow] (0,-5-|g2.60) -- (g2.60);
        \draw[arrow] (0,-5-|gn.60) -- (gn.60);

        % Input
        \node[document, minimum width=14cm, minimum height=4cm,tape bend height=0.5cm] at (7.5, -3.75) (input) {};

        \node[nothing] at (1.5,-2.375) {Invoer};

        \node[inner,minimum width=8.75cm, minimum height=2.25cm] at (5.625,-4) (plan) {};
        \node[nothing] at (2.5,-3.375) {Testplan};
        \node[inner,minimum width=2.25cm] at (2.75,-4.375) (c1) {Context 1};
        \node[inner,minimum width=2.25cm] at (5.25,-4.375) (c2) {Context 2};
        \node[nothing,text height=1.5ex, text depth=.25ex] at (6.875,-4.375) {\ldots};
        \node[inner,minimum width=2.25cm] at (8.5,-4.375) (cn) {Context $n$};

        \draw[arrow] (exercise.south) -- (exercise|-plan.north);

        \node[inner,minimum width=3.25cm, minimum height=1.25cm, text height=4ex] at (12.125,-4.5) (solution) {Ingediende \\ oplossing};

        \draw[arrow] (dodonaIn.south east) -- (dodonaIn.south east|-solution.north east);

        \node[inner,minimum width=3.25cm] at (12.125,-3) (config) {Configuratie};

        \draw[arrow] (11.5,-10.33|-solution.south) -- (11.5,-10.33) -- (11.5,-10.33-|con.east);
        \draw[arrow] (11.5,-10.1625) -- (11.5,-10.1625-|coa.west);

        \draw[arrow] (dodonaIn.south) -- (dodonaIn|-config.north);

        \draw[arrow] (c1) |- (g1.60|-2,-6.25) -- (g1.60);
        \draw[arrow] (c2) |- (g2.120|-6.75,-6.25) -- (g2.120);
        \draw[arrow] (cn) |- (gn.120|-12.5,-6.25) -- (gn.120);

        \draw[arrow] (g1) --(t1);
        \draw[arrow] (g2) --(t2);
        \draw[arrow] (gn) --(tn);

        \draw[arrow,inda] (t1) --(co1);
        \draw[arrow,inda] (t2) --(co2);
        \draw[arrow,inda] (tn) --(con);
        \draw[arrow,prea] (ta) --(coa);

        \node[document,minimum width=3cm,ind] at (1.5,-11.5) (e1) {Executable 1};
        \node[document,minimum width=3cm,ind] at (5.5,-11.5) (e2) {Executable 2};
        \node[document,minimum width=3cm,ind] at (9.5,-11.5) (en) {Executable $n$};
        \node[document,minimum width=3cm,pre] at (13.5,-11.5) (ea) {Executable};

        \draw[arrow,inda] (co1) --(e1);
        \draw[arrow,inda] (co2) --(e2);
        \draw[arrow,inda] (con) --(en);
        \draw[arrow,prea] (coa) --(ea);

        \node[process,minimum width=3cm] at (2.5,-13.5) (u1) {Uitvoeren};
        \node[process,minimum width=3cm] at (7.5,-13.5) (u2) {Uitvoeren};
        \node[process,minimum width=3cm] at (12.5,-13.5) (un) {Uitvoeren};

        \draw[arrow,inda] (e1) |- (u1.135|-0,-12.33) -- (u1.135);
        \draw[arrow,inda] (e2) |- (u2.135|-0,-12.33) -- (u2.135);
        \draw[arrow,inda] (en) |- (un.135|-0,-12.33) -- (un.135);

        \draw[arrow,prea] (ea) |- (u1.45|-0,-12.66) -- (u1.45);
        \draw[arrow,prea] (u2.45|-0,-12.66) -- (u2.45);
        \draw[arrow,prea] (un.45|-0,-12.66) -- (un.45);

        \node[document,minimum width=3cm] at (2.5,-15) (r1) {Resultaat 1};
        \node[document,minimum width=3cm] at (7.5,-15) (r2) {Resultaat 2};
        \node[document,minimum width=3cm] at (12.5,-15) (rn) {Resultaat $n$};

        \draw[arrow] (u1) --(r1);
        \draw[arrow] (u2) --(r2);
        \draw[arrow] (un) --(rn);

        \node[process,minimum size=1cm,ae] at (1,-16.5) (ae1) {PE};
        \node[process,minimum size=1cm,ie] at (2.5,-16.5) (ie1) {GE};
        \node[process,minimum size=1cm,se] at (4,-16.5) (se1) {SE};

        \draw[arrow,aea] (r1) --(ae1);
        \draw[arrow,iea] (r1) --(ie1);
        \draw[arrow,sea] (r1) --(se1);

        \node[process,minimum size=1cm,ae] at (6,-16.5) (ae2) {PE};
        \node[process,minimum size=1cm,ie] at (7.5,-16.5) (ie2) {GE};
        \node[process,minimum size=1cm,se] at (9,-16.5) (se2) {SE};

        \draw[arrow,aea] (r2) --(ae2);
        \draw[arrow,iea] (r2) --(ie2);
        \draw[arrow,sea] (r2) --(se2);

        \node[process,minimum size=1cm,ae] at (11,-16.5) (aen) {PE};
        \node[process,minimum size=1cm,ie] at (12.5,-16.5) (ien) {GE};
        \node[process,minimum size=1cm,se] at (14,-16.5) (sen) {SE};

        \draw[arrow,aea] (rn) --(aen);
        \draw[arrow,iea] (rn) --(ien);
        \draw[arrow,sea] (rn) --(sen);

        \node[document, minimum width=14cm, minimum height=3cm,tape bend height=0.5cm] at (7.5,-19) (b) {};

        \node[inner,minimum width=3cm,text height=4ex,minimum height=1.5cm] at (2.5,-18.75) (b1) {Beoordeling \\ context 1};
        \node[inner,minimum width=3cm,text height=4ex,minimum height=1.5cm] at (7.5,-18.75) (b2) {Beoordeling \\ context 2};
        \node[inner,minimum width=3cm,text height=4ex,minimum height=1.5cm] at (12.5,-18.75) (bn) {Beoordeling \\ context $n$};

        \draw[arrow,aea] (ae1) --(b1);
        \draw[arrow,iea] (ie1) --(b1);
        \draw[arrow,sea] (se1) --(b1);
        \draw[arrow,aea] (ae2) --(b2);
        \draw[arrow,iea] (ie2) --(b2);
        \draw[arrow,sea] (se2) --(b2);
        \draw[arrow,aea] (aen) --(bn);
        \draw[arrow,iea] (ien) --(bn);
        \draw[arrow,sea] (sen) --(bn);

        \node[nothing,fill=white] at (2,-20) {Beoordeling};

        \node[io] at (7.5,-21.5) (dodonaOut) {Dodona};

        \draw[arrow] (b) -- (dodonaOut);

    \end{tikzpicture}

\end{document}

    \caption{
        Flowchart van een beoordeling door TESTed.
        In het schema worden kleuren gebruikt als er een keuze gemaakt moet worden voor een volgende stap.
        Er kan steeds slechts één mogelijkheid gekozen worden.
        De afkortingen PE, GE en SE staan respectievelijk voor geprogrammeerde evaluatie, generieke evaluatie en (programmeertaal)specifieke evaluatie.
    }
    \label{fig:tested-flow}
\end{figure}


\section{Beschrijven van een oefening}\label{sec:testplan}

De beoordeling van een ingediende oplossing van een oefening begint bij de invoer die TESTed krijgt.
Centraal in deze invoer is een \term{testplan}, een specificatie die op een programmeertaalonafhankelijke manier beschrijft hoe een oplossing voor een oefening beoordeeld moet worden.
Het vervangt de taalspecifieke testen van de bestaande judges (ie.\ de jUnit-tests of de doctests in respectievelijk Java en Python).
Het testplan \latin{sensu lato} wordt opgedeeld in verschillende onderdelen, die hierna besproken worden.

\subsection{Het testplan}\label{subsec:het-testplan}

Het testplan \latin{sensu stricto} beschrijft de structuur van de beoordeling van een ingediende oplossing voor een oefening.
Deze structuur lijkt qua opbouw sterk op de structuur van de feedback zoals gebruikt door Dodona.
Dat de structuur van de oplossing in Dodona en van het testplan op elkaar lijken, heeft als voordeel dat er geen mentale afbeelding moet gemaakt worden tussen de structuur van het testplan en dat van Dodona.
Concreet is de structuur een hiërarchie met volgende elementen:

\begin{description}
    \item[Plan] Het top-level object van het testplan.
    Dit object bevat twee belangrijke objecten: de tabbladen en de configuratie.
    Deze configuratie is de plaats om opties aan TESTed mee te geven.
    \item[Tab] Een testplan bestaat uit verschillende \termen{tab}s of tabbladen.
    Deze komen overeen met de tabbladen in de gebruikersinterface van Dodona.
    Een tabblad kan een naam hebben, die zichtbaar is voor de gebruikers.
    \item[Context] Elk tabblad bestaat uit een of meerdere \termen{context}en.
    Een context is een onafhankelijke uitvoering van een evaluatie.
    De nadruk ligt op de "onafhankelijkheid", zoals al vermeld.
    Elke context wordt in een nieuw proces en in een eigen map (directory) uitgevoerd, zodat de kans op het delen van informatie klein is.
    Hierbij willen we vooral onbedoeld delen van informatie (zoals statische variabelen of het overschrijven van bestanden) vermijden.
    De gemotiveerde student zal nog steeds informatie kunnen delen tussen de uitvoeringen, door bv.\ in een andere locatie een bestand aan te maken en later te lezen.
    \item[Testcase] Een context bestaat uit een of meerdere \termen{testcase}s of testgevallen.
    Een testgeval bestaat uit invoer en een aantal tests.
    De testgevallen kunnen onderverdeeld worden in twee soorten:
    \begin{description}
        \item[Main testcase] of hoofdtestgeval.
        Van deze soort is er maximaal één per context (geen hoofdtestgeval is ook mogelijk).
        Dit testgeval heeft als doel het uitvoeren van de main-functie (of de code zelf als het gaat om een scripttaal zoals Bash of Python).
        Als invoer voor dit testgeval kunnen enkel de standaardinvoerstroom en de programma-argumenten meegegeven worden.
        De exitcode van een uitvoering kan ook enkel in het hoofdtestgeval gecontroleerd worden.
        \item[Normal testcase] of normaal testgeval.
        Hiervan kunnen er nul of meer zijn per context.
        Deze testgevallen dienen om andere aspecten van de ingediende oplossing te testen, nadat de code van de gebruiker met success ingeladen is.
        De invoer is dan ook uitgebreider: het kan gaan om het standaardinvoerkanaal, functieoproepen en variabeletoekenningen.
        Een functieoproep of variabeletoekenning is verplicht (zonder functieoproep of toekenning aan een variabele is er geen code om te testen).
    \end{description}
    Het hoofdtestgeval wordt altijd als eerste uitgevoerd.
    Dit is verplicht omdat bepaalde programmeertalen (zoals Python en andere scripttalen) de code onmiddellijk uitvoeren bij het inladen.
    Om te vermijden dat de volgorde van de testgevallen zou verschillen tussen de programmeertalen, wordt het hoofdtestgeval altijd eerst uitgevoerd.
    \item[Test] De beoordeling van een testgeval bestaat uit meerdere \term{test}s, die elk één aspect van het testgeval controleren.
    Met aspect bedoelen we de standaarduitvoerstroom, de standaardfoutstroom, opgevangen uitzonderingen (\english{exceptions}), de teruggegeven waarden van een functieoproep (returnwaarden) of de inhoud van een bestand.
    De exitcode is ook mogelijk, maar enkel in het hoofdtestgeval.
    Het beoordelen van de verschillende aspecten wordt in meer detail beschreven in \cref{sec:oplossingen-beoordelen}
\end{description}

Bij de keuze voor een formaat voor het testplan (\acronym{JSON}, \acronym{XML}, \ldots), zijn vooraf enkele vereisten geformuleerd waaraan het gekozen formaat moet voldoen.
Het moet:

\begin{itemize}
    \item leesbaar zijn voor mensen,
    \item geschreven kunnen worden met minimale inspanning, met andere woorden de syntaxis dient eenvoudig te zijn, en
    \item programmeertaalonafhankelijk zijn.
\end{itemize}

Uiteindelijk is gekozen om het testplan op te stellen in \acronym{JSON}.
Niet alleen voldoet \acronym{JSON} aan de vooropgestelde voorwaarden, het wordt ook door veel talen ondersteund.

Toch zijn er ook enkele nadelen aan het gebruik van \acronym{JSON}.
Zo is \acronym{JSON} geen beknopte of compacte taal om met de hand te schrijven.
Een oplossing hiervoor gebruikt de eigenschap dat veel talen \acronym{JSON} kunnen produceren: andere programma's kunnen desgewenst het testplan in het json-formaat genereren, waardoor het niet met de hand geschreven moet worden.
Hiervoor denken we aan een \acronym{DSL} (\english{domain specific language}), maar dit valt buiten de thesis en wordt verder besproken in \cref{ch:beperkingen-en-toekomstig-werk}.

Een tweede nadeel is dat \acronym{JSON} geen programmeertaal is.
Terwijl dit de implementatie van de judge bij het interpreteren van het testplan weliswaar eenvoudiger maakt, is het tevens beperkend: beslissen of een testgeval moet uitgevoerd worden op basis van het resultaat van een vorig testgeval is bijvoorbeeld niet mogelijk.
Ook deze beperking wordt uitgebreider besproken in \cref{ch:beperkingen-en-toekomstig-werk}.

Tot slot bevat \cref{lst:testplan} een testplan met één context voor de voorbeeldoefening Lotto uit \cref{ch:dodona}.

\begin{listing}
    \inputminted{python}{code/testplan.json}
    \caption{
        Een ingekorte versie van het testplan voor de voorbeeldoefening Lotto.
        Het testplan bevat maar één context.
    }
    \label{lst:testplan}
\end{listing}

\subsection{Dataserialisatie}\label{subsec:dataserialisatie}

Bij de beschrijving van het testplan wordt gewag gemaakt van returnwaarden en variabeletoekenningen.
Aangezien het testplan programmeertaalonafhankelijk is, moet er dus een manier zijn om data uit de verschillende programmeertalen voor te stellen en te vertalen: het \term{serialisatieformaat}.

\subsubsection{Keuze van het formaat}

Zoals bij het testplan, werd voor de voorstelling van waarden ook een keuze voor een bepaald formaat gemaakt.
Daarvoor werden opnieuw enkele voorwaarden vooropgesteld, waaraan het serialisatieformaat moet voldoen.
Het formaat moet:

\begin{itemize}
    \item door mensen geschreven kunnen worden (\english{human writable}),
    \item onderdeel van het testplan kunnen zijn,
    \item in meerdere programmeertalen bruikbaar zijn, en
    \item de basisgegevenstypes ondersteunen die we willen aanbieden in het programmeertaalonafhankelijke deel van het testplan.
    Deze gegevenstypes zijn:
    \begin{itemize}
        \item Primitieven: gehele getallen, reële getallen, Boolese waarden en tekenreeksen.
        \item Collecties: rijen (eindige, geordende reeks; \texttt{list} of \texttt{array}), verzamelingen (eindige, ongeordende reeks zonder herhaling; \texttt{set}) en afbeeldingen (elk element wordt afgebeeld op een ander element; \texttt{map}, \texttt{dict} of \texttt{object}).
    \end{itemize}
\end{itemize}

Een voor de hand liggende oplossing is om ook hiervoor \acronym{JSON} te gebruiken, en zelf in \acronym{JSON} een structuur op te stellen voor de waarden.
In tegenstelling tot de situatie bij het testplan bestaan er al een resem aan dataserialisatieformaten, waardoor het de moeite loont om na te gaan of er geen bestaand formaat voldoet aan de vereisten.
Hiervoor is gestart van een overzicht op Wikipedia \autocite{wiki2020}.
Uiteindelijk is niet gekozen voor een bestaand formaat, maar voor de \acronym{JSON}-oplossing.
De redenen hiervoor zijn samen te vatten als:

\begin{itemize}
    \item Het gaat om een binair formaat.
    Binaire formaten zijn uitgesloten op basis van de eerste twee voorwaarden die we opgesteld hebben: mensen kunnen het niet schrijven zonder hulp van bijkomende tools en het is moeilijk in te bedden in een \acronym{JSON}-bestand (zonder gebruik te maken van encoderingen zoals base64).
    Bovendien zijn binaire formaten moeilijker te implementeren in sommige talen.
    \item Het formaat ondersteunt niet alle gewenste types.
    Sommige formaten hebben ondersteuning voor complexere datatypes, maar niet voor alle complexere datatypes die wij nodig hebben.
    Uiteraard kunnen de eigen types samengesteld worden uit basistypes, maar dan biedt de ondersteuning voor de complexere types weinig voordeel, aangezien er toch een eigen dataschema voor die complexere types opgesteld zal moeten worden.
    \item Sommige formaten zijn omslachtig in gebruik.
    Vaak ondersteunen dit soort formaten meer dan wat wij nodig hebben.
    \item Het formaat is niet eenvoudig te implementeren in een programmeertaal waarvoor geen ondersteuning is.
    Sommige dezer formaten ondersteunen weliswaar veel talen, maar we willen niet dat het serialisatieformaat een beperkende factor wordt in welke talen door de judge ondersteund worden.
    Het mag niet de bedoeling zijn dat het implementeren van het serialisatieformaat het meeste tijd in beslag neemt.
\end{itemize}

Een lijst van de overwogen formaten met een korte beschrijving:

\begin{description}
    \item[Apache Avro] Een volledig "systeem voor dataserialisatie".
    De specificatie van het formaat gebeurt in \acronym{JSON} (vergelijkbaar met \acronym{JSON} Schema), terwijl de eigenlijke data binair geëncodeerd wordt.
    Heeft uitbreidbare types, met veel ingebouwde types \autocite{avro}.
    \item[Apache Parquet] Minder relevant, dit is een bestandsformaat voor Hadoop \autocite{parquet}.
    \item[\acronym{ASN}.1] Staat voor \english{Abstract Syntax Notation One}, een formaat uit de telecommunicatie.
    De hoofdstandaard beschrijft enkel de notatie voor een dataformaat.
    Andere standaarden beschrijven dan de serialisatie, bijvoorbeeld een binair formaat, \acronym{JSON} of \acronym{XML}.
    De meerdere serialisatievormen zijn in theorie aantrekkelijk: elke taal moet er slechts een ondersteunen, terwijl de judge ze allemaal kan ondersteunen.
    In de praktijk blijkt echter dat voor veel talen er slechts één serialisatieformaat is, en dat dit vaak het binaire formaat is \autocite{x680}.
    \item[Bencode] Schema gebruikt in BitTorrent.
    Het is gedeeltelijk binair, gedeeltelijk in text \autocite{cohen2017}.
    \item[Binn] Binair dataformaat \autocite{ramos2019}.
    \item[\acronym{BSON}] Een binaire variant op \acronym{JSON}, geschreven voor en door MongoDB \autocite{bson}.
    \item[\acronym{CBOR}] Een lichtjes op \acronym{JSON} gebaseerd formaat, ook binair.
    Heeft een goede standaard, ondersteunt redelijk wat talen \autocite{rfc7049}.
    \item[FlatBuffers] Lijkt op ProtocolBuffers, allebei geschreven door Google, maar verschilt wat in implementatie van ProtocolBuffers.
    De encodering is binair \autocite{flatbuffers}.
    \item[Fast Infoset] Is eigenlijk een manier om \acronym{XML} binair te encoderen (te beschouwen als een soort compressie voor xml), waardoor het minder geschikt voor ons gebruik wordt \autocite{x981}.
    \item[Ion] Een superset van \acronym{JSON}, ontwikkeld door Amazon.
    Het heeft zowel een tekstuele als binaire voorstelling.
    Naast de gebruikelijke \acronym{JSON}-types, bevat het enkele uitbreidingen. \autocite{ion}.
    \item[MessagePack] Nog een binair formaat dat lichtjes op \acronym{JSON} gebaseerd is.
    Lijkt qua types sterk op \acronym{JSON}.
    Heeft implementaties in veel talen \autocite{messagepack}.
    \item[\acronym{OGDL}] Afkorting voor \english{Ordered Graph Data Language}.
    Daar het om een serialisatieformaat voor grafen gaat, is het niet nuttig voor ons doel \autocite{ogdl}.
    \item[\acronym{OPC} Unified Architecture] Een protocol voor intermachinecommunicatie.
    Complex: de specificatie bevat 14 documenten, met ongeveer 1250 pagina's \autocite{tr62541}.
    \item[Open\acronym{DLL}] Afkorting voor de \english{Open Data Description Language}.
    Een tekstueel formaat, bedoeld om arbitraire data voor te stellen.
    Wordt niet ondersteund in veel programmeertalen, in vergelijking met bijvoorbeeld \acronym{JSON} \autocite{openddl}.
    \item[ProtocolBuffers] Lijkt zoals vermeld sterk op FlatBuffers, maar heeft nog extra stappen nodig bij het encoderen en decoderen, wat het minder geschikt maakt \autocite{protobuf}.
    \item[Smile] Nog een binaire variant van \acronym{JSON} \autocite{smile}.
    \item[\acronym{SOAP}] Afkorting voor \english{Simple Object Access Protocol}.
    Niet bedoeld als formaat voor dataserialisatie, maar voor communicatie tussen systemen over een netwerk \autocite{soap}.
    \item[\acronym{SDXF}] Binair formaat voor data-uitwisseling.
    Weinig talen ondersteunen dit formaat \autocite{rfc3072}.
    \item[Thrift] Lijkt sterk op ProtocolBuffers, maar geschreven door Facebook \autocite{slee2007}.
    \item[\acronym{UBJSON}] Nog een binaire variant van \acronym{JSON} \autocite{ubjson}.

\end{description}

Geen enkel overwogen formaat heeft grote voordelen tegenover een eigen structuur in \acronym{JSON}.
Daarenboven hebben veel talen het nadeel dat ze geen \acronym{JSON} zijn, waardoor we een nieuwe taal moeten inbedden in het bestaande \acronym{JSON}-testplan.
Dit nadeel, gekoppeld met het ontbreken van voordelen, heeft geleid tot de keuze voor \acronym{JSON}.

\subsubsection{Dataschema}

Json is slechts een formaat en geeft geen semantische betekenis aan \acronym{JSON}-elementen.
Hiervoor stellen we een dataschema op, dat uit twee onderdelen bestaat:

\begin{itemize}
    \item Het encoderen van waarden.
    \item Het beschrijven van de gegevenstypes van deze waarden.
\end{itemize}

Elke waarde wordt in het serialisatieformaat voorgesteld als een object met twee elementen: de geëncodeerde waarde en het bijhorende gegevenstype.
Een concreet voorbeeld is \cref{lst:serialisation}.

\begin{listing}
    \inputminted{json}{code/format.json}
    \caption{Een lijst bestaande uit twee getallen, geëncodeerd in het serialisatieformaat.}
    \label{lst:serialisation}
\end{listing}

Het encoderen van waarden slaat op het voorstellen van waarden als \acronym{JSON}-waarden.
Json heeft slechts een beperkt aantal gegevenstypes, dus worden alle waarden voorgesteld als een van deze types.
Zo worden bijvoorbeeld zowel \texttt{array}s en \texttt{set}s voorgesteld als een \acronym{JSON}-lijst.

Het verschil tussen beiden wordt dan duidelijk gemaakt door het bijhorende gegevenstype.
Er is dus nood aan een systeem om aan te geven wat het gegevenstype van een waarde is.

\Cref{lst:type-schema} bevat het onder andere de structuur van een waarde in het serialisatieformaat, in een vereenvoudigde versie van \acronym{JSON} Schema.
Hierbij staat \texttt{<types>} voor een van de gegevenstypes die hierna besproken werden.

\begin{listing}
    \inputminted{json}{code/type-schema.json}
    \caption{Het schema voor waarden, expressies en statements, in een vereenvoudigde versie van \acronym{JSON} Schema.}
    \label{lst:type-schema}
\end{listing}

\subsubsection{Gegevenstypes}

Het systeem om de gegevenstypes aan te duiden vervult meerdere functies.
Het wordt gebruikt om:

\begin{itemize}
    \item het gegevenstype van concrete data aan te duiden (beschrijvende modus).
    Dit gaat om de serialisatie van waarden uit de uitvoeringsomgeving naar TESTed, zoals het geval is bij returnwaarden van functies.
    \item te beschrijven welk gegevenstype verwacht wordt (voorschrijvende modus).
    Een voorbeeld hiervan is het aangeven van het gegevenstype van een variabele.
    \item zelf code te schrijven (letterlijke modus).
    Dit gaat om serialisatie vanuit het testplan zelf naar de uitvoeringsomgeving.
    Een voorbeeld hiervan is het opnemen van functieargumenten in het testplan: deze argumenten worden tijdens de serialisatie omgezet naar echte code.
\end{itemize}

Bij het ontwerp van het systeem voor de gegevenstypes zorgen deze verschillende functies soms voor tegenstrijdige belangen: voor het beschrijven van een waarde moet het systeem zo eenvoudig mogelijk zijn.
Een waarde met bijhorend gegevenstype \texttt{union[string, int]} is niet bijster nuttig: een waarde kan nooit tegelijk een \texttt{string} en een \texttt{int} zijn.
Aan de andere kant zijn dit soort complexe gegevenstypes wel nuttig bij het aangeven van het verwachte gegevenstype van bijvoorbeeld een variabele.
Daarnaast moet ook rekening gehouden worden met het feit dat deze gegevenstypes in veel programmeertalen implementeerbaar moeten zijn.
Een gegevenstype als \texttt{union[string, int]} is eenvoudig te implementeren in Python, maar dat is niet het geval in bijvoorbeeld Java of C\@.
Ook heeft elke programmeertaal een eigen niveau van details bij gegevenstypes.
Python heeft bijvoorbeeld enkel \texttt{integer} voor gehele getallen, terwijl C beschikt over \texttt{int}, \texttt{unsigned}, \texttt{long}, enz.
Daarenboven heeft het schrijven van code bijkomende vereisten: als functieargument zijn waarden alleen niet voldoende, ook andere variabelen moeten gerefereerd kunnen worden en het resultaat van andere functieoproepen moeten ook als argument gebruikt kunnen worden.

Om deze redenen zijn de gegevenstypes opgedeeld in drie categorieën:

\begin{enumerate}
    \item De basistypes.
    Deze gegevenstypes zijn bruikbaar in alle modi.
    De lijst van basistypes omvat:
    \begin{description}
        \item[\texttt{integer}] Gehele getallen, zowel positief als negatief.
        \item[\texttt{rational}] Rationale getallen.
        Het gaat hier om \texttt{float}s, die ook vaak gebruikt worden als benadering van gehele getallen.
        \item[\texttt{text}] Een tekenreeks of string (alle vormen).
        \item[\texttt{char}] Een enkel teken.
        \item[\texttt{boolean}] Een Boolese waarde (of boolean).
        \item[\texttt{sequence}] Een wiskundige rij, wat wil zeggen dat de volgorde belangrijk is en dat dubbele elementen toegelaten zijn.
        \item[\texttt{set}] Een wiskundige verzameling, wat wil zeggen dat de volgorde niet belangrijk is en dat dubbele elementen niet toegelaten zijn.
        \item[\texttt{map}] Een wiskundige afbeelding: elk element wordt afgebeeld op een ander element.
        In Java is dit bijvoorbeeld een \texttt{Map}, in Python een \texttt{dict} en in Javascript een \texttt{object}.
        \item[\texttt{nothing}] Geeft aan dat er geen waarde is, ook wel \texttt{null}, \texttt{None} of \texttt{nil} genoemd.
    \end{description}
    Een lijst van de implementaties in de verschillende programmeertalen is \cref{tab:basistypes}.
    Elke implementatie van een programmeertaal moet een keuze maken wat de standaardimplementatie van deze types is.
    Zo implementeert de Java-implementatie het gegevenstype \texttt{sequence} als een \texttt{List<>}, niet als een \texttt{array}.
    Een implementatie in een programmeertaal kan ook aangeven dat een bepaald type niet ondersteund wordt, waardoor testplannen met dat type niet zullen werken.
    \item De uitgebreide types: dit zijn een hele reeks bijkomende types.
    Deze gegevenstypes staan toe om meer details over de types te serialiseren en in het testplan op te nemen.
    Een voorbeeld is de lijst van types in \cref{tab:vertaling}, die voor een reeks gegevenstypes voor gehele getallen de concrete types in verschillende programmeertalen geeft.
    Het grote verschil is dat deze uitgebreide types standaard vertaald worden naar een van de basistypes.
    Voor talen die bijvoorbeeld geen \texttt{tuple} uit Python ondersteunen, zal het type omgezet worden naar \texttt{list}.
    Er is ook de mogelijk dat implementaties voor programmeertalen expliciet een bepaald type niet ondersteunen.
    Zo zal de Java-implementatie geen \texttt{uint64} (een unsigned 64-bit integer) ondersteunen, omdat er geen equivalent bestaat in de taal\footnote{Dit is slechts ter illustratie: in de implementatie van TESTed wordt \texttt{BigInteger} gebruikt.}.
    \item Voorschrijvende types.
    Gegevenstypes in deze categorie kunnen enkel gebruikt worden bij het aangeven welk gegevenstype verwacht wordt, niet bij de eigenlijke encodering van waarden.
    In de praktijk gaat het om het type van variabelen.
    In deze categorie zouden gegevenstypes als \texttt{union[str, int]} komen.
    Er is echter expliciet gekozen om dit soort types niet te ondersteunen, door de moeilijkheid om dit te implementeren in statisch getypeerde talen, zoals Java of C\@.
    Twee types die wel ondersteund worden in deze modus zijn:
    \begin{description}
        \item[any] Het \texttt{any}-type geeft aan dat het type van een variabele onbekend is.
        Merk op dat dit in sommige talen tot moeilijkheden zal leiden: zo zal dit in C-code als \texttt{long} beschouwd worden (want C heeft geen equivalent van een \texttt{any}-type).
        \item[custom] Een eigen type, waarbij de naam van het type gegeven wordt.
        Dit is nuttig om bijvoorbeeld variabelen aan te maken met als gegevenstype een eigen klasse, zoals een klasse die de student moest implementeren.
    \end{description}
\end{enumerate}

\begin{table}
    \centering
    \caption{Implementaties van de basistypes in de verschillende programmeertalen.}
    \label{tab:basistypes}
    \begin{tabular}{|l|lll|}
        \hline
        Type              & Python          & Java              & Haskell           \\
        \hline
        \texttt{integer}  & \texttt{int}    & \texttt{long}     & \texttt{Integer}  \\
        \texttt{rational} & \texttt{float}  & \texttt{double}   & \texttt{Double}   \\
        \texttt{text}     & \texttt{str}    & \texttt{String}   & \texttt{String}   \\
        \texttt{char}     & \texttt{str}    & \texttt{char}     & \texttt{Char}     \\
        \texttt{boolean}  & \texttt{bool}   & \texttt{boolean}  & \texttt{Boolean}  \\
        \texttt{sequence} & \texttt{list}   & \texttt{List<>}   & \texttt{List}     \\
        \texttt{set}      & \texttt{set}    & \texttt{Set<>}    & -        \\
        \texttt{map}      & \texttt{dict}   & \texttt{Map<>}    & -        \\
        \texttt{nothing}  & \texttt{None}   & \texttt{null}     & \texttt{Nothing}  \\
        \hline
    \end{tabular}
\end{table}

\begin{table}
    \centering
    \caption{Voorbeeld van de implementatie van types voor gehele getallen, met als basistype \texttt{integer}.}
    \label{tab:vertaling}
    \begin{threeparttable}
        \begin{tabular}{|l|llllllll|}
            \hline
                       & \texttt{int8} & \texttt{uint8} & \texttt{int16} & \texttt{uint16} & \texttt{int32} & \texttt{uint32} & \texttt{int64} & \texttt{uint64} \\
            \hline
            Python     & \texttt{int}  & \texttt{int}   & \texttt{int}   & \texttt{int}    & \texttt{int}   & \texttt{int}    & \texttt{int}   & \texttt{int}    \\
            Java       & \texttt{byte} & \texttt{short} & \texttt{short} & \texttt{int}    & \texttt{int}   & \texttt{long}   & \texttt{long}  & -               \\
            C\tnote{1} & \texttt{int8\_t} & \texttt{uint8\_t} & \texttt{int16\_t} & \texttt{uint16\_t} & \texttt{int32\_t} & \texttt{uint32\_t} & \texttt{int64\_t} & \texttt{uint64\_t} \\
            Haskell    & \texttt{Int8} & \texttt{Word8} & \texttt{Int16} & \texttt{Word16} & \texttt{Int32} & \texttt{Word32} & \texttt{Int64} & \texttt{Word64} \\
            \hline
        \end{tabular}
    \begin{tablenotes}
        \item[1] Uiteraard met de gebruikelijke aliassen van \texttt{short}, \texttt{unsigned}, \ldots
    \end{tablenotes}
    \end{threeparttable}
\end{table}

\subsection{Expressions en statements}\label{subsec:expressions-and-statements}

Een ander onderdeel van het testplan verdient ook speciale aandacht: toekennen van waarden aan variabelen (\english{assignments}) en functieoproepen.

In heel wat oefeningen, en zeker bij objectgerichte en imperatieve programmeertalen, is het toekennen van een waarde aan een variabele, om deze later te gebruiken, onmisbaar.
Bijvoorbeeld zou een opgave kunnen bestaan uit het implementeren van een klasse.
Bij de evaluatie dient dan een instantie van die klasse aangemaakt te worden, waarna er methoden kunnen aangeroepen worden, zoals hieronder geïllustreerd in een fictief voorbeeld.

\inputminted{java}{code/assignment.jshell}

Om deze reden is het testplan uitgebreid met ondersteuning voor statements en expressies.
Toch moet meteen opgemerkt worden dan deze ondersteuning beperkt is tot wat er nodig is om het scenario van hiervoor te kunnen uitvoeren;
het is zeker niet de bedoeling om een volledige eigen programmeertaal te ontwerpen.

In \cref{lst:type-schema} staat onder andere het formaat van een expressie en een functieoproep in een vereenvoudigde versie van \acronym{JSON} Schema.
Een expressie is een dezer drie dingen:
\begin{enumerate}
    \item Een waarde, zoals hiervoor besproken in subparagraaf \emph{Dataschema} van \cref{subsec:dataserialisatie}.
    \item Een \texttt{identifier}, voorgesteld als een string.
    \item Een functieoproep, die bestaat uit:
    \begin{description}
        \item[\texttt{type}] Het soort functie.
        Kan een van deze waarden zijn:
        \begin{description}
            \item[\texttt{function}] Een \english{top-level} functie.
            Afhankelijk van de programmeertaal zal deze functie toch omgezet worden naar een \texttt{namespace}-functie.
            Zo worden dit soort functies in Java omgezet naar statische functies.
            \item[\texttt{namespace}] Een methode (functie van een object) of een functie in een namespace.
            De invulling hiervan is gedeeltelijk programmeertaalafhankelijk: in Java gaat het om methodes, terwijl het in Haskell om functies van een module gaat.
            Bij dit soort functies moet de \texttt{namespace} gegeven worden.
            \item[\texttt{constructor}] Deze soort functie heeft dezelfde semantiek als een top-level functie, met dien verstande dat het om een constructor gaat.
            In Java zal bijvoorbeeld het keyword \texttt{new} vanzelf toegevoegd worden.
            De functienaam doet dienst als naam van de klasse.
            \item[\texttt{property}] De property van een instantie wordt gelezen.
            Deze soort functie heeft dezelfde semantiek van een namespace-functie, maar heeft geen argumenten.
        \end{description}
        \item[\texttt{namespace}] De namespace van de functie.
        \item[\texttt{name}] De naam van de functie.
        \item[\texttt{arguments}] De argumenten van de functie.
        Dit is een lijst van expressies.
    \end{description}
\end{enumerate}

De ondersteuning voor statements in het testplan beperkt zich tot variabeletoekenningen of \english{assignment}s.
Er is expliciet voor gekozen om expressies geen statements te maken.
De reden hiervoor is dat dit de implementatie ingewikkelder zou maken, zonder noemenswaardig voordeel.
Een assignment kent een naam toe aan het resultaat van een expression.
\Cref{lst:type-schema} toont ook de vereenvoudigde \acronym{JSON} Schema van een statement (en dus van een assignment, daar er maar één soort statement bestaat).
Hier staat \texttt{<datatype>} voor een van de gegevenstypes die hiervoor besproken zijn.

De \texttt{name} is de naam die aan de variabele gegeven zal worden.
Het veldje \texttt{expression} moet een expressie zijn, zoals reeds besproken.
Ook moet het gegevenstype van de variabele gegeven worden.
Hiervoor kunnen types het het serialisatieformaat gebruikt worden, inclusief de types uit de letterlijke modus.

Een gecombineerd voorbeeld staat hieronder.
Hier wordt de string \texttt{'Dodona'} toegekend aan een variabele met naam \texttt{name}.

\inputminted{json}{code/assign-variable.json}

Tot slot is het nog het vermelden waard dat waarden van de gegevenstypes \texttt{sequence} en \texttt{map} als elementen geen andere waarden hebben, maar expressies.
Dit niet het geval in de beschrijvende modus van de gegevenstypes, bijvoorbeeld bij het aangeven wat de verwachte returnwaarde van een functie is.
Het testplan biedt namelijk geen ondersteuning voor het serialiseren van identifiers en functieoproepen, enkel waarden.
Dit betekent dat constructies zoals deze mogelijk zijn in het testplan:

\inputminted{java}{code/advanced.jshell}

\subsection{Controle ondersteuning voor programmeertalen}\label{subsec:vereiste-functies}

In het stappenplan uit \cref{sec:ontwerp} is al vermeld dat vóór een beoordeling start, een controle plaatsvindt om zeker te zijn dat het testplan uitgevoerd kan worden in de programmeertaal van de ingediende oplossing.
Bij de controle worden volgende zaken nagekeken:
\begin{itemize}
    \item Controle of de programmeertaal de nodige gegevenstypes ondersteunt.
    Dit gaat van de basistypes (zoals \texttt{sequence}) tot de geavanceerde types (zoals \texttt{tuple}).
    Voor elke programmeertaal binnen TESTed wordt bijgehouden welke types ondersteund worden en welke niet.
    Bevat een testplan bijvoorbeeld waarden met als type \texttt{set} (verzamelingen), dan kunnen enkel programmeertalen die verzamelingen ondersteunen gebruikt worden.
    Dat zijn bijvoorbeeld Python en Java, maar geen Bash.
    \item Controle of de programmeertaal over de nodige taalconstructies beschikt, zoals \texttt{exceptions} of \texttt{objects}.
    Bij deze controle wordt ook gecontroleerd of dat de programmeertaal optionele functieargumenten of functieargumenten met heterogene gegevenstypes nodig heeft.
    Een voorbeeld van een functie met argumenten met heterogene gegevenstypes komt bijvoorbeeld uit de \acronym{ISBN}-oefening (deze oefening wordt besproken in \cref{ch:nieuwe-oefening}):
    \begin{minted}{pycon}
>>> is_isbn("9789027439642")
True
>>> is_isbn(9789027439642)
False
    \end{minted}
    In talen als Python en Java kan deze functie geïmplementeerd worden, maar in talens als C is dat veel moeilijker.
\end{itemize}

\section{Oplossingen uitvoeren}\label{sec:oplossingen-uitvoeren}

De eerste stap die wordt uitgevoerd bij de beoordeling van een ingediende oplossing is het genereren van de testcode, die de ingediende oplossing zal beoordelen.

\subsection{Testcode genereren}\label{subsec:testcode-genereren}

Het genereren van de testcode gebeurt met een sjabloonsysteem genaamd Mako \autocite{mako}.
Dit soort systemen wordt traditioneel gebruikt bij webapplicaties (zoals Ruby on Rails met \acronym{ERB}, Phoenix met \acronym{EEX}, Laravel met Blade, enz.) om bijvoorbeeld html-pagina's te genereren.
In ons geval zijn de sjablonen verantwoordelijk voor de vertaling van programmeertaalonafhankelijke specificaties in het testplan naar concrete testcode in de programmeertaal van de ingediende oplossing.
Hierbij denken we aan de functieoproepen, assignments, enz.
Ook zijn de sjablonen verantwoordelijk voor het genereren van de code die de oplossing van de student zal oproepen en evalueren.

\subsubsection{Sjablonen}

TESTed heeft een aantal standaardsjablonen nodig, waaraan vastgelegde parameters meegegeven worden en die een vaste functie moeten uitvoeren.
Deze verplichte sjablonen zijn:
\begin{description}
    \item[\texttt{assignment}] Vertaalt een toekenningsopdracht uit het testplan naar code.
    \item[\texttt{context}] Een sjabloon dat code genereert om een context te beoordelen.
    Deze code moet uitvoerbaar zijn (dat wil zeggen een main-functie bevatten of een script zijn).
    \item[\texttt{selector}] Een sjabloon dat code genereert om een bepaalde context uit te voeren.
    Om performantieredenen (hierover later meer) wordt de code van alle contexten soms uit een keer gegenereerd en gecompileerd.
    Aan de hand van een parameter (de naam van de context), wordt bij het uitvoeren van deze selectiecode de testcode voor de juiste context gekozen.
    Dit sjabloon is enkel nodig indien batchcompilatie ondersteund wordt en de programmeertaal dit nodig heeft (bijvoorbeeld niet nodig in Python, maar wel in Java).
    \item[\texttt{evaluator\_executor}] Een sjabloon dat code genereert om een geprogrammeerde evaluatie te starten.
    \item[\texttt{function}] Vertaalt een functie-oproep naar testcode.
\end{description}

Daarnaast moet het encoderen naar het serialisatieformaat ook geïmplementeerd worden in elke programmeertaal.
Veel programmeertalen hebben dus nog enkele bijkomende bestanden met code.
In alle bestaande configuraties van programmeertalen is dit geïmplementeerd als een module of een klasse met naam \texttt{Value}.
Dit wordt geïllustreerd in \cref{ch:nieuwe-taal}, dat het toevoegen van een nieuwe programmeertaal aan TESTed volledig uitwerkt.

\subsubsection{Testcode compileren}

TESTed ondersteunt twee modi waarin de code gecompileerd kan worden (bij programmeertalen die geen compilatie ondersteunen wordt deze stap overgeslagen):

\begin{description}
    \item[Batchcompilatie] In deze modus wordt de code voor alle contexten in een keer gecompileerd.
    Dit wordt gedaan om performantieredenen.
    In talen die resulteren in een uitvoerbaar bestand (zoals Haskell, C/C++), resulteert deze modus in één uitvoerbaar bestand voor alle contexten.
    Bij het uitvoeren wordt dan aan de hand van een parameter de juiste context uitgevoerd (met het \texttt{selector}-sjabloon van hierboven).
    \item[Contextcompilatie] Hierbij wordt elke context afzonderlijk gecompileerd.
\end{description}

Dit wordt getoond in \cref{fig:tested-flow} uit \cref{sec:ontwerp} door twee kleuren te gebruiken: de stappen die enkel gebeuren bij batchcompilatie zijn in het \textcolor{ugent-ps}{groen}, terwijl stappen die enkel bij contextcompilatie gebeuren in het \textcolor{ugent-we}{blauw} staan.
Stappen die altijd gebeuren staan in de flowchart in het zwart.

Dit gedrag is configureerbaar in het testplan, maar standaard wordt de batchcompilatie gebruikt.
Als er een compilatiefout optreed bij de compilatie in batchcompilatie, wordt valt TESTed terug op contextcompilatie.
Deze terugval is handig voor programmeertalen waar de compilatie veel fouten ontdekt (vaak de meer statische programmeertalen).
Een voorbeeldscenario is als volgt: stel een oefening waarbij de student twee functies moet implementeren.
De student implementeert de eerste functie en dient een oplossing in om al feedback te krijgen.
Bij programmeertalen als Java of Haskell zal dit niet lukken: daar alle contexten in één keer gecompileerd worden, zal de ontbrekende tweede functie ervoor zorgen dat de volledige compilatie faalt.
In individuele modus is dit geen probleem: de contexten die de eerste functie testen zullen compileren en kunnen uitgevoerd worden.
De individuele modus brengt wel een niet te verwaarlozen kost qua uitvoeringstijd met zich mee (zie ook \cref{ch:beperkingen-en-toekomstig-werk}).

\Cref{lst:generated-context-python,lst:generated-context-java} bevatten de testcode gegenereerd voor een context uit de voorbeeldoefening Lotto (het gaat om dezelfde context uit het voorbeeld van het testplan in \cref{lst:testplan}), in respectievelijk Python en Java.
Daarnaast bevat Z de code voor de \texttt{selector} in Java.
Hiervan is geen versie in Python, daar Python selector nodig heeft in batchcompilatie (in Python kunnen meerdere onafhankelijke bestanden tegelijk gecompileerd worden).
De selector bevat twee contexten om de werking duidelijk te maken.

\begin{listing}
    \inputminted{python}{code/generated-context-1.py}
    \caption{
        Gegenereerde testcode in Python voor de eerste context uit het testplan van de voorbeeldoefening Lotto.
    }
    \label{lst:generated-context-python}
\end{listing}

\begin{listing}
    \inputminted{java}{code/generated-context-1.java}
    \caption{
        Gegenereerde testcode in Java voor de eerste context uit het testplan van de voorbeeldoefening Lotto.
        Enkele hulpfuncties en imports zijn verwijderd om de code korter te maken.
    }
    \label{lst:generated-context-java}
\end{listing}

\begin{listing}
    \inputminted{java}{code/Selector.java}
    \caption{
        Gegenereerde selectiecode in Java voor twee contexten uit het testplan van de voorbeeldoefening Lotto.
    }
    \label{lst:selector-java}
\end{listing}

\subsection{Testcode uitvoeren}\label{subsec:testcode-uitvoeren}

Vervolgens wordt de (gecompileerde) testcode voor elke context uit het testplan afzonderlijk uitgevoerd en worden de resultaten (het gedrag en de neveneffecten) verzameld.
Het uitvoeren zelf gebeurt op de normale manier waarop code voor de programmeertaal uitgevoerd wordt: via de commandoregel.
Deze aanpak heeft als voordeel dat er geen verschil is tussen hoe TESTed de ingediende code uitvoert en hoe de student zijn code zelf uitvoert op zijn eigen computer.
Dit voorkomt dat er subtiele verschillen in de resultaten sluipen.

\Cref{lst:mapstructuur} illustreert dit met een voorbeeld voor een ingediende oplossing in de programmeertaal Python.
Deze mapstructuur stelt de toestand van de werkmap van TESTed voor na het uitvoeren van de code.
In de map \texttt{common} zit alle testcode en de gecompileerde bestanden voor alle contexten.
Voor elke context worden de gecompileerde bestanden gekopieerd naar een andere map, bv.\ \texttt{context\_0\_1}, wat de map is voor context \texttt{1} van tabblad \texttt{0} van het testplan.

\begin{listing}
    \inputminted{text}{code/dir-listing.txt}
    \caption{Mapstructuur na het uitvoeren van de testcode van een oplossing in Python.
    \texttt{context\_0\_0} staat voor de eerste context van het eerste tabblad.
    }
    \label{lst:mapstructuur}
\end{listing}

\subsection{Beoordelen van gedrag}\label{subsec:beoordelen-van-gedrag}

Het uitvoeren van de testcode genereert resultaten (gedrag en neveneffecten) die door TESTed beoordeeld moeten worden.
Er zijn verschillende soorten gedragingen en neveneffecten die interessant zijn.
Elke soort gedrag of neveneffect wordt een \term{uitvoerkanaal} genoemd.
TESTed verzamelt volgende uitvoerkanalen:
\begin{itemize}
    \item De standaarduitvoerstroom.
    Dit wordt verzameld als tekstuele uitvoer.
    \item De standaardfoutstroom.
    Ook dit wordt als tekst verzameld.
    \item Fatale uitzonderingen.
    Hiermee bedoelen we uitzonderingen die tot aan de testcode geraken.
    Een uitzondering die afgehandeld wordt door de ingediende oplossing wordt niet verzameld.
    De uitzonderingen worden verzameld in een bestand.
    \item Returnwaarden.
    Deze waarden worden geëncodeerd en ook verzameld in een bestand.
    \item Exitcode.
    Het gaat om de exitcode van de testcode voor een context.
    Daar de code per context wordt uitgevoerd, wordt de exitcode ook verzameld per context (en niet per testcase, zoals de andere uitvoerkanalen).
    \item Bestanden.
    Tijdens het beoordelen van de verzamelde resultaten is het mogelijk de door de ingediende oplossing gemaakte bestanden te bekijken.
\end{itemize}

De standaarduitvoer- en standaardfoutstroom worden rechtstreeks opgevangen door TESTed.
De andere uitvoerkanalen (uitzonderingen en returnwaarden) worden naar een bestand geschreven.
De reden dat deze niet naar een andere \term{file descriptor} geschreven worden is eenvoudig: niet alle talen (zoals Java) ondersteunen het openen van bijkomende file descriptors.

Alle uitvoerkanalen (met uitzondering van de exitcode en de bestanden) worden per testcase verzameld.
Aangezien de uitvoerkanalen pas verzameld worden na het uitvoeren van de context, moet er een manier zijn om de uitvoer van de verschillende testgevallen te onderscheiden.
De testcode is hier verantwoordelijk voor, en schrijft een \english{separator} naar alle uitvoerkanalen tussen elk testgeval, zoals te zien is in \cref{lst:uitvoer}.

\begin{listing}
    \begin{minted}{text}
    {"data":"1 - 3 - 6 - 8 - 10 - 15","type":"text"}--gL9koJNv3-- SEP
    \end{minted}
    \caption{Voorbeeld van het uitvoerkanaal voor returnwaarden na het uitvoeren van de eerste context uit de voorbeeldoefening Lotto.}
    \label{lst:uitvoer}
\end{listing}

Tijdens het genereren van de code krijgen de sjablonen een reeks willekeurige tekens mee, de \english{secret}.
Deze secret wordt gebruikt voor verschillende dingen, zoals:
\begin{itemize}
    \item De separator.
    Door het gebruik van de willekeurige tekens is de kans dat de separator overeenkomt met een echte waarde praktisch onbestaand.
    \item Bestandsnamen.
    De testcode is verantwoordelijk voor het openen van de bestanden voor de uitvoerkanalen die naar een bestand geschreven worden.
    Bij het openen zal de testcode de secret in de bestandsnaam gebruiken.
    Dit is om het per abuis overschrijven van deze bestanden door de ingediende oplossing tegen te gaan.
\end{itemize}

\section{Oplossingen beoordelen}\label{sec:oplossingen-beoordelen}

Na het uitvoeren van de testcode voor elke context heeft TESTed alle relevante uitvoer gemeten en verzameld.
Deze uitvoer moet vervolgens beoordeeld worden om na te gaan in hoeverre deze uitvoer voldoet aan de verwachte uitvoer.
Dit kan op drie manieren:
\begin{enumerate}
    \item Generieke evaluatie: de uitvoer wordt beoordeeld door TESTed zelf.
    \item Geprogrammeerde evaluatie: de uitvoer wordt beoordeeld door programmacode geschreven door degene die de oefening opgesteld heeft, in een aparte omgeving (de evaluatieomgeving).
    \item Programmeertaalspecifieke evaluatie: de uitvoer wordt onmiddellijk na het uitvoeren van de testcode beoordeeld in het hetzelfde proces.
\end{enumerate}

\subsection{Generieke evaluatie}\label{subsec:ingebouwde-evaluator}

Voor eenvoudige beoordelingen (bijvoorbeeld tussen twee waarden) volstaat de generieke evaluatie binnen TESTed.
Het is mogelijk om de verwachte resultaten in het testplan op te nemen.
TESTed zal deze resultaten uit het testplan dan vergelijken met de resultaten geproduceerd door het uitvoeren van de testcode.
Als \english{proof of concept} zijn drie eenvoudige evaluatiemethoden ingebouwd in TESTed, die hieronder besproken worden.

\subsubsection{Tekstevaluatie}

Deze evaluator vergelijkt de verkregen uitvoer van een uitvoerkanaal (standaarduitvoer, standaardfout, \ldots) met de verwachte uitvoer uit het testplan.
Deze evaluator biedt enkele opties om het gedrag aan te passen:

\begin{description}
    \item[\texttt{ignoreWhitespace}]
    Witruimte voor en na het resultaat wordt genegeerd.
    Dit gebeurt op de volledige tekst, niet regel per regel.
    \item[\texttt{caseInsensitive}] Er wordt geen rekening gehouden met het verschil tussen hoofdletters en kleine letters.
    \item[\texttt{tryFloatingPoint}]
    De tekst zal geïnterpreteerd worden als een zwevendekommagetal (\english{floating point}).
    Bij het vergelijken met de verwachte waarde zal de functie \mintinline{python}{math.isclose()}\footnote{Documentatie is hier te vinden: \url{https://docs.python.org/3/library/math.html\#math.isclose}} uit de standaardbibliotheek van Python gebruikt worden.
    Deze functie controleert of twee zwevendekommagetallen "dicht bij elkaar" liggen.
    De standaardfoutmarges van Python worden gebruikt.
    Een punt voor de toekomst is het configureerbaar maken van deze foutmarges.
    \item[\texttt{applyRounding}] Of zwevendekommagetallen afgrond moeten worden tijdens het vergelijken.
    Indien wel wordt het aantal cijfers genomen van de optie \texttt{roundTo}.
    Na de afronding worden ze ook vergeleken met de functie \mintinline{python}{math.isclose()}.
    Deze afronding is enkel van toepassing op het vergelijken, niet op de uitvoer.
    \item[\texttt{roundTo}] Het aantal cijfers na de komma.
    Enkel nuttig als \texttt{applyRounding} waar is.
\end{description}

Deze configuratieopties worden op het niveau van de testen meegegeven.
Dit laat toe om voor elke test (zelfs binnen eenzelfde testgeval) andere opties mee te geven.
Een nadeel is wel dat dezelfde opties mogelijk veel herhaald moeten worden, bijvoorbeeld als een bepaalde oefening een optie voor elke test wil instellen.
Echter wordt er verwacht dat dit soort zaken opgelost kunnen worden door een \acronym{DSL} of door het testplan te genereren.

Dit is de standaardevaluatievorm in het testplan als niets anders gegeven wordt.
\Cref{lst:testplan-text} toont een fragment uit een testplan: de uitvoerspecificatie van een testgeval waarbij de tekstevaluatie gebruikt wordt.

\begin{listing}
    \inputminted{json}{code/testplan-text.json}
    \caption{Fragment uit een testplan dat de uitvoerspecificatie van de standaarduitvoerstroom voor een testgeval toont, waarbij de tekstevaluatie gebruikt wordt.}
    \label{lst:testplan-text}
\end{listing}

\subsubsection{Bestandsevaluatie}

In deze evaluatievorm worden twee bestanden vergeleken met elkaar.
Hiervoor bevat het testplan enerzijds een pad naar een bestand die met de oefening gegeven wordt met de verwachte inhoud en anderzijds de naam (of pad) van de locatie waar het verwachte bestand zich moet bevinden.
De bestandsevaluatie ondersteunt enkel tekstuele bestanden, geen binaire bestanden.
Het vergelijken van de bestanden gebeurt op één dezer manieren:

\begin{description}
    \item[\texttt{exact}] Beide bestanden moet exact hetzelfde zijn, inclusief regeleindes.
    \item[\texttt{lines}] Elke regel wordt vergeleken met overeenkomstige regel in het andere bestand.
    De evaluatie van de lijnen is exact, maar zonder de regeleindes.
    Dit betekent dat de witruimte bijvoorbeeld ook moet overeenkomen.
    \item[\texttt{values}] Elke regel in het bestand wordt afzonderlijk vergeleken met de tekstevaluatie.
    Indien deze modus gebruikt wordt, kunnen ook alle opties van de tekstevaluatie meegegeven worden.
\end{description}

Een voorbeeld van hoe dit eruitziet is \cref{lst:testplan-file}.
In dit fragment wordt de modus \texttt{values} gebruikt, en worden de opties van de tekstevaluatie ook meegegeven.
Het bestand met de verwachte inhoud heeft als naam \texttt{bestand-uit-de-oefening.txt} gekregen, terwijl de ingediende oplossing een bestand moet schrijven naar \texttt{waar-het-verwachte-bestand-komt.txt}.
Beide paden zijn relatief, maar ten opzichte van andere mappen: het bestand met verwachte inhoud is relatief tegenover de map van de oefening, terwijl het pad waar de ingediende oplossing naar moet schrijven relatief is ten opzichte van de werkmap van de context waarin de oplossing wordt uitgevoerd (zie \cref{lst:mapstructuur} voor een overzicht van de structuur).

\begin{listing}
    \inputminted{json}{code/testplan-file.json}
    \caption{Fragment uit een testplan dat de uitvoerspecificatie van een bestand voor een testgeval toont, waarbij de bestandsevaluatie gebruikt wordt.}
    \label{lst:testplan-file}
\end{listing}

\subsubsection{Waarde-evaluatie}

Voor uitvoerkanalen zoals de returnwaarden moet meer dan alleen tekst met elkaar vergeleken kunnen worden.
Staat er in het testplan welke waarde verwacht wordt (geëncodeerd in het serialisatieformaat), dan kan TESTed dit vergelijken met de eigenlijke waarde die geproduceerd werd door de ingediende oplossing.

Het vergelijken van een waarde bestaat uit twee stappen:
\begin{enumerate}
    \item Het gegevenstype wordt vergeleken, waarbij beide waarden (de verwachte waarde uit het testplan en de geproduceerde waarde uit de ingediende oplossing) hetzelfde type moeten hebben.
    Hierbij wordt rekening gehouden met de vertalingen tussen de verschillende programmeertalen, waarbij twee gevallen onderscheiden kunnen worden:
    \begin{enumerate}
        \item Specifieert het testplan een basistype, dan zullen alle types die tot dit basistype herleid kunnen worden als hetzelfde beschouwd worden.
        Is de verwachte waarde bijvoorbeeld \texttt{sequence}, zullen ook \texttt{array}s uit Java en \texttt{tuple}s uit Python goedgekeurd worden.
        \item Specifieert het testplan een uitgebreid type, dan zal het uitgebreid type gebruikt worden voor talen die dat type ondersteunen, terwijl voor andere talen het basistype gebruikt zal worden.
        Stel dat het testplan bijvoorbeeld een waarde met als gegevenstype \texttt{tuple} heeft.
        In Python en Haskell (twee talen die dat gegevenstype ondersteunen) zullen enkel \texttt{tuple}s goedgekeurd worden.
        Voor andere talen, zoals Java, worden alle gegevenstypes goedgekeurd die herleidbaar zijn tot het basistype.
        Concreet zullen dus \texttt{List}s en \texttt{array}s goedgekeurd worden.
        Merk op dat momenteel bij collecties (\texttt{sequence}s, \texttt{set}s en \texttt{map}s) enkel het type van de collectie gecontroleerd wordt.
    \end{enumerate}
    \item De twee waarden worden vergeleken op inhoud (indien de vergelijking van de gegevenstypes uit de vorige stap positief is).
    Hierbij maakt TESTed gebruik van de ingebouwde vergelijking van Python om twee waarden te evalueren.
    Dit betekent dat de regels voor \english{value comparisons} uit Python\footnote{Zie \url{https://docs.python.org/3/reference/expressions.html?highlight=comparison\#value-comparisons}} gevolgd worden.
    Eén uitzondering is zwevendekommagetallen, waarvoor opnieuw \mintinline{python}{math.isclose()} gebruikt wordt in plaats van \mintinline{python}{==}.
\end{enumerate}

Bij deze evaluatievorm zijn geen configuratieopties.
Een voorbeeld van het gebruik binnen een testplan is \cref{lst:testplan-value}.
Hier wordt als returnwaarde een verzameling met drie elementen (5, 10 en 15) verwacht.

\begin{listing}
    \inputminted{json}{code/testplan-value.json}
    \caption{Fragment uit een testplan dat de uitvoerspecificatie van de returnwaarde voor een testgeval toont, waarbij de waarde-evaluatie gebruikt wordt.}
    \label{lst:testplan-value}
\end{listing}

\subsection{Geprogrammeerde evaluatie}\label{subsec:geprogrammeerde-evaluatie}

Bij oefeningen met niet-deterministische resultaten, zoals de voorbeeldoefening Lotto, kunnen de verwachte waarden niet in het testplan komen.
Ook andere oefeningen waar geen directe vergelijking kan gemaakt worden, zoals het uitlijnen van sequenties (\english{sequence alignment}) uit de bio-informatica, volstaat een vergelijking met een verwachte waarde uit het testplan niet.

Toch is deze evaluatie niet programmeertaalafhankelijk: de logica om een sequentie uit te lijnen is dezelfde ongeacht de programmeertaal waarin dit gebeurt.
Voor dergelijke scenario's is geprogrammeerde evaluatie een oplossing: hierbij wordt code geschreven om de evaluatie te doen, maar deze evaluatiecode staat los van de ingediende oplossing en moet ook niet in dezelfde programmeertaal geschreven zijn.
Binnen TESTed wordt dit mogelijk gemaakt door geproduceerde waarden uit de ingediende oplossing te serialiseren bij het uitvoeren van de testcode, en terug te deserialiseren bij het uitvoeren van de evaluatiecode.

Deze evaluatiecode kan geschreven worden in een programmeertaal naar keuze, al moet de programmeertaal wel ondersteund worden door TESTed.
De implementatie volgt in alle programmeertalen hetzelfde stramien, maar de implementatiedetails kunnen verschillen.
In Python bestaat de evaluatiecode uit een module (een \texttt{.py}-bestand) met een functie die voldoet aan de definitie, zoals gegeven in \cref{lst:evaluation-python-custom}.
TESTed stelt ook een module \texttt{evaluation\_utils} ter beschikking.
De functie van hierboven moet dan één oproep doen naar de functie \texttt{evaluated()}.
Deze module is redelijk eenvoudig, zoals te zien in \cref{lst:evaluation-util-python}

\begin{listing}
    \inputminted{python}{code/custom_signature.py}
    \caption{De definitie van de functie die aanwezig moet zijn in de evaluatiecode voor een geprogrammeerde evaluatie geschreven in Python.}
    \label{lst:evaluation-python-custom}
\end{listing}

\begin{listing}
    \inputminted{python}{../../judge/src/tested/languages/python/templates/evaluation_utils.py}
    \caption{De implementatie van de module \texttt{evaluation\_utils}}
    \label{lst:evaluation-util-python}
\end{listing}

In de Java-implementatie is de situatie gelijkaardig: het gaat om het implementeren van een abstracte klasse.
Deze abstracte klasse biedt ook de functionaliteit aan van de module \texttt{evaluation\_utils} bij Python.
De te implementeren klasse en haar ouderklasse staan in \cref{lst:evaluation-util-java,lst:evaluation-java-custom}.

\begin{listing}
    \inputminted{java}{../../judge/src/tested/languages/java/templates/AbstractCustomEvaluator.java}
    \caption{De implementatie van de klasse \texttt{AbstractCustomEvaluator}.}
    \label{lst:evaluation-java-custom}
\end{listing}

\begin{listing}
    \inputminted{java}{../../judge/src/tested/languages/java/templates/AbstractEvaluator.java}
    \caption{De implementatie van de klasse \texttt{AbstractEvaluator}.}
    \label{lst:evaluation-util-java}
\end{listing}

Een geprogrammeerde evaluatie wordt gebruikt in de voorbeeldoefening Lotto.
Het gebruik in het testplan wordt getoond in \cref{lst:testplan-custom}, waar de evaluatiecode voor de aangepaste evaluatie in Python geschreven is.
Er worden ook argumenten meegegeven aan deze code.
De evaluatiecode zelf is gegeven in \cref{lst:evaluation-lotto}.

\begin{listing}
    \inputminted{java}{code/testplan-custom.json}
    \caption{Fragment uit het testplan van de voorbeeldoefening Lotto, waar een geprogrammeerde evaluatie gebruikt wordt.}
    \label{lst:testplan-custom}
\end{listing}

\begin{listing}
    \inputminted{python}{../../exercise/lotto/evaluation/evaluator.py}
    \caption{De evaluatiecode voor de geprogrammeerde evaluatie van de voorbeeldoefening Lotto.}
    \label{lst:evaluation-lotto}
\end{listing}

\subsection{Programmeertaalspecifieke evaluatie}\label{subsec:programmeertaalspecifieke-evaluatie}

In sommige scenario's moeten programmeertaalspecifieke concepten beoordeeld worden.
Een mogelijkheid is deze oefeningen niet aanbieden in TESTed, maar in de programmeertaalspecifieke judges.
Toch zijn er nog voordelen om ook deze oefeningen in TESTed aan te bieden:
\begin{itemize}
    \item Het bijkomende werk om meer programmeertalen te ondersteunen beperkt zich tot een minimum.
    \item Het werk om een nieuwe programmeertaal toe te voegen aan TESTed is kleiner dan een volledig nieuwe judge te implementeren.
\end{itemize}
Het is desalniettemin het vermelden waard dat het niet zeker is of deze evaluatiemethode (en dit scenario meer algemeen) veel zal voorkomen.
Oefeningen die programmeertaalspecifieke aspecten moeten beoordelen zijn, net door hun programmeertaalspecifieke aard, moeilijker aan te bieden in meerdere programmeertalen.
Een oefening in de programmeertaal C die bijvoorbeeld beoordeelt op juist gebruik van pointers zal weinig nut hebben in Python.

\begin{listing}
    \inputminted{json}{code/testplan-specific.json}
    \caption{Fragment uit een testplan waar een programmeertaalspecifieke evaluatie gebruikt wordt.}
    \label{lst:testplan-specific}
\end{listing}

In gebruik lijkt de programmeertaalspecifieke evaluatie sterk op de geprogrammeerde evaluatie, met dat verschil dat het testplan niet evaluatiecode in één programmeertaal bevat, maar evaluatiecode in alle programmeertalen waarin de oefening aangeboden wordt, zoals geïllustreerd in \cref{lst:testplan-specific}.
Als de programmeertaalspecifieke evaluatie gebruikt wordt en er wordt geen evaluatiecode voor een bepaalde programmeertaal, zal de oefening niet opgelost kunnen worden in die programmeertaal.

Ook de implementatie lijkt op de geprogrammeerde evaluatie, zij het dat de te implementeren functie afwijkt.
In Python wordt dit \cref{lst:evaluation-python-specific}, in Java \cref{lst:evaluation-java-specific}.
Om het resultaat van de evaluatie aan de judge te geven, wordt dezelfde \texttt{evaluated}-functie als bij de aangepaste evaluator gebruikt (zie \cref{lst:evaluation-util-python,lst:evaluation-util-java}).
Het gebruik in het testplan is \cref{lst:testplan-specific}.

\begin{listing}
    \inputminted{python}{code/specific_signature.py}
    \caption{De definitie van de functie die aanwezig moet zijn in de evaluatiecode voor een programmeertaalspecifieke evaluatie geschreven in Python.}
    \label{lst:evaluation-python-specific}
\end{listing}

\begin{listing}
    \inputminted{java}{../../judge/src/tested/languages/java/templates/AbstractSpecificEvaluator.java}
    \caption{De implementatie van de klasse \texttt{AbstractSpecificEvaluator}.}
    \label{lst:evaluation-java-specific}
\end{listing}

\section{Performantie}\label{sec:performantie}

Zoals eerder vermeld (\cref{subsec:testcode-uitvoeren}), wordt de testcode voor elke context afzonderlijk uitgevoerd.
Dat de contexten strikt onafhankelijk van elkaar uitgevoerd worden, werd reeds in het begin als een doel vooropgesteld.
Dit geeft wel enkele uitdagingen op het vlak van performantie.
Het belang van performante judges in Dodona is niet te verwaarlozen, in die zin dat Dodona een interactief platform is, waar studenten verwachten dat de feedback op hun ingediende oplossing onmiddellijk beschikbaar is.
Deze paragraaf beschrijft de evolutie van de implementatie van TESTed vanuit het perspectief van de performantie.

\subsection{Jupyter-kernels}\label{subsec:jupyter-kernels}

Het eerste prototype van TESTed gebruikte Jupyter-kernels voor het uitvoeren van de testcode.
Jupyter-kernels zijn de achterliggende technologie van Jupyter Notebooks \autocite{jupyter2016}.
De werking van een Jupyter-kernel kan als volgt samengevat worden: een Jupyter-kernel is een lokaal proces, dat code kan uitvoeren en de resultaten van die uitvoer teruggeeft.
Zo kan men naar de Python-kernel de expressie \mintinline{python}{5 + 9} sturen, waarop het antwoord \mintinline{python}{14} zal zijn.
Een andere manier om een Jupyter-kernel te bekijken is als een programmeertaalonafhankelijk protocol bovenop een \acronym{REPL} (een \english{read-eval-print loop}).
Deze keuze voor Jupyter-kernels als uitvoering was gebaseerd op volgende argumenten:
\begin{itemize}
    \item Hergebruik van bestaande kernels.
    Hierdoor is het niet nodig om voor elke programmeertaal veel tijd te besteden aan de implementatie of de configuratie: aangezien het protocol voor Jupyter-kernels programmeertaalonafhankelijk is, kunnen alle bestaande kernels gebruikt worden.
    \item De functionaliteit aangeboden door een Jupyter-kernel is de functionaliteit die nodig is voor TESTed: het uitvoeren van fragmenten code en het resultaat van die uitvoering verzamelen.
    \item Eerder werk \autocite{petegem2018}, dat gebruik maakt van Jupyter-kernels voor een gelijkaardig doel, rapporteert geen problemen met het gebruik van Jupyter-kernels.
\end{itemize}

Omdat contexten per definitie onafhankelijk van elkaar zijn, moet de gebruikte Jupyter-kernel gestopt en opnieuw gestart worden tussen elke context.
Dit brengt een onaanvaardbare performantiekost met zich mee, daar de meeste oefeningen niet computationeel intensief zijn.
Het beoordelen van eenvoudige oefeningen, zoals de Lotto-oefening, duurde als snel meerdere minuten.

Enkele ideeën om de performantiekost te verkleinen waren:

\begin{itemize}
    \item Het gebruiken van een \english{pool} kernels (een verzameling kernels die klaar staan voor gebruik).
    De werkwijze is als volgt:
    \begin{enumerate}
        \item Bij de start van het beoordelen van een ingediende oplossing worden meerdere kernels gestart.
        \item Bij elke context wordt een kernel uit de pool gehaald om de testcode uit te voeren.
        \item Na het uitvoeren wordt de kernel op een andere draad (\english{thread}) opnieuw opgestart en terug aan de pool toegevoegd.
    \end{enumerate}
    Het idee achter deze werkwijze is dat door op een andere draad de kernels te herstarten, er altijd een kernel klaarstaat om de testcode uit te voeren, en er dus niet gewacht moet worden op het opnieuw opstarten van die kernels.
    In de praktijk bleek echter dat zelfs met een twintigtal kernels in de pool, het uitvoeren van de testcode van eenvoudige oefeningen dermate snel gaat in vergelijking met het opnieuw opstarten van de kernels, de kernels nooit op tijd herstart zijn.
    \item De kernels niet opnieuw opstarten, maar de interne toestand opnieuw instellen.
    In bepaalde kernels, zoals de Python-kernel, is dit mogelijk.
    De Python-kernel (IPython) heeft bijvoorbeeld een magisch commando \texttt{\%reset}, dat de toestand van de kernel opnieuw instelt.
    Het probleem is er maar weinig kernels zijn die een gelijkaardig commando hebben.
\end{itemize}

Op dit punt is besloten dat de overhead van de Jupyter-kernels niet naar tevredenheid kon opgelost worden.
Bovendien is een bijkomend nadeel van het gebruik van Jupyter-kernels ondervonden: de kernels voor andere programmeertalen dan Python zijn van gevarieerde kwaliteit.
Zo schrijven de Java-kernel en de Julia-kernel bij het opstarten altijd een boodschap naar de standaarduitvoerstroom, wat in bepaalde gevallen problemen gaf bij het verzamelen van de resultaten van de uitvoer van de testcode.

\subsection{Sjablonen}\label{subsec:sjablonen}

We kozen ervoor om verder te gaan met een systeem van sjablonen (zie ook \cref{subsec:testcode-genereren}).
Hierbij wordt de code gegenereerd en vervolgens uitgevoerd via de commandoregel.

Voordelen ten opzichte van de Jupyter-kernels zijn:
\begin{itemize}
    \item Het is sneller om twee onafhankelijke programma's uit te voeren op de commandolijn dan een Jupyter-kernel tweemaal te starten.
    \item Het laat meer vrijheid toe in hoe de resultaten (gedrag en neveneffecten) van het uitvoeren van de testcode verzameld worden.
    \item Het implementeren en configureren van een programmeertaal in TESTed met de basisfunctionaliteit is minder werk dan dan het implementeren van een nieuwe Jupyter-kernel.
    Optionele functionaliteit, zoals linting, kan ervoor zorgen dat meer tijd en werk nodig is bij het toevoegen van een programmeertaal.
    Het systeem met de sjablonen hanteert echter het designprincipe dat optionele functionaliteit geen bijkomend werk mag vragen voor programmeertalen die er geen gebruik van maken.
    \item Er is geen verschil tussen hoe TESTed de ingediende oplossing uitvoert en hoe de student diezelfde oplossing uitvoert.
    Bij de Jupyter-kernels is dat soms wel het geval: bij de R-kernel zitten subtiele verschillen tussen het uitvoeren in de kernel en het uitvoeren op de commandoregel.
    \item Herbruikbaar in die zin dat het genereren van testcode in een programmeertaal op basis van de programmeertaalonafhankelijke specificatie in het testplan sowieso nodig is.
    \item De gegenereerde testcode bestaat uit normale codebestanden (bv.\ \texttt{.py}- of \texttt{.java}-bestanden), wat het toevoegen van een programmeertaal aan TESTed eenvoudiger te debuggen maakt: alle gegenereerde testcode is in een bestand beschikbaar voor inspectie.
\end{itemize}

Elke medaille heeft ook een keerzijde:
\begin{itemize}
    \item Het gebruik van sjablonen zorgt ervoor dat het uitvoeren van de testcode minder dynamisch is.
    Daar de testcode eerst gegenereerd moet worden, moet vóór het uitvoeren bepaald worden welke testcode zal uitgevoerd worden.
    \item Er moet meer zelf geïmplementeerd worden: het ecosysteem van Jupyter is groot.
    Er bestaan meer dan honderden kernels\footnote{Zie deze pagina voor een actuele lijst van kernels: \url{https://github.com/jupyter/jupyter/wiki/Jupyter-kernels}.}, goed voor ondersteuning voor meer dan tachtig programmeertalen.
    Niet elke kernel is onmiddellijk bruikbaar, maar er kan verder gewerkt worden op wat bestaat.
    \item Het uitvoeren op de commandoregel is bij veel programmeertalen trager indien er geen reset moet gebeuren tussen de verschillende uitvoeringen.
    Bij het uitvoeren op de commandoregel is er bij veel programmeertalen een performantiekost bij het opstarten van de interpreter (zoals Python) of virtuele machine (zoals Java).
    Bij een kernel is de opstartkost weliswaar groter, maar deze wordt maar één keer opgestart als er geen reset nodig is.
\end{itemize}

Bij een eerste iteratie van het systeem met sjablonen bestond enkel de \textbf{contextcompilatie} (\cref{subsec:testcode-genereren}).
Dit heeft een grote performantiekost bij alle programmeertalen, maar in het bijzonder bij talen zoals Java.
Daar moest voor elke context eerst een compilatie plaatsvinden, waarna de gecompileerde testcode werd uitgevoerd.
Bij een testplan met bijvoorbeeld 50 contexten vertaalt dit zich in 50 keer de \texttt{javac}-compiler uitvoeren en ook 50 keer het uitvoeren van de code zelf met \texttt{java}.

In een tweede iteratie werd een \textbf{partiële compilatie} geïmplementeerd.
Het idee hier is dat er veel testcode is die hetzelfde blijft voor elke context (denk aan de ingediende oplossing en hulpbestanden van TESTed).
In de partiële compilatie wordt de gemeenschappelijke testcode eerst gecompileerd (bij programmeertalen die dat ondersteunen).
Bij het beoordelen van een context wordt dan enkel de testcode specifiek voor de te beoordelen context gecompileerd.

In een derde iteratie werd de partiële compilatie uitgebreid naar \textbf{batchcompilatie}.
Hoewel dit een grote winst voor de uitvoeringstijd betekende, heeft deze modus wel een ander groot nadeel (zoals reeds vermeld in \cref{subsec:testcode-genereren}): bij statische talen zorgen compilatiefouten in één context ervoor dat geen enkele context beoordeeld kan worden, daar de compilatiefout ervoor zorgt dat er geen code gegenereerd wordt.
Dit wordt best geïllustreerd met het reeds vermelde scenario van een oefening waarbij de student twee functies dient te implementeren.
In batchcompilatie in bijvoorbeeld Java kan de student niet de oplossing voor de eerste functie indienen en laten beoordelen: omdat de tweede functie ontbreekt zullen compilatiefouten optreden.

De laatste en huidige iteratie bevat de mogelijkheid om contextcompilatie en batchcompilatie te combineren, waarbij TESTed terugvalt op de contextcompilatie indien er iets misgaat bij de batchcompilatie.
Dit terugvalmechanisme is geen geavanceerd systeem: faalt de compilatie, wordt onmiddellijk elke context afzonderlijk gecompileerd.
Een mogelijke optimalisatie bestaat er uit een tussenoplossing toe te passen, door bijvoorbeeld compilatie op het niveau van tabbladen proberen (dit wordt verder besproken in \cref{subsec:future-performance}).

\subsection{Parallelle uitvoering van contexten}\label{subsec:parallelle-uitvoering-van-contexten}

Een ander aspect is de ondersteuning voor parallelle uitvoering van de contexten.
Zoals al enkele malen aangehaald zijn de contexten van het testplan volledig onafhankelijk van elkaar, waardoor ze zich ertoe lenen om parallel uitgevoerd te worden.

Een eerste opmerking hierbij is dat de tijdswinst op Dodona kleiner waarschijnlijk zal zijn dan de tijdswinst geobserveerd in de tijdsmetingen (\cref{tab:meting}).
De reden hiervoor is dat Dodona een beperkt aantal \english{workers} heeft die beoordelingen uitvoeren.
Doordat er meerdere gebruikers tegelijk oplossingen indienen, zal er niet altijd ruimte zijn voor parallellisatie binnen de judge.

Een tweede opmerking is dat er verschillende beperkingen of opmerkingen zijn in de implementatie:

\begin{itemize}
    \item De parallelle uitvoering heeft enkel betrekking op het uitvoeren van de contexten, niet op de beoordeling ervan.
    De beoordeling gebeurt nog steeds sequentieel.
    \item De parallellisatie gebeurt enkel binnen tabbladen.
    De tabbladen zelf worden sequentieel uitgevoerd.
    Dit betekent dat de contexten van tabblad 1 parallel worden uitgevoerd, en pas aan tabblad 2 begonnen wordt als alle contexten van tabblad 1 klaar zijn.
    \item De volgorde waarin het resultaat van de beoordeling van een context naar Dodona gestuurd wordt blijft dezelfde volgorde als in het testplan.
    Dit impliceert dat het mogelijk is dat, bijvoorbeeld bij het overschrijden van de tijdslimiet, de uitvoer van uitgevoerde contexten niet getoond wordt.
    Stel dat contexten 1, 2, 4 en 5 klaar zijn wanneer de tijdslimiet overschreden wordt.
    Omdat TESTed aan het wachten was op context 3, zullen contexten 4 en 5 ook als niet uitgevoerd beschouwd worden.
    \item De tijdslimieten zullen minder nauwkeurig zijn.
    Door de parallelle uitvoering zal er meer variatie zitten op de totale tijd die nodig is voor het beoordelen van een oefening.
    Dit heeft geen effect op de tijdslimieten bij hun functie voor het detecteren van programmeerfouten, zoals oneindige lussen.
    Waar er wel moet opgelet worden is indien de tijdslimieten gebruikt worden om te controleren dat een oplossing efficiënt is.
    Door de parallelle uitvoering is het mogelijk dat een oplossing die normaal nooit onder de tijdslimiet zou zitten, nu wel aanvaard wordt.
\end{itemize}

\subsection{Snellere geprogrammeerde evaluatie}\label{subsec:snellere-geprogrammeerde-evaluatie}

Een laatste performantieverbetering werd behaald op het vlak van geprogrammeerde evaluatie (zie \cref{subsec:geprogrammeerde-evaluatie-is-traag} voor verdere ideeën voor verbeteringen).
Zoals beschreven in \cref{subsec:geprogrammeerde-evaluatie}, wordt normaliter voor elke geprogrammeerde evaluatie de evaluatiecode opnieuw gecompileerd en uitgevoerd in een nieuw subproces.
Voor evaluatiecode die geschreven is in Python worden deze stappen overgeslagen.
Er is speciale ondersteuning in TESTed ingebouwd om geprogrammeerde evaluaties waarvan de evaluatiecode in Python geschreven is, rechtstreeks in TESTed zelf uit te voeren.

\subsection{Tijdsmetingen}\label{subsec:tijdsmetingen}

\begin{table}
    \centering
    \begin{tabular}{ll|r|r|r|r|}
        \cline{3-6}
        & & \multicolumn{2}{c|}{Python (s)} & \multicolumn{2}{c|}{Java (s)}  \\
        \hline
        \multicolumn{1}{|l|}{Oefening}               & Compilatiemodus  & 1 thread & 4 threads & 1 thread & 4 threads \\
        \hline
        \multicolumn{1}{|l|}{\multirow{3}{*}{Lotto}} & Context          & 13       & 9         & 51       & 38        \\
        \multicolumn{1}{|l|}{}                       & Partieel         & 9        & 7         & 46       & 25        \\
        \multicolumn{1}{|l|}{}                       & Batch            & 9        & 6         & 12       & 10        \\
        \multicolumn{1}{|l|}{}                       & Batch+Eval       & 5        & 3         & 9        & 6         \\
        \hline
        \multicolumn{1}{|l|}{\multirow{3}{*}{Echo}}  & Context           & 6        & 3         & 48       & 28        \\
        \multicolumn{1}{|l|}{}                       & Partieel          & 5        & 3         & 46       & 26        \\
        \multicolumn{1}{|l|}{}                       & Batch             & 5        & 3         & 8        & 5         \\
        \hline
    \end{tabular}
    \caption{Tijdsmetingen voor de oefeningen Lotto en Echo, voor de programmeertalen Python en Java, in contextcompilatie, partiële compilatie en batchcompilatie.
        Bij Lotto is deze laatste uitgevoerd met en zonder optimalisatie voor geprogrammeerde evaluaties.}
    \label{tab:meting}
\end{table}

\Cref{tab:meting} toont enkele tijdsmetingen van de verschillende implementaties van het systeem met sjablonen, voor de programmeertalen Python en Java.
In deze tabel is de tijd gemeten in seconden om een (juiste) ingediende oplossing voor twee oefeningen te beoordelen:

\begin{itemize}
    \item De Lotto-oefening, zoals beschreven in \cref{sec:probleemstelling}.
    Deze oefening heeft een geprogrammeerde evaluatie en bestaat uit 50 contexten, die elks één testgeval bevatten.
    \item Een eenvoudigere oefening: Echo.
    Bij deze oefening bestaat de opgave er uit de invoer van de standaardinvoerstroom te lezen en te printen naar de standaarduitvoerstroom.
    Deze oefening bevat enkel een generieke evaluatie en bestaat ook uit 50 contexten, met telkens één testgeval per context.
\end{itemize}

Elke beoordeling is uitgevoerd met contextcompilatie, partiële compilatie en batchcompilatie.
De Lotto-oefening is ook gemeten met de optimalisatie voor geprogrammeerde evaluatie (daar de evaluatiecode voor deze oefening geschreven is in Python).
Dit is aangeduid met de naam "Batch+eval" in de kolom "Compilatiemodus".
Deze variant is niet uitgevoerd op de Echo-oefening, daar die oefening geen geprogrammeerde evaluatie heeft.
Elke tijdsmeting is ook uitgevoerd met en zonder parallelle uitvoering van de contexten.
Deze tijdsmetingen zijn uitgevoerd op een standaardcomputer (Windows 10, Intel \version{i7-8550U}, Python \version{3.8.1}, 64-bit), niet op de Dodona-server en niet in een Docker-container.
Die laatste twee factoren kunnen ervoor zorgen dat uitvoeringstijden op Dodona sterk verschillen van deze metingen.

Zoals verwacht levert batchcompilatie het meeste tijdswinst op bij Java: daar heeft de compilatiestap ook een veel groter aandeel in de uitvoeringstijd.
Bij Python is de compilatie veel sneller en ook veel minder belangrijk (en zelfs optioneel, de stap wordt enkel uitgevoerd om bepaalde syntaxisfouten vroeg op te vangen).

\section{Bijkomende taken}\label{sec:andere-taken}

Sommige judges doen meer dan enkel het oordelen over de juistheid van de ingediende oplossing.
Zo heeft onder andere de Python-judge ondersteuning voor \term{linting}.
Dit betekent dat de code van de ingediende oplossing geanalyseerd (maar niet uitgevoerd) wordt om zo allerlei mogelijke problemen op te sporen, zoals stijlfouten, mogelijke bugs, verdachte constructies, enz.

TESTed voorziet hier ondersteuning voor middels een optionele functie in de configuratie van een programmeertaal (zie \cref{ch:nieuwe-taal} voor de details over het configureren van een taal).
Deze functie krijgt als argumenten onder andere de configuratie, het testplan en het pad naar de code van de ingediende oplossing mee.
Als resultaat geeft de functie een lijst van berichten en annotaties die naar Dodona gestuurd worden.
Ter illustratie dat deze aanpak werkt, is linting geïmplementeerd voor Python.
Ook geeft dit veel vrijheid: de implementatie in Python gebruikt een bestand voor het configureren van de linter, en het is mogelijk om de naam van een eigen configuratiebestand (uit de oefeningenrepository) mee te geven via de programmeertaalspecifieke configuratie van TESTed.
Tot slot nog opmerken dat het formaat van de annotaties wordt voorgeschreven door Dodona, maar dit is voldoende generiek voor praktisch alle scenario's.
Het formaat ondersteunt onder andere:

\begin{itemize}
    \item Het aangeven van een gedetailleerde plaats in de ingediende oplossing, zoals de regel en kolom.
    Het aanduiden van meerdere opeenvolgende regels of kolommen is ook mogelijk.
    \item De boodschap is vrij tekstveld.
    \item De ernst van de boodschap, met ondersteuning voor een veelvoorkomende indeling: informatiebericht, waarschuwing en foutbericht.
\end{itemize}

Een mogelijke uitbreiding is bijvoorbeeld het toevoegen van identificatienummers aan de berichten.
Bij veel linters heeft elke boodschap een unieke code.
Aan de hand van die code zou Dodona de uitleg over de boodschap kunnen tonen (maar dit vereist wel dat Dodona de uitleg voor specifieke linters heeft).
Een alternatief zou zijn dat TESTed deze zelf toevoegt.
De gemakkelijkste oplossing is echter de annotatie uitbreiding met ondersteuning voor een \acronym{URL}: bij veel linters kan met de identificatiecode van het bericht een \acronym{URL} geconstrueerd worden waarop meer informatie staat.
Deze \acronym{URL} zou dan aan de studenten kunnen getoond worden.

\section{Robuustheid}\label{sec:robuustheid}

Een belangrijk aspect bij \term{educational software testing} is de feedback als het verkeerd loopt.
De feedback bij een verkeerde oplossing is in veel gevallen zelfs belangrijker dan de feedback bij een juiste oplossing: het is namelijk de bedoeling dat als studenten een verkeerde oplossing indienen, dat de feedback ze terug op weg kan helpen om hun oplossing te verbeteren.
De feedback die Dodona toont is afkomstig van de judges: het is de taak van TESTed om kwalitatief hoogstaande feedback te voorzien.
Met kwalitatief hoogstaand wordt bedoeld dat de feedback nuttige informatie bevat, maar ook geen verkeerde of misleidende informatie bevat.

\subsection{In de praktijk}\label{subsec:in-de-praktijk}

In het vak \emph{Scriptingtalen} van de bachelor informatica krijgen studenten elke week een reeks oefeningen om op te lossen (op Dodona).
Een van die oefeningen is de ISBN-oefening.
Om een idee te krijgen van hoe TESTed werkt in de praktijk is besloten om tijdens een van de weken de ISBN-oefening aan te vullen door een oefening met dezelfde opgave, maar die beoordeeld wordt door TESTed in plaats van de Python-judge.

De algemene bevindingen die uit de praktijktest naar boven kwamen waren:

\begin{itemize}
    \item TESTed heeft geen ondersteuning voor de Python Tutor.
    Dit is al vermeld, en wordt verder besproken in \cref{subsec:kleinere-functies}.
    \item De evaluatie met TESTed duurt ongeveer dubbel zo lang als met de Python-judge.
    Aan de ene kant zou de optimalisatie van de geprogrammeerde evaluatie (zie \cref{sec:performantie}) dit moeten verbeteren (maar wel in dit specifieke geval, moest de evaluatiecode in een andere programmeertaal dan Python geschreven zijn, zou het geen verschil maken).
    Aan de andere kant ligt een tragere uitvoer in vergelijking met de Python-judge binnen de verwachtingen: TESTed voert elke context uit in een afzonderlijk subproces.
    De Python-judge doet dit niet.
    Qua uitvoer worden wel contexten gebruikt, maar de interne werking van de Python-judge kan het best vergeleken worden met een oefeningen waar alle testen in dezelfde context opgenomen zijn.
    \item TESTed heeft geen verwerking van stacktraces.
    Deze functionaliteit is opgenomen als mogelijke uitbreiding als onderdeel van \cref{subsec:kleinere-functies}.
    \item TESTed heeft geen linting.
    Naar aanleiding van de praktijktest is hier ondersteuning voor toegevoegd, zie \cref{sec:andere-taken}.
    \item TESTed heeft geen ondersteuning voor vertalingen op het vlak van natuurlijke talen.
    Deze functie wordt besproken als uitbreiding in \cref{subsec:ondersteuning-voor-natuurlijke-talen}.
\end{itemize}

Verder heeft professor Dawyndt een heleboel fouten en scenario's uitgeprobeerd.
Deze worden hieronder besproken.

\subsection{Soorten fouten}\label{subsec:soorten-fouten}

TESTed moet robuust zijn tegen allerlei vormen van fouten in de ingediende oplossingen.
Hieronder volgt een lijst van (categorieën) van fouten waarvoor TESTed nuttige feedback geeft.
TESTed is zo opgebouwd dat er altijd iets van feedback komt, ook in onvoorziene omstandigheden, maar voor de soorten fouten op onderstaande lijst is expliciet gecontroleerd wat de kwaliteit van de feedback is.
Waar nuttig zijn deze gevallen omgezet naar een unit test voor TESTed.

\begin{description}
    \item[Compilatiefouten] De uitvoer van de compiler wordt altijd getoond aan de studenten, dus op dat vlak wordt juist afgehandeld.
    Er bestaat ook de mogelijkheid om de compilatie-uitvoer te verwerken en bijvoorbeeld om te zetten naar annotaties, die dan in de code worden getoond.
    Hierbij wordt vooral gedacht aan foutboodschappen als "syntaxisfout op regel 5, kolom 2".
    De stacktraces bij deze uitvoer bevatten wel referenties naar code die TESTed gegenereerd heeft.
    Deze verwijzingen naar interne code wegfilteren is opgenomen als uitbreiding als onderdeel van \cref{subsec:kleinere-functies}.
    Bij een batchcompilatie wordt de uitvoer in een nieuw tabblad getoond, terwijl bij contextcompilatie de uitvoer bij de relevant context staat.
    \item[Uitvoeringsfouten] Hier gaat het om crashes tijdens de uitvoering, zoals delen door nul.
    Ook hier wordt nog geen verdere verwerking van de foutboodschap gedaan (ook onderdeel van \cref{subsec:kleinere-functies}).
    \item[Tijdslimieten] TESTed heeft ondersteuning voor tijdslimieten in de judge.
    Dit laat toe om meer uitvoer te tonen dan als de tijdslimiet aan Dodona wordt overgelaten.
    Momenteel werkt de implementatie ervan als volgt: TESTed houdt bij hoe lang de beoordeling al duurt, en gebruikt de resterende tijd als een limiet op de uitvoering van een context.
    De eerste context heeft dan als limiet de volledige toegestane tijd (minus een percentage dat voorbehouden wordt voor TESTed zelf).
    De laatste context zal de kleinste limiet krijgen (de tijd die nog overschiet).
    Scenario's waar dit voorkomt zijn bijvoorbeeld oplossingen die te traag werken, maar ook oplossingen met bijvoorbeeld oneindige lussen.
    Als uitbreiding (opgenomen in \cref{subsec:kleinere-functies}) wordt beschreven dat er ook een tijdslimiet per context mogelijk zou kunnen zijn (zodat de eerste context niet de volledige beschikbare tijd krijgt, maar slechts een deel ervan).
    De hoofdreden om zelf de tijdslimiet te implementeren in TESTed is om zo de overige, niet-uitgevoerde testen ook te kunnen tonen.
    Momenteel is dat met een boodschap die uitlegt dat ze niet zijn uitgevoerd, maar op termijn is het de bedoeling in Dodona een bijkomende status toe te voegen.
    \item[Te grote uitvoer] Bij te grote uitvoer zal TESTed deze limiteren.
    Hierbij gaat het om scenario's zoals een oneindige lus die blijft schrijven naar stdout.
    TESTed beperkt tekstuele uitvoer tot tienduizend tekens.
    Dit is niet van toepassing op andere vormen van uitvoer: het inkorten van grote verzamelingen (bijvoorbeeld een lijst van duizend elementen) is opgenomen als uitbreiding in \cref{subsec:kleinere-functies}.
    \item[Te veel uitvoer] Hierbij wordt gedacht aan zaken zoals uitvoer op stdout of stderr terwijl die niet verwacht wordt.
    Standaard wordt een oefening als fout beschouwd indien er te veel uitvoer is, maar de auteur van de oefening kan kiezen om het teveel aan uitvoer te negeren (bijvoorbeeld als de studenten debugberichten schrijven naar stdout is dit niet in elke oefening een probleem).
    \item[Vroegtijdig stoppen van uitvoering] Hier gaat het om code van de studenten die bijvoorbeeld \mintinline{python}{exit(-5)} bevat.
    Dit zal standaard fout gerekend worden door TESTed.
    De exitcode is echter een uitvoerkanaal zoals elk ander, waar de verwachte waarde ingesteld kan worden.
    Is het dus de bedoeling om exitcode \texttt{-5} te krijgen, zal de oefening juist gerekend worden.

\end{description}

    %! Suppress = EscapeHashOutsideCommand
%! Suppress = TooLargeSection
\chapter{Configuratie van een oefening}\label{ch:nieuwe-oefening}

\lettrine{I}{n dit hoofdstuk} behandelen we het configureren van oefeningen voor \tested{}.
We bespreken verschillende eenvoudige oefeningen, met als doel dat deze oefeningen als voorbeeld kunnen dienen bij het samenstellen van complexere oefeningen of oefeningen die meerdere functionaliteiten tegelijk gebruiken.
We bespreken oefeningen:

\begin{itemize}
    \item die invoer (\texttt{stdin}) lezen en uitvoer (\texttt{stdout}) produceren (\english{I/O}-oefeningen),
    \item waarbij functies beoordeeld worden door \tested{},
    \item waarbij een programmeertaalspecifieke evaluator gebruikt wordt,
    \item waarbij een geprogrammeerde evaluator gebruikt wordt,
    \item waarbij commandoargumenten gelezen worden,
    \item waarbij statements (assignments) gebruikt worden en
    \item waarbij objectgerichte zaken gebruikt worden.
\end{itemize}

Bij elke oefening bespreken we de opgave, hoe we willen dat de beoordeling van een oplossing er zal uitzien en hoe het testplan er uiteindelijk uitziet.

\section{Oefeningen in het Dodona-platform}\label{sec:oefeningen-in-het-dodona-platform}

\tested{} zelf legt geen structuur of formaat vast voor de oefeningen, buiten het formaat van het testplan.
De locatie van de relevante bestanden worden meegegeven bij het uitvoeren van \tested{}.
\tested{} kan daardoor ook los van Dodona gebruikt worden.

De configuratie in de manuele uitvoeringsmodus van \tested{} gaat er wel vanuit dat de mappenstructuur van de oefening de structuur van Dodona volgt.
Voor de volledigheid volgt hieronder een mappenstructuur met de belangrijkste elementen van de oefening:

\inputminted{text}{code/dirs-exercise.txt}

Voor meer details en bijkomende informatie (want het voorbeeld hierboven is slechts een basis), raden we aan om de relevante handleiding uit de Dodona-documentatie te lezen\footnote{Beschikbaar op \url{https://dodona-edu.github.io/en/references/exercise-directory-structure/}}.

\section{Lotto}\label{sec:oefening-lotto}

Als eerste oefening beschouwen we de voorbeeldoefening Lotto uit \cref{sec:probleemstelling}, die we uitgebreid zullen bespreken.
De opgave is al opgenomen in \cref{sec:probleemstelling}, terwijl de voorbeeldoplossingen in \cref{sec:probleemstelling} en \cref{lst:python-solution,lst:java-solution} staan.

\subsection{Structuur van de beoordeling}\label{subsec:oefening-lotto-structuur}

Na het lezen van de opgave is het duidelijk dat de oefening bestaat uit het implementeren van een functie.
In abstracto zullen we, om te beoordelen of deze functie correct geïmplementeerd is, de functie een aantal keren oproepen met verschillende argumenten en dan het resultaat vergelijken met een verwachte waarde.

Zoals vermeld in \cref{subsec:het-testplan} bestaat het testplan uit een hiërarchie van elementen, beginnend met een aantal tabbladen, die elk een aantal contexten hebben, die op hun beurt bestaan uit testgevallen, die tot slot bestaan uit testen.
In dit geval lijkt het logisch om één tabblad te gebruiken: per slot van rekening beoordelen we één functie.

Alle functieoproepen zijn onafhankelijk van elkaar, wat suggereert dat elke functieoproep in een aparte context dient te gebeuren.
Een context bestaat uit een optioneel testgeval voor de \texttt{main}-functie en een reeks normale testgevallen.
Bij de Lotto-oefening hebben we geen \texttt{main}-functie, dus zullen we enkel normale testgevallen hebben.
Er zal één testgeval per context zijn, want in elke context willen we de functie éénmaal oproepen.

Vertaald naar pseudocode willen we dus dat onze beoordeling de volgende vorm aanneemt:

\inputminted{python}{code/lotto-eval.py}

Hierbij is het geheel van \texttt{assert}s dus ons tabblad, terwijl we elke \texttt{assert} in een aparte context en dus ook apart testgeval onderbrengen.

\subsection{Evaluatie}\label{subsec:oefening-lotto-evaluatie}

Iets dat we in de vorige paragraaf genegeerd hebben, is hoe we het vergelijken met een verwachte waarde exact gaan doen.
Lottonummers worden namelijk willekeurig getrokken: de pseudocode van hierboven zou dus slechts heel zelden tot een juiste beoordeling leiden.
We lossen dit op door een geprogrammeerde evaluatie te gebruiken.
Dit is een functie die \tested{} oproept om de door de ingediende oplossing geproduceerde waarde te vergelijken met een verwachte waarde.
Hierbij is het mogelijk om in het testplan argumenten mee te geven aan deze evaluatiefunctie, iets dat we hier ook doen.
De argumenten die we meegeven aan de \texttt{loterij}-functie geven we ook mee aan de evaluatiefunctie.
Conceptueel kunnen we dat ook vertalen naar dezelfde pseudocode:

\inputminted{python}{code/lotto-eval-programmed.py}

We zullen de geprogrammeerde evaluatie in Python doen: dit is de aanbevolen programmeertaal voor geprogrammeerde evaluaties in \tested{}, met als eenvoudige reden dat ze het snelst is.
Hieronder is een fragment van de evaluatiecode: dit is de functie die door \tested{} zal opgeroepen worden.

\inputminted[firstline=42,lastline=48]{python}{sources/lotto-evaluator.py}

Wat doet deze functie nu juist?

\begin{enumerate}
    \item Een evaluatiefunctie van een geprogrammeerde evaluatie kan ook argumenten meekrijgen.
    In ons geval geven we de argumenten van de \texttt{loterij}-functie mee als argument aan de evaluatiecode.
    \item De functie \texttt{valid\_lottery\_numbers} wordt opgeroepen.
    We hebben deze functie niet opgenomen in het codefragment hierboven omdat het een lange functie is, maar deze functie controleert in feite of de geproduceerde waarde voldoet aan de vereisten (klopt het aantal getallen, is de lijst gesorteerd, enzovoort).
    \item Indien de geproduceerde waarde geldig is, dan gebruiken we die ook als verwachte waarde.
    Dit voorkomt dat de oplossing juist is, maar Dodona toch een verschil toont tussen de geproduceerde en verwachte waarde.
    \item Tot slot construeren we een \texttt{EvaluationResult} als returnwaarde.
    Dit object verwacht vier waarden: het resultaat van de evaluatie (juist of fout), optioneel de verwachte waarde indien deze overschreven moet worden, optioneel de geproduceerde waarde indien deze overschreven moet worden en een optionele lijst van feedbackberichten (meer details in \cref{subsec:geprogrammeerde-evaluatie}).
\end{enumerate}

\subsection{Het testplan}\label{subsec:oefening-lotto-testplan}

Nu we weten welke structuur we willen en hoe we gaan beoordelen kunnen, we een testplan opstellen in \acronym{JSON}.
Hieronder volgt ter illustratie één context uit het testplan.
In werkelijkheid (en ook bij deze oefening) wordt het testplan niet met de hand geschreven;
het wordt gegenereerd door een Python-script.
Dit script bevindt zich in de \texttt{preparation}-map van de oefening.

We merken op dat we wel een verwachte waarde opnemen in het testplan, ook al gaat het om willekeurige returnwaarden.
Bij deze oefening wordt deze waarde niet gebruikt bij de geprogrammeerde evaluatie.
Toch is het nuttig ze op te nemen, omdat deze verwachte waarde in de feedback getoond zal worden als de geproduceerde waarde verkeerd is of als de tijdslimiet bijvoorbeeld overschreden wordt.
Het is mogelijk om in de geprogrammeerde evaluatie geavanceerde zaken te doen, zoals indien de geproduceerde waarde juist is, maar niet gesorteerd, de geproduceerde waarde te sorteren en als verwachte waarde te gebruiken.

\inputminted[firstline=6,lastline=50,gobble=8]{json}{sources/lotto-plan.tson}

\section{Echo}\label{sec:oefening-echo}

Een volgende oefening die we bekijken, is een eenvoudige \term{invoer-uitvoer}-oefening.
Deze oefening bestaat uit een invoer lezen van \texttt{stdin} en deze invoer schrijven naar \texttt{stdout}.
Bij deze oefeningen zijn voor een aantal programmeertalen de volledige testcode die door \tested{} gegenereerd wordt en de voorbeeldoplossingen opgenomen in \cref{ch:echo-oefening}.

\subsection{Opgave}\label{subsec:oefening-echo-opgave}

De volledige opgave voor deze oefening volgt hieronder:

\begin{quote}
    \markdownInput[renderers = {
    headingOne = {\chapter*{#1}},
    headingTwo = {\section*{#1}},
    headingThree = {\subsection*{#1}},
    }]{sources/echo/description.md}
\end{quote}

\subsection{Structuur van de beoordeling en evaluatie}\label{subsec:oefening-echo-structuur}

Voor de structuur van de beoordeling geldt grotendeels hetzelfde als bij \cref{subsec:oefening-lotto-structuur}: we hebben een tabblad, met daarin een aantal contexten.
In elke context gebruiken we een andere waarde als \texttt{stdin}.
In tegenstelling tot de Lotto-oefening hebben we nu wel een \texttt{main}-functie.
De context zal dus bestaan uit het testgeval voor de \texttt{main}-functie en geen normale testgevallen hebben.
Het testgeval voor de \texttt{main}-functie wordt in het testplan apart aangeduid met \texttt{context\_testcase}.

Qua evaluatie is deze oefening eenvoudig: we kunnen de ingebouwde evaluatie van \tested{} gebruiken.
De geproduceerde tekstuele waarde moet vergeleken worden met de verwachte waarde (en moet exact overeenkomen).

\subsection{Testplan}\label{subsec:oefening-echo-testplan}

Als voorbeeldtestplan nemen we een testplan met twee contexten.
Bij het gebruik van deze oefening zal het testplan vijftig contexten bevatten.
Het wordt ook niet met de hand geschreven: een Python-script genereert het.

\inputminted{json}{sources/echo/two.tson}

\section{Echofunctie}\label{sec:oefening-echofunctie}

Een variant van de vorige oefening is de oefening \emph{Echofunctie}, waarbij een \texttt{echo}-functie geïmplementeerd dient te worden.
Ook bij deze oefening zijn voor een aantal programmeertalen de gegenereerde testcode en voorbeeldoplossingen opgenomen in \cref{ch:echo-function-oefening}.

\subsection{Opgave}\label{subsec:oefening-echofunctie-opgave}

De volledige opgave luidt als volgt:

\begin{quote}
    \markdownInput[renderers = {
    headingOne = {\chapter*{#1}},
    headingTwo = {\section*{#1}},
    headingThree = {\subsection*{#1}},
    }]{sources/echo-function/description.md}
\end{quote}

Het is nuttig om stil te staan bij waarom de opgave vermeldt dat de invoer altijd een \texttt{string} zal zijn.
Dit is om de oefening in zoveel mogelijk programmeertalen te kunnen aanbieden.
In bepaalde programmeertalen, zoals C, is het moeilijk om een functie te schrijven die een argument van een willekeurig type aanvaardt (we noemen dit heterogene argumenten binnen \tested{}).
Om die reden verhindert \tested{} dat oefeningen waar dit vereist is opgelost kunnen worden in die programmeertalen.
Toch is het mogelijk om oefeningen op te stellen waar heterogene argumenten gebruikt worden.
Deze oefeningen zullen dan in minder programmeertalen beoordeeld kunnen worden.
Stel dat we in het testplan bijvoorbeeld de \texttt{echo}-functie oproepen met een getal in plaats van een \texttt{string}.
\tested{} zal dan automatisch detecteren dat deze oefening enkel opgelost kan worden in programmeertalen die heterogene functieargumenten ondersteunen.
Als we dan toch zouden proberen om een oplossing geschreven in C te laten beoordelen, zal \tested{} een foutmelding tonen (zie \cref{subsec:vereiste-functies} voor meer details over dit mechanisme).

\subsection{Structuur van de beoordeling en evaluatie}\label{subsec:oefening-echofunctie-structuur}

De structuur is volledig analoog aan de \emph{Echo}-oefening.
We zullen een reeks contexten hebben, waarbij we in elke context de te implementeren functie oproepen met andere invoer en het resultaat controleren.
Ook de evaluatie is analoog: we gebruiken de ingebouwde evaluatie van \tested{}.
Het verschil is in welke kanalen de invoer en uitvoer zich bevindt: bij deze oefening gebruiken we functieargumenten in plaats van \texttt{stdin} en het resultaat is een returnwaarde in plaats van \texttt{stdout}.

\subsection{Testplan}\label{subsec:oefening-echofunctie-testplan}

Ter illustratie tonen we hier een testplan met één context, die twee testgevallen bevat.
Dit komt met opzet niet overeen met wat we hierboven bij de structuur van de beoordeling besproken hebben.
De reden hiervoor is dat we ook een illustratie van meerdere testgevallen willen in \cref{ch:echo-function-oefening}.
In het testplan dat gebruikt zou worden bij het beoordelen van oplossingen van studenten zal elk testgeval wel in een eigen context geplaatst worden (en zullen er opnieuw meer contexten zijn).

\inputminted{json}{sources/echo-function/one-testcase.tson}

\section{ZeroDivisionError}\label{sec:oefening-zero}

Een interessante oefening is de oefening \emph{ZeroDivisionError} uit het boek \emph{De Programmeursleerling} van Pieter Spronck \autocite{programmeursleerling}.\footnote{Een interactieve versie is beschikbaar op Dodona: \url{https://dodona.ugent.be/nl/courses/293/series/2535/activities/270198713/}}
In deze oefening moet een programma geschreven worden dat bij het uitvoeren een exception gooit.
In Python gaat het om een \texttt{ZeroDivisionError}.
We nemen de opgave hier niet op, omdat de opgave lang is en weinig bijdraagt aan het doel dat we hier hebben, het uitleggen hoe oefeningen voor \tested{} geschreven moeten worden.

\subsection{Structuur van de beoordeling}\label{subsec:oefening-zero-structuur}

Daar het programma een exception moet gooien als het uitgevoerd wordt, lijkt het aangewezen dat we met een \texttt{main}-functie zitten.
Ook uniek aan deze oefening is dat we de oplossing één keer moeten uitvoeren, dus zullen we ook één context hebben.
De structuur zal dus een tabblad met een context met een testgeval voor de \texttt{main}-functie zijn.

\subsection{Evaluatie}\label{subsec:oefening-zero-evaluatie}

Het feit dat specifiek een \texttt{ZeroDivisionError} moet gegooid worden in Python, zorgt ervoor dat we hier een programmeertaalspecifieke evaluatie zullen moeten gebruiken.
In Java gaat het bijvoorbeeld om een \texttt{ArithmeticException}, terwijl delen door nul is Haskell zal zorgen voor een \texttt{DivideByZero}.

We bekijken hier eens niet de evaluatiecode in Python, maar die in Java.
De evaluatiecode is ook beschikbaar in Haskell.

\inputminted{java}{sources/division-evaluator.java}

De evaluatiecode is redelijk rechtdoorzee: indien het een exception van het juiste type is, wordt de oplossing als juist beschouwd, terwijl alle andere exceptions (of \texttt{null} als er geen exception is) fout gerekend worden.

\subsection{Testplan}\label{subsec:oefening-zero-testplan}

Ook het testplan is een vrij eenvoudige vertaling van de structuur die we hiervoor hebben bedacht.
Uniek hier is dat dit testplan ook het volledige testplan is zoals het gebruikt wordt bij de oefening in Dodona.

\inputminted{json}{sources/division-plan.tson}

\section{Som}\label{sec:oefeningen-som}

Deze oefening is andermaal afkomstig uit het boek \emph{De Programmeursleerling}\footnote{\url{https://dodona.ugent.be/nl/courses/293/series/2556/exercises/1653208777/}}.
De oefening is interessant om te illustreren hoe commandoargumenten werken en hoe de exitcode werkt.
De opgave bestaat eruit om een reeks getallen in te lezen uit de commandoargumenten en de som ervan uit te schrijven op \texttt{stdout}.
Als een van de commandoargumenten geen geldig getal is, dan moet een foutboodschap naar \texttt{stderr} geschreven worden en moet het programma eindigen met exitcode \texttt{1}.
Bijvoorbeeld:

\begin{minted}{console}
> python ./som
0
> echo $?
0
> python ./som 1 2 3 4 5 6 7 8 9 10
55
> python ./som 1 -2 3 -4 5 -6 7 -8 9 -10
-5
> python ./som spam eggs bacon
som: ongeldige argumenten
> echo $?
1
\end{minted}

Ook hier nemen we de opgave niet op door zijn lengte en geringe nut.

\subsection{Structuur van de beoordeling en evaluatie}\label{subsec:oefening-som-structuur}

Qua structuur en evaluatie lijkt deze oefening sterk op de \emph{Echo}-oefening, met dat verschil dat we hier commandoargumenten hebben.
Om de ingediende oplossing te beoordelen, zullen we de oplossing meerdere malen uitvoeren met telkens andere commandoargumenten.
We plaatsen elk stel argumenten in een eigen context.
Bij deze opgave is er geen keuze: per context is er maximaal één stel commandoargumenten, want de \texttt{main}-functie wordt hoogstens eenmaal opgeroepen per context.
Het verwachte resultaat op \texttt{stdout} en de foutboodschappen zijn opnieuw deterministisch te berekenen op basis van de commandoargumenten, dus kunnen we de ingebouwde evaluatie van \tested{} gebruiken.

\subsection{Testplan}\label{subsec:oefening-som-testplan}

In het testplan zijn drie contexten opgenomen: één waarbij getallen gegeven zijn, één waarbij geen argumenten gegeven worden en één waar geen geldige getallen als argumenten gegeven worden.
Dit opnieuw om het testplan kort te houden;
bij gebruik in Dodona zal het testplan meer contexten bevatten.

\inputminted{json}{sources/sum-plan.tson}

\section{ISBN}\label{sec:oefening-isbn}

Een volgende oefeningen die we behandelen is de \emph{\acronym{ISBN}}-oefening.
Deze oefening is al vermeld in \cref{subsec:in-de-praktijk}, waar we besproken hebben dat we deze oefening al hebben laten oplossen door studenten.
Vanuit het oogpunt van het schrijven van oefeningen voor \tested{} is deze oefening interessant doordat het een "ingewikkeldere" oefening is, waarbij ook statements (\latin{in case} assignments) gebruikt worden.

\subsection{Opgave}\label{subsec:oefeningen-isbn-opgave}

Hieronder volgt (een fragment van) de opgave:

\begin{quote}
    \markdownInput[renderers = {
        headingOne = {\chapter*{#1}},
        headingTwo = {\section*{#1}},
        headingThree = {\subsection*{#1}},
    }, slice=opgave voorbeeld]{sources/isbn-description.md}
\end{quote}

\subsection{Structuur van de beoordeling en evaluatie}\label{subsec:oefening-isbn-structuur}

Uit de opgave volgt dat er twee functies geïmplementeerd zullen moeten worden.
Het is gebruikelijk bij Dodona om elk van deze functies in een apart tabblad te beoordelen.

De contexten in het eerste tabblad voor de functie \texttt{is\_isbn} zijn niet speciaal.
We roepen de functie \texttt{is\_isbn} per context één keer op met andere argumenten.

In het tweede tabblad voor de functie \texttt{are\_isbn} ligt de situatie iets anders.
De eerste parameter van deze functie is een lijst van potentiële \acronym{ISBN}'s.
Om de overzichtelijkheid te verbeteren willen we, zoals in het voorbeeld in de opgave, deze lijst eerst toekennen aan een variabele (een assignment) en dan de variabele gebruiken als argument voor de functie.

In \tested{} heeft een testgeval altijd één statement als invoer.
In de situatie hierboven hebben we twee statements: eerst de assignment en vervolgens de functieoproep.
We zullen dus per context twee testgevallen hebben.

Op het vlak van evaluatie is deze oefening eenvoudig: door deterministische resultaten kunnen we de ingebouwde evaluatie van \tested{} gebruiken.

\subsection{Testplan}\label{subsec:oefening-isbn-testplan}

Als testplan tonen we hier een testplan met een context uit het tweede tabblad (dus met de assignment).
Om het testplan niet te lang te maken hebben we geen context opgenomen uit het eerste tabblad, vermits deze contexten niets nieuws doen.

De expressie van de assignment is in dit geval een waarde: een lijst van elementen.
Dit is de eerste keer dat we een collectie gebruiken als waarde, dus loont het de moeite om daar even stil bij te staan.
Het gebruikt illustreert dat de types van de elementen in een collectie niet hetzelfde gegevenstype moeten hebben: deze lijst bevat tekst, getallen en booleans.
Ook dit is niet mogelijk in alle programmeertalen: \tested{} detecteert dit als een heterogene collectie.
Dit zal ervoor zorgen dat we deze oefening niet in alle programmeertalen kunnen oplossen, naast het feit dat de programmeertaal ook collecties moet ondersteunen, wat bijvoorbeeld (nog) niet het geval is in C\@.

\inputminted{json}{sources/isbn-plan.tson}

\section{EqualChecker}\label{sec:oefening-equal}

Als laatste oefeningen bekijken we een oefeningen die gebruik maakt van een klasse, om te illustreren hoe objectgerichte oefeningen ook mogelijk zijn.

\subsection{Opgave}\label{subsec:oefeningen-equal-opgave}

Hieronder volgt de opgave:

\begin{quote}
    \markdownInput[renderers = {
    headingOne = {\chapter*{#1}},
    headingTwo = {\section*{#1}},
    headingThree = {\subsection*{#1}},
    }]{sources/equal-description.md}
\end{quote}

\subsection{Structuur van de beoordeling en evaluatie}\label{subsec:oefening-equal-structuur}

Bij het evalueren van deze oefeningen zullen we opnieuw meerdere statements nodig hebben: een statement om een instantie van de klasse \texttt{EqualCheck} te initialiseren, gevolgd door een statement die een methode van het aangemaakte object oproept.
Hier hebben we de keuze gemaakt dat we per context drie testgevallen zullen hebben: één om het object te maken en twee die een methode oproepen.

De evaluatie gebeurt met de ingebouwde evaluatie van \tested{}.

\subsection{Testplan}\label{subsec:oefening-equal-testplan}

Als testplan tonen we hier een testplan met één context, die zoals gezegd drie testgevallen heeft.
Nieuwe elementen in dit testplan zijn:

\begin{itemize}
    \item In het eerste testgeval gebruiken we een assignment met een constructor om het object aan te maken.
    In het testplan wordt een constructor voorgesteld als een speciale functie, waarbij de naam van de functie de naam van de klasse is.
    Ook het type van de variabele die we maken bij de assignment is speciaal: het gaat hier namelijk niet om een ingebouwd type van \tested{}, maar om de klasse \texttt{EqualChecker}.
    Aangezien dit een apart testgeval is, gebeurt de evaluatie van dit testgeval zoals voor elke functie.
    Stel dat de constructor uitvoer genereert op \texttt{stdout}, dan zal dit testgeval fout gerekend worden.
    \item In de volgende twee testgevallen gebruiken we een \texttt{namespace}-functie.
    Als \texttt{namespace} geven we de naam van de variabele op die we hiervoor gemaakt hebben: \texttt{instance}.
    Voor het overige gebeurt dit analoog aan de functies die we tot nu toe gezien hebben.
\end{itemize}

\inputminted{json}{sources/equal-plan.tson}
    \chapter{Configuratie van een programmeertaal in TESTed}\label{ch:nieuwe-taal}

In dit hoofdstuk wordt in detail uitgelegd hoe een nieuwe programmeertaal aan TESTed toegevoegd kan worden.
We doen dat door te beschrijven hoe de programmeertaal C aan TESTed toegevoegd is.
Dit hoofdstuk sluit qua vorm en stijl dan ook dichter aan bij een handleiding.
Enkele nuttige links en verwijzingen hierbij zijn:

\begin{itemize}
    \item Bestaande configuraties: \url{https://github.com/dodona-edu/universal-judge/tree/new-master/judge/src/tested/languages}
    \item Een versie van (een deel van) dit hoofdstuk in Markdown, wat gemakkelijker is in gebruik (voor bijvoorbeeld het kopiëren van code): \url{https://github.com/dodona-edu/universal-judge/blob/new-master/thesis/src/c-language.md}
    \item \Cref{ch:echo-oefening,ch:echo-function-oefening} bevatten volledige oefeningen: de opgave, het testplan en de gegenereerde code.
    Dat laatste kan nuttig zijn om ook het concrete resultaat te zien van de sjablonen.
\end{itemize}

\section{TESTed lokaal uitvoeren}\label{sec:tested-lokaal-uitvoeren}

Tijdens het configureren van een programmeertaal is het nuttig om TESTed lokaal uit te voeren, zonder daarvoor het volledige Dodona-platform te moeten uitvoeren.
Buiten de \english{dependencies} voor de bestaande programmeertalen is TESTed een Python-package, die op de normale manier uitgevoerd kan worden.

\subsection{De broncode}\label{subsec:de-broncode}

Na het klonen van de repository van TESTed beschikken we over volgende mappenstructuur:

\inputminted{text}{code/tested-dir.txt}

Merk op dat dit de toestand is op het moment van het schrijven van deze tekst.
Het is te voorzien dat in een later stadium alles behalve de mappen \texttt{judge} en \texttt{exercise} verhuizen naar een andere repository.
In dit hoofdstuk interesseren we ons enkel in die mappen, dus we voorzien geen grote problemen.

\subsection{Dependencies}\label{subsec:dependencies}

De dependencies van TESTed zelf zijn opgelijst is een \texttt{requirements.txt}-bestand, zoals gebruikelijk is bij Python-projecten.
Vereisten voor het uitvoeren van tests staan in \texttt{requirements-test.txt}.
TESTed gebruikt Python 3.8 of later.
Het installeren van deze vereisten gebeurt op de gebruikelijke manier:

\begin{minted}{console}
> pip install -r requirements.txt
\end{minted}

Voor de programmeertalen zijn volgende dependencies nodig:

\begin{description}
    \item[Python] Er zijn geen dependencies nodig voor het beoordelen van Python-oplossingen, buiten dat \texttt{python} beschikbaar moet zijn in de \texttt{PATH}.
    \item[Java] TESTed vereist Java 11, maar heeft verder geen dependencies.
                De commando's \texttt{javac} en \texttt{java} moeten beschikbaar zijn in de \texttt{PATH}.
    \item[Haskell] Voor Haskell is \acronym{GHC} 8.6 or later nodig.
    Daarnaast is \texttt{aeson} nodig.
    Beiden moeten globaal beschikbaar zijn in de \texttt{PATH}.
\end{description}

Merk op dat de dependencies voor de programmeertalen optioneel zijn.
Om bijvoorbeeld enkel Python-oplossingen te beoordelen zijn geen andere dependencies nodig.

Voor de programmeertaal C gaan we gebruik maken van \acronym{GCC}, waarbij versie 8.1 of later nodig is.

TESTed werkt op elk besturingssysteem dat ondersteund wordt door Python.
Sommige dependencies, zoals \texttt{gcc}, vragen wel meer moeite om te installeren op Windows.\footnote{Gebruikers op Windows kunnen Min\acronym{GW} of \acronym{MSYS}2 proberen.}

\subsection{Uitvoeren}\label{subsec:uitvoeren}

We gaan er voor de rest van het hoofdstuk vanuit dat commando's uitgevoerd worden in de map \texttt{judge/src}.

Er zijn twee manieren om TESTed uit te voeren.
Ten eerste is er de "gewone" manier;
dit is ook hoe Dodona TESTed uitvoert.
Bij het uitvoeren op deze manier zal TESTed een configuratie lezen van \texttt{stdin} en zal het resultaat van de beoordeling in Dodona-formaat uitgeschreven worden naar \texttt{stdout}.

\begin{minted}{console}
> python -m tested
\end{minted}

Bij het configureren van een programmeertaal of het werken aan TESTed is het echter nuttiger om meer uitvoer te zien en is het vervelend om telkens een configuratie te lezen vanop \texttt{stdin}.
Daarom is er een tweede manier:

\begin{minted}{console}
> python -m tested.manual
\end{minted}

Deze uitvoer verschilt op een aantal vlakken van de gewone uitvoering:

\begin{enumerate}
    \item Er wordt geen configuratie gelezen van \texttt{stdin}.
    De configuratie is gedefinieerd in de code zelf en gebruikt een van de oefeningen die in de map \texttt{exercise} zitten.
    \item Er worden, naast de resultaten van de beoordeling, logs uitgeschreven naar \texttt{stdout} die aangeven wat TESTed doet.
    Als er bijvoorbeeld een fout optreedt tijdens het compileren zullen deze logs nuttig zijn: zo wordt uitgeschreven welk commando TESTed exact uitvoert voor de compilatie en ook in welke map dat gebeurt.
    \item De configuratie is zo opgesteld dat de werkmap van de judge de map \texttt{workdir} zal zijn.
    Dit laat toe om de gegenereerde code te inspecteren.
\end{enumerate}

\section{Algemeen stappenplan voor het configureren van een programmeertaal}\label{sec:algemeen-stappenplan-voor-het-configureren}

Het configureren van een programmeertaal in TESTed bestaat uit drie grote onderdelen:

\begin{enumerate}
    \item Het configuratiebestand, met enkele opties voor de programmeertaal.
    \item De configuratieklasse, met de meer dynamische opties, zoals het compilatiecommando.
    \item De sjablonen, die gebruikt worden om code te genereren.
\end{enumerate}

We overlopen nu elk onderdeel in functie van de programmeertaal C\@.

\section{De programmeertaal C}\label{sec:de-programmeertaal-c}

\markdownInput[shiftHeadings=2]{c-language.md}

\section{Hoe lang duurt het implementeren van een programmeertaal?}\label{sec:hoe-lang-duurt-het-implementeren-van-een-programmeertaal?}

TODO

\section{Stabiliteit van TESTed}\label{sec:stabiliteit-van-tested}

Met stabiliteit wordt hier bedoeld hoe flexibel TESTed is bij het toevoegen van een nieuwe programmeertaal.
Dit is een maat van hoeveel er aan de kern van TESTed gewijzigd moet worden bij het toevoegen van een nieuwe programmeertaal.

We kunnen de interne werking van TESTed in grote lijnen verdelen in drie onderdelen:

\begin{itemize}
    \item De interface naar de oefeningen toe.
    Hieronder vallen vooral het testplan en het serialisatieformaat.
    \item De interface naar de programmeertalen toe.
    Hieronder vallen vooral het configuratiebestand, de configuratieklasse en de verplichte sjablonen.
    \item Het interne deel, waar het testplan uitgevoerd wordt, de ingediende oplossingen beoordeeld worden en de resultaten naar Dodona geschreven worden.
\end{itemize}

De grenzen tussen deze onderdelen zijn niet strikt: zo zal de configuratieklasse ook gebruikt worden in het interne deel tijdens de uitvoering van het testplan.

We willen opmerken dat de focus in deze thesis vooral lag op de interface naar de oefeningen toe (dus het testplan en het serialisatieformaat).
We verwachten dan ook niet dat het toevoegen van een programmeertaal grote onverwachte wijzigingen vereist in het testplan of het serialisatieformaat.
We spreken van onverwachte wijzigingen, omdat sommige delen van TESTed juist voorzien zijn op wijzigingen.
Een voorbeeld zijn de gegevenstypes uit het serialisatieformaat.
Daar is het juist de bedoeling dan nieuwe programmeertalen bijkomende geavanceerde gegevenstypes toevoegen indien nodig.
Als een nieuwe programmeertaal bijvoorbeeld ondersteuning wil bieden voor oneindige generators (bijvoorbeeld een functie die getallen blijft teruggeven, dan is dat mogelijk.
Er zullen ook geen wijzigingen nodig zijn aan de bestaande programmeertalen.

Het omgekeerde is ook waar: we verwachten dat er wijzigingen nodig kunnen zijn aan de configuratieklasse en/of de sjablonen bij nieuwe programmeertalen.
Een stabiele interface tussen de programmeertalen enerzijds en de kern van TESTed anderzijds is geen doel in deze thesis.
Hierbij wel twee nuances:

\begin{itemize}
    \item Hoewel er een poging gedaan is om de configuratieklasse flexibel te maken, is de functionaliteit ervan vooral voorzien op de ondersteunde programmeertalen en de programmeertalen die er op lijken.
    Het is ook op voorhand moeilijk in te schatten welke functionaliteit nodig zal zijn voor nieuwe programmeertalen.
    \item De wijzigingen zullen naar verwachting vooral in de configuratieklasse zijn.
    De methodes om de code te compileren en uit te voeren krijgen momenteel als argumenten de nodige informatie.
    Het is goed mogelijk dat een nieuwe programmeertaal meer informatie nodig zal hebben.
    Een ander aspect waar wijzigingen mogelijk zijn is welke informatie beschikbaar wordt gesteld aan de sjablonen.
    Ook hier is het mogelijk dat een programmeertaal meer informatie nodig heeft dan de informatie die TESTed momenteel aanreikt.
\end{itemize}

De ontwikkeling van TESTed vond grotendeels plaats met drie programmeertalen: Python, Java en Haskell.
C is pas in een later stadium toegevoegd, waar de meeste functionaliteit van TESTed reeds bestond.
Bij het configureren van C zijn wijzigingen nodig geweest aan de configuratieklasse.
Een voorbeeld is dat bij de methode voor het uitvoeren van de code nu ook het volledige pad naar de map waarin het uitvoeren gebeurt een argument is.
C compileert naar een uitvoerbaar bestand, wat tot dan nog niet gebeurde (Haskell gebruikte nog \texttt{runhaskell}).
Om het uitvoerbare bestand uit te voeren is een absoluut pad naar dat uitvoerbaar bestand nodig, maar die informatie was niet voorhanden in de methode.

    \chapter{Beperkingen en toekomstig werk}\label{ch:beperkingen-en-toekomstig-werk}

Wat kunnen we al en vooral wat niet?
Waar kan nog aan gewerkt worden?

Korte samenvatting

\section{Performance}

-> Uitleg over eerste implementatie met jupyter kernels
-> Uitleg over verschillende stadia van codegeneratie (alles apart -> zoveel mogelijk samen)

\section{Functies}

-> Dynamisch testplan
-> Herhaalde uitvoeringen

    % Allow a bit more space
    \emergencystretch=1em
    
    \printbibliography
    
    \appendix
    
    \chapter{Mapstructuur na uitvoeren}\label{ch:mapstructuur-na-uitvoeren}

Het eerste codefragment toont de mapstructuur van werkmap na het uitvoeren van een beoordeling voor een ingediende oplossing in Python.
In de map common zit alle testcode en de gecompileerde bestanden voor alle contexten.
Voor elke context worden de gecompileerde bestanden gekopieerd naar een andere map, bijvoorbeeld \texttt{context\_0\_1}, wat de map is voor context 1 van tabblad 0 van het testplan.
Het tweede codefragment toont de mapstructuur na het uitvoeren van een geprogrammeerde evaluatie.

\inputminted{text}{code/dirs-python-solution.txt}

\inputminted{text}{code/dirs-python-eval.txt}

    %! Suppress = EscapeHashOutsideCommand
\chapter{Echo-oefening}\label{ch:echo-oefening}

In deze bijlage is een volledige oefening beschikbaar, met opgave, testplan en de gegenereerde code.
Om de codefragmenten enigszins kort te houden, zijn er maar twee contexten in de oefening.

\section{Opgave}\label{sec:echo-opgave}

De opgave van deze oefening luidt als volgt:

%! Suppress = TooLargeSection
\begin{quote}
    \markdownInput[renderers = {
        headingOne = {\chapter*{#1}},
        headingTwo = {\section*{#1}},
        headingThree = {\subsection*{#1}},
    }]{sources/echo/description.md}
\end{quote}

\section{Testplan}\label{sec:echo-testplan}

Een testplan met twee testgevallen, elk in een andere context:

\inputminted{json}{sources/echo/two.tson}

\section{Python}\label{sec:echo-python}

\subsection{Oplossing}\label{subsec:echo-python-oplossing}

\inputminted{python}{sources/echo/correct.py}

\subsection{Gegenereerde code}\label{subsec:echo-python-gegenereerde-code}

De testcode die gegenereerd wordt bij het uitvoeren van bovenstaande testplan.
Deze code is lichtjes aangepast: overtollige witruimte is verwijderd.
Anders is de code identiek aan de door \tested{} gegenereerde code.

In Python is geen selector nodig, waardoor de gegenereerde code identiek is voor batchcompilatie en voor contextcompilatie.
In beide gevallen worden twee bestanden gegenereerd:

\begin{enumerate}
    \item \texttt{context\_0\_0.py}
    \item \texttt{context\_0\_1.py}
\end{enumerate}

\subsubsection{\texttt{context\_0\_0.py}}

%! Suppress = EscapeUnderscore
\inputminted{python}{sources/echo/context_0_0.py}

\subsubsection{\texttt{context\_0\_1.py}}

%! Suppress = EscapeUnderscore
\inputminted{python}{sources/echo/context_0_1.py}

\subsection{Uitvoeren}\label{subsec:echo-python-uitvoeren}

Bij het uitvoeren worden de gegenereerde bestanden uitgevoerd.
In beide gevallen (batchcompilatie en contextcompilatie) wordt de juiste context uitgevoerd:

%! Suppress = EscapeUnderscore
\begin{minted}{console}
> python context_0_0.py
<uitvoer context_0_0>
> python context_0_1.py
<uitvoer context_0_1>
\end{minted}

\section{Java}\label{sec:echo-java}

\subsection{Oplossing}\label{subsec:echo-java-oplossing}

\inputminted{java}{sources/echo/correct.java}

\subsection{Gegenereerde code}\label{subsec:echo-java-gegenereerde-code}

De testcode die gegenereerd wordt bij het uitvoeren van bovenstaande testplan.
Deze code is lichtjes aangepast: overtollige witruimte is verwijderd.
Anders is de code identiek aan de door \tested{} gegenereerde code.

Bij batchcompilatie worden alle testgevallen in één keer gecompileerd.
Daarvoor wordt het bestand \texttt{Selector.java} gegenereerd.
Bij contextcompilatie wordt geen \texttt{Selector.java} gegenereerd, maar zijn de bestanden anders identiek.
Volgende bestanden werden gegenereerd (met \texttt{Selector.java} dus enkel in batchcompilatie):

\begin{enumerate}
    \item \texttt{Context00.java}
    \item \texttt{Context01.java}
    \item \texttt{Selector.java}
\end{enumerate}

\subsubsection{\texttt{Context00.java}}

\inputminted{java}{sources/echo/Context00.java}

\subsubsection{\texttt{Context01.java}}

\inputminted{java}{sources/echo/Context01.java}

\subsubsection{\texttt{Selector.java}}

\inputminted{java}{sources/echo/Selector.java}

\subsection{Uitvoeren}\label{subsec:echo-java-uitvoeren}

Bij batchcompilatie wordt \texttt{Selector.java} gecompileerd, wat leidt tot een reeks \texttt{.class}-bestanden.
Vervolgens wordt de selector tweemaal uitgevoerd:

%! Suppress = EscapeUnderscore
%! Suppress = LineBreak
\begin{minted}{console}
> ./java -cp . Selector Context00
<uitvoer context_0_0>
> ./java -cp . Selector Context01
<uitvoer context_0_1>
\end{minted}

Een opmerking hierbij teneinde verwarring tegen te gaan: in de commando's hierboven wordt met het argument \texttt{Selector} de klasse meegegeven aan Java die uitgevoerd moet worden.
Het argument \texttt{Context0*} is een programma-argument zoals gebruikelijk.

Bij contextcompilatie wordt elke context afzonderlijk gecompileerd, wat ook leidt tot een reeks \texttt{.class}-bestanden.
Deze keer worden de contexten zelf uitgevoerd:

%! Suppress = EscapeUnderscore
%! Suppress = LineBreak
\begin{minted}{console}
> java -cp . Context00
<uitvoer context_0_0>
> java -cp . Context01
<uitvoer context_0_1>
\end{minted}

\section{C}\label{sec:echo-c}

\subsection{Oplossing}\label{subsec:echo-c-oplossing}

\inputminted{c}{sources/echo/correct.c}

\subsection{Gegenereerde code}\label{subsec:echo-c-gegenereerde-code}

De testcode die gegenereerd wordt bij het uitvoeren van bovenstaande testplan.
Deze code is lichtjes aangepast: overtollige witruimte is verwijderd.
Anders is de code identiek aan de door \tested{} gegenereerde code.

Bij batchcompilatie worden alle testgevallen in één keer gecompileerd.
Daarvoor wordt het bestand \texttt{selector.c} gegenereerd.
Bij contextcompilatie wordt geen \texttt{selector.c} gegenereerd, maar zijn de bestanden anders identiek.
Volgende bestanden werden gegenereerd (met \texttt{selector.c} dus enkel in batchcompilatie):

\begin{enumerate}
    \item \texttt{context\_0\_0.c}
    \item \texttt{context\_0\_1.c}
    \item \texttt{selector.c}
\end{enumerate}

\subsubsection{\texttt{context\_0\_0.c}}

%! Suppress = EscapeUnderscore
\inputminted{c}{sources/echo/context_0_0.c}

\subsubsection{\texttt{context\_0\_1.c}}

%! Suppress = EscapeUnderscore
\inputminted{c}{sources/echo/context_0_1.c}

\subsubsection{\texttt{selector.c}}

\inputminted{c}{sources/echo/selector.c}

\subsection{Uitvoeren}\label{subsec:echo-c-uitvoeren}

Bij batchcompilatie wordt \texttt{selector.c} gecompileerd, wat leidt tot een uitvoerbaar bestand \texttt{selector}.
Dit laatste bestand wordt dan tweemaal uitgevoerd:

%! Suppress = EscapeUnderscore
\begin{minted}{console}
> ./selector context_0_0
<uitvoer context_0_0>
> ./selector context_0_1
<uitvoer context_0_1>    
\end{minted}

Bij contextcompilatie wordt elke context afzonderlijk gecompileerd, wat leidt tot twee uitvoerbare bestanden: \texttt{context\_0\_0} en \texttt{context\_0\_1}.
Deze uitvoerbare bestanden worden vervolgens uitgevoerd:

%! Suppress = EscapeUnderscore
\begin{minted}{console}
> ./context_0_0
<uitvoer context_0_0>
> ./context_0_1
<uitvoer context_0_1>
\end{minted}

    \chapter{Echofunctie-oefening}\label{ch:echo-function-oefening}

In deze bijlage is een volledige oefening beschikbaar, met opgave, testplan en de gegenereerde code.
Om de codefragmenten enigszins kort te houden, is er maar een context, maar wel met twee testgevallen in dezelfde context.

\section{Opgave}\label{sec:echo-function-opgave}

De opgave van deze oefening luidt als volgt:

\begin{quote}
    \markdownInput[renderers = {
        headingOne = {\chapter*{#1}},
        headingTwo = {\section*{#1}},
        headingThree = {\subsection*{#1}},
    }]{../../exercise/echo-function/description/description.nl.md}
\end{quote}

\section{Testplan}\label{sec:echo-function-testplan}

Een testplan met een context met twee testgevallen:

\inputminted{json}{../../exercise/echo-function/evaluation/one-testcase.tson}

\section{Oplossing}\label{sec:echo-function-oplossing}

We voeren het testplan uit met deze oplossing:

\inputminted{c}{../../exercise/echo-function/solution/correct.c}

\section{Gegenereerde code}\label{sec:echo-function-gegenereerde-code}

De gegenereerde code die gegenereerd wordt bij het uitvoeren van bovenstaande testplan.
Deze code is lichtjes aangepast: overtollige witruimte is verwijderd.
Anders is de code identiek aan de door TESTed gegenereerde code.

\subsection{Batchcompilatie}\label{subsec:echo-function-batchcompilatie}

In batchcompilatie worden alle testgevallen in één keer gecompileerd.
Volgende bestanden werden gegenereerd:

\begin{enumerate}
    \item \texttt{context\_0\_0.c}
    \item \texttt{selector.c}
\end{enumerate}

\subsubsection{\texttt{context\_0\_0.c}}

\inputminted{c}{code/echo-function-c/context_0_0.c}

\subsubsection{\texttt{selector.c}}

\inputminted{c}{code/echo-function-c/selector.c}

\subsubsection{Uitvoeren}

Bij het compileren wordt \texttt{selector.c} gecompileerd, wat leidt tot een uitvoerbaar bestand \texttt{selector}.
Dit laatste bestand wordt dan uitgevoerd:

\begin{minted}{console}
> ./selector context_0_0
<uitvoer context_0_0>
\end{minted}

\subsection{Contextcompilatie}\label{subsec:echo-function-contextcompilatie}

Bij contextcompilatie wordt slechts een bestand gegenereerd:

\begin{enumerate}
    \item \texttt{context\_0\_0.c}
\end{enumerate}

De inhoud van dit bestanden is identiek aan de inhoud van hetzelfde bestand bij batchcompilatie.

Dit bestand wordt gecompileerd, wat leidt tot een uitvoerbaar bestand: \texttt{context\_0\_0}.
Dit bestand wordt vervolgens uitgevoerd:

\begin{minted}{console}
> ./context_0_0
<uitvoer context_0_0>
\end{minted}

\end{document}
