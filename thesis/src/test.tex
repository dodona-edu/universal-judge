\chapter{Tests}\label{ch:tests}

\lettrine{D}{it} is de inleiding van mijn thesis.
Hallo aan iedereen!
Omdat we met een initiaal werken, is het aangewezen dat de eerste alinea redelijk wat tekst bevat.


Verder is dit aan de orde:
\[
    y = 5 + 6
\]

\textsc{Hallo ik ben kleinkapitaal!}

Hallo, dit is tekst met code: \texttt{dit is dan de code}.

\lettrine{Z}{eekomkommers} (Holothuroidea) vormen een groep van ongewervelde dieren die behoren tot de klasse van stekelhuidigen.
De meeste soorten hebben een langwerpig en worstvormig lichaam dat zowel aan de voor- als achterzijde stomp eindigt.
Er zijn ook vormen met een sliertige of een bolle lichaamsvorm.

Zeekomkommers zijn net als andere stekelhuidigen uitgesproken zeedieren;
ze kunnen niet overleven in zoet water of op het land.
De verschillende soorten komen voor van de getijdenzone tot in de diepzee.
De meeste soorten leven op de bodem, waar ze zich kruipend voortbewegen.
Ook zijn er strikt gravende soorten bekend en enkele soorten zijn zelfs goede zwemmers.

De meeste groepen van stekelhuidigen hebben een stervormig lichaam (radiaal symmetrisch).
Zeekomkommers wijken hier uitwendig sterk van af.
De inwendige anatomie vertoont echter wel grote gelijkenissen met andere stekelhuidigen zoals zeesterren, slangsterren en zee-egels.
Er zijn ongeveer 1.700 verschillende soorten zeekomkommers beschreven, waarvan er slechts twee weleens langs de kust van België en Nederland worden gevonden.

Zeekomkommers leven van kleine, zwevende voedingsdeeltjes of kleine organismen die uit het water worden gefilterd.
Ze hebben hiertoe vaak duidelijke tentakels rond hun mond waarmee ze hun voedsel verzamelen.
Sommige vertegenwoordigers zijn carnivoor en eten kleine diertjes, andere leven enkel van organische deeltjes in de modder van de zeebodem.
De dieren worden zelf gegeten door verschillende andere dieren zoals roofvissen en zeeschildpadden.
Sommige soorten kunnen een bepaald deel van het darmstelsel uitstoten ter verdediging.
Deze sliertige structuren zijn plakkerig en bovendien giftig.

Verschillende soorten worden vooral in Azië gebruikt in gerechten of worden gebruikt als een traditioneel medicijn.
Zeekomkommers worden vaak eerst gedroogd voor ze worden verwerkt als voedsel of in medicijnen.

Zeekomkommers lijken in het geheel niet op de andere stekelhuidigen maar hebben inwendig een sterk vergelijkbare bouw.
Een van de belangrijkste verschillen is het feit dat de zeekomkommer een voor- en een achterzijde heeft.
Feitelijk ligt een zeekomkommer altijd op zijn zijkant.
Bij de zee-egels en de zeesterren hebben de dieren een boven- en onderzijde.
De anus is bij deze groepen aan de bovenzijde gelegen en de mondopening aan de onderzijde.

Rond de slokdarm is in het lichaam van de zeekomkommer een harde, ringvormige structuur aanwezig die bestaat uit sterk verkalkt weefsel.
Deze ring heeft niet zozeer een verstevigende functie zoals botten van hogere dieren maar dient als aanhechtingspunt voor verschillende lichaamsspieren.
Kalkhoudende structuren komen binnen de stekelhuidigen veel voor maar hebben meestal slechts een beschermende functie.
De vorm van de kalkring verschilt per groep, soorten uit de familie Paracucumidae hebben een 'kale' kalkring zonder aanhangsels, die uit de familie Sclerodactylidae hebben enkelvoudige aanhangsels aan de kalkring en die uit de familie Placothuriidae bezitten lange en gepaarde aanhangsels.

Zeekomkommers hebben van alle stekelhuidigen het hoogst ontwikkelde vaatstelsel, ze gebruiken lichaamsvloeistof als bloedvloeistof en voor het transport van voedingsstoffen en gassen door het lichaam.
Het zijn de enige stekelhuidigen die een met meerdere kamers gevulde lichaamsholte hebben die sterk lijkt op het hart van hogere gewervelde dieren.
De endeldarm van sommige soorten bevat een speciaal orgaan dat dient ter verdediging.
Dit orgaan ontspruit aan de waterlongen en wordt wel aangeduid met de buizen van Cuvier, zie ook onder het kopje vijanden en verdediging.

De vijfvoudig straalsgewijze lichaamsbouw aan de buitenzijde van het lichaam komt bij de meeste soorten duidelijk tot uiting in het watervaatstelsel.
Dit is een stelsel van kanalen, blaasjes en lichaamsuitstulpingen.
Het watervaatstelsel ontspruit uit het ringvormige, verkalkte kanaal rondom de slokdarm, het ringkanaal, en bestaat uit vijf radiaalkanalen.
Ieder radiaalkanaal staat in verbinding met een van de vijf rijen kleine zakvormige uitstulpingen aan de buitenzijde die vaak -maar niet altijd- op een evenredige afstand van elkaar gelegen zijn.
Deze uitstulpingen worden de voetjes (podia) genoemd.

Bij de meeste stekelhuidigen, zoals zeesterren en zee-egels, staat het watervaatstelsel in verbinding met het omringende zeewater.
Deze verbinding wordt het steenkanaal genoemd omdat de wanden van het kanaal verkalkt zijn.
De eigenlijke verbinding is de madreporiet of zeefplaat, dit is een verhard en verkalkt plaatje met kleine poriën waardoor zeewater het lichaam in wordt gezogen.
Zeekomkommers hebben tot wel 100 steenkanalen, maar deze eindigen vrijwel altijd binnen in de lichaamsholte en staan niet in verbinding met de huid.
De dieren gebruiken dus geen zeewater als vloeistof in het watervaatstelsel zoals veel andere stekelhuidigen, maar lichaamsvocht.
Alleen van soorten uit de orde Elasipodida zijn uitwendige madreporieten bekend.

De functie van de inwendig gelegen steenkanalen en zeefplaten is niet geheel duidelijk.
Waarschijnlijk dienen de structuren om de waterhuishouding van het lichaamsvocht en bloedvloeistof te reguleren.
Uit het ringkanaal ontspruiten ook verschillende blaasjes van Poli.
Deze dienen waarschijnlijk eveneens om het watervaatstelsel te ondersteunen als opslagvat.
Uiterlijk lijken de blaasjes op de ampullen die aan de voetjes gebonden zijn, maar ze zijn groter en in het lichaam gelegen.
De blaasjes van Poli zijn niet direct verbonden met de radiaalkanalen zoals bij de ampullen het geval is.
Het aantal blaasjes kan variëren van een enkel blaasje tot meer dan vijftig.

Sommige zeekomkommers ademen met behulp van hun tentakels.
Dit zijn in feite omgebouwde buisvoetjes die hun oorspronkelijke functie hebben verloren.
Alle andere soorten ademen met behulp van waterlongen.
Deze hebben eenzelfde functie als de kieuwen van vissen.
Waterlongen komen alleen voor bij zeekomkommers en zijn niet bekend bij andere dieren, ook niet bij andere stekelhuidigen.

De waterlongen zijn sterk vertakte structuren die ontspruiten aan de achterzijde van het lichaam, ze staan in verbinding met de endeldarm.
De waterlongen zijn in feite uitlopers van de cloaca die zich ontwikkeld hebben tot ademhalingsorganen.
Ze vullen een groot deel van de lichaamsholte en lopen door tot vooraan in het lichaam.
De zeekomkommer haalt adem door de waterlongen vol water te zuigen via de anus en de gespierde endeldarm.
In de waterlongen vindt de gasuitwisseling plaats waarbij zuurstof wordt opgenomen en afvalstoffen worden afgegeven.
Sommige zeekomkommers zuigen meerdere keren per minuut zeewater in en uit en anderen 'ademen' heel diep in om vervolgens het water er in één keer weer uit te persen.
Hierbij wordt zuurstof afgegeven aan de lichaamsvloeistof die door het lichaam stroomt.
Vele kleine trilharen die de binnenzijde van de lichaamsholte bekleden zorgen dat de stroming van de lichaamsvloeistof de juiste kant op gaat

Zeekomkommers hebben geen hart of een vergelijkbaar orgaan en ze hebben ook geen echt bloed.
Wel stroomt er een vloeistof door het lichaam die afgescheiden is van het watervaatstelsel en voedingsstoffen naar de lichaamscellen brengt.
Deze kleurloze vloeistof wordt de bloedvloeistof genoemd en bestaat uit lichaamsvocht dat verrijkt is met zuurstof, afkomstig van het ademhalingsapparaat.
De ruimtes in het lichaam waar bloedvloeistof aanwezig is worden de bloedlacunes genoemd en het geheel van kanalen wordt met bloedlacunestelsel aangeduid.
De verbindingskanalen tussen de bloedlacunes lijken op de aderen van hogere dieren, maar hebben geen eigen vaatwand.
Het zijn feitelijk kanaalachtige ruimtes tussen de verschillende lichaamsweefsels waarin de bloedvloeistof stroomt.

Door samentrekkingen van de lengtespieren van de darm wordt de bloedvloeistof in het lichaam rondgepompt.
De darmwand is voorzien van vele zeer fijn vertakte bloedlacunes die voedingsstoffen onttrekken uit de darm.
De voedingsstoffen worden door vijf kanalen verdeeld over het lichaam, deze kanalen zijn onder de radiaalkanalen van het watervaatstelsel gelegen.
De uiteinden van deze bloedlacunes zijn eveneens vertakt maar ze eindigen doodlopend zodat het geheel geen ringvormig, gesloten systeem is.

"Dit is duidelijk voor iedereen".